\ifdefined\niveldos\else
\documentclass[12pt,letterpaper]{report}
 
%Russian-specific packages
%--------------------------------------
\usepackage[T2A]{fontenc}
\usepackage[utf8]{inputenc}
\usepackage[russian]{babel}
\usepackage[mathscr]{euscript}
\usepackage{mathrsfs}
\usepackage{amsmath,amsthm,amssymb,latexsym,amsfonts}
\usepackage{showkeys}
\usepackage{pythonhighlight}
\usepackage{mdframed}
\usepackage{lipsum}
\usepackage{soul}
\usepackage{amsmath,amssymb}
\usepackage{parskip}
\usepackage{graphicx}
\usepackage{mathtools}
\usepackage[usestackEOL]{stackengine} 
\usepackage{tocloft}
\usepackage{xcolor}
\usepackage{hyperref}
\usepackage{tikz}
\usepackage{thmtools}
\usepackage{amsthm}
\usepackage{mathtools}
\usepackage{pdfcomment}
\usepackage{soul}
\usepackage{blindtext}
\usepackage{todonotes}
\usepackage{authblk}
\graphicspath{ {./images/} }
%%

\DeclarePairedDelimiter\abs{\lvert}{\rvert}%
\DeclarePairedDelimiter\norm{\lVert}{\rVert}%

% Swap the definition of \abs* and \norm*, so that \abs
% and \norm resizes the size of the brackets, and the 
% starred version does not.
\makeatletter
\let\oldabs\abs
\def\abs{\@ifstar{\oldabs}{\oldabs*}}
%
\let\oldnorm\norm
\def\norm{\@ifstar{\oldnorm}{\oldnorm*}}
\makeatother

\newtheorem{theorem}{Теорема}
\newtheorem*{theorem-non}{Теорема}
\newtheorem{lemma}[theorem]{Лемма}
\theoremstyle{definition}
\newtheorem*{conj}{Определение}
\declaretheorem[numbered=no]{definition}

% Definindo novas cores
\definecolor{verde}{rgb}{0.25,0.5,0.35}
\definecolor{jpurple}{rgb}{0.5,0,0.35}
\definecolor{darkgreen}{rgb}{0.0, 0.2, 0.13}
%\definecolor{oldmauve}{rgb}{0.4, 0.19, 0.28}
% Configurando layout para mostrar codigos Java
\usepackage{listings}

\newcommand{\estiloJava}{
\lstset{
    language=Java,
    basicstyle=\ttfamily\small,
    keywordstyle=\color{jpurple}\bfseries,
    stringstyle=\color{red},
    commentstyle=\color{verde},
    morecomment=[s][\color{blue}]{/**}{*/},
    extendedchars=true,
    showspaces=false,
    showstringspaces=false,
    numbers=left,
    numberstyle=\tiny,
    breaklines=true,
    backgroundcolor=\color{cyan!10},
    breakautoindent=true,
    captionpos=b,
    xleftmargin=0pt,
    tabsize=2
}}

\newcommand{\estiloR}{
  \lstset{ %
    language=R,                     % the language of the code
    basicstyle=\footnotesize,       % the size of the fonts that are used for the code
    numbers=left,                   % where to put the line-numbers
    numberstyle=\tiny\color{gray},  % the style that is used for the line-numbers
    stepnumber=1,                   % the step between two line-numbers. If it's 1, each line
                                    % will be numbered
    numbersep=5pt,                  % how far the line-numbers are from the code
    backgroundcolor=\color{white},  % choose the background color. You must add \usepackage{color}
    showspaces=false,               % show spaces adding particular underscores
    showstringspaces=false,         % underline spaces within strings
    showtabs=false,                 % show tabs within strings adding particular underscores
    frame=single,                   % adds a frame around the code
    rulecolor=\color{black},        % if not set, the frame-color may be changed on line-breaks within not-black text (e.g. commens (green here))
    tabsize=2,                      % sets default tabsize to 2 spaces
    captionpos=b,                   % sets the caption-position to bottom
    breaklines=true,                % sets automatic line breaking
    breakatwhitespace=false,        % sets if automatic breaks should only happen at whitespace
    title=\lstname,                 % show the filename of files included with \lstinputlisting;
                                    % also try caption instead of title
    keywordstyle=\color{blue},      % keyword style
    commentstyle=\color{darkgreen},   % comment style
    stringstyle=\color{red},      % string literal style
    escapeinside={\%*}{*)},         % if you want to add a comment within your code
    morekeywords={*,...}          % if you want to add more keywords to the set
}}

\newcommand{\Z}{\mathbb{Z}}
\newcommand{\Q}{\mathbb{Q}}
\newcommand{\N}{\mathbb{N}}
\newcommand{\R}{\mathbb{R}}
\newcommand{\follow}{\textbf{\textit{Следствие:}}}
\newcommand{\notice}{\underline{\textit{Замечание }}}

\newcommand*\circled[1]{\tikz[baseline=(char.base)]{
            \node[shape=circle,draw,inner sep=2pt] (char) {#1};}}

\definecolor{linkcolor}{HTML}{47528f} % цвет ссылок
\definecolor{urlcolor}{HTML}{47528f} % цвет гиперссылок
\hypersetup{pdfstartview=FitH,  linkcolor=linkcolor,urlcolor=urlcolor, colorlinks=true}
\newcommand{\RomanNumeralCaps}[1]
  {\MakeUppercase{\romannumeral #1}}
\usepackage{titlesec}
\renewcommand{\thesection}{\arabic{section}}
\renewcommand{\listtheoremname}{Определения и теоремы}
\renewcommand\qedsymbol{$\blacksquare$}
\newcommand\oast{\stackMath\mathbin{\stackinset{c}{0ex}{c}{0ex}{\ast}{\bigcirc}}}
\makeatletter
\renewenvironment{proof}[1][\proofname]{%
   \par\pushQED{\qed}\normalfont%
   \topsep6\p@\@plus6\p@\relax
   \trivlist\item[\hskip\labelsep\bfseries#1\@addpunct{.}]%
   \ignorespaces
}{%
   \popQED\endtrivlist\@endpefalse
}
\makeatother
%--------------------------------------
\DeclareMathOperator{\Mr}{M_{\mathbb{R}}}
\addtolength{\oddsidemargin}{-.875in}
	\addtolength{\evensidemargin}{-.875in}
	\addtolength{\textwidth}{1.75in}

	\addtolength{\topmargin}{-.875in}
    \addtolength{\textheight}{1.75in}
%--------------------------------------
\def\calloutsym{%
  \ensurestackMath{%
  \scalebox{1.7}{\color{red}\stackunder[0pt]{\bigcirc}{\downarrow}}}%
}
\def\calloutsymup{%
  \ensurestackMath{%
  \scalebox{1.7}{\color{red}\stackon[0pt]{\bigcirc}{\uparrow}}}%
}
\newcommand\callouttext[1]{%
  \def\stacktype{S}\renewcommand\useanchorwidth{T}\stackText%
  \stackunder{\calloutsym}{\scriptsize\Longstack{#1}}\stackMath%
}
\newcommand\callout[3][2.5pt]{%
  \def\stacktype{L}\stackMath\stackunder[#1]{#2}{\callouttext{#3}}%
}
\newcommand\callouttextup[1]{%
  \def\stacktype{S}\renewcommand\useanchorwidth{T}\stackText%
  \stackon{\calloutsymup}{\scriptsize\Longstack{#1}}\stackMath%
}
\newcommand\calloutup[3][1.5pt]{%
  \def\stacktype{L}\stackMath\stackunder[#1]{#2}{\callouttextup{#3}}%
}
%%%%%%%%%%%%%%%%%%%%%%%%%%%%%%%%%%
%                                %
%                                %
%              Title             %
%                                %
%                                %
%%%%%%%%%%%%%%%%%%%%%%%%%%%%%%%%%%
\title{Конспект лекций по математическому анализу

СПбГУ, МКН, 1 курс
\ \\
\ \\ \ \\
} 


%\thanks{Авторы: %\href{https://github.com/maxmartynov08}{maxmart%ynov08}, \href{https://github.com/K-dizzled}{K-%dizzled}, \href{https://github.com/SmnTin}{SmnT%in}, \href{https://github.com/muldrik}{muldrik}%}} 
\author{Лектор: Храбров Александр Игоревич \\
\ \\ \ \\ \ \\ \ \\ \ \\ \ \\ \ \\ \ \\
\ \\ 
\begin{flushright}
Составили: Андрей \href{https://github.com/K-dizzled}{K-dizzled} Козырев \\ Никита  \href{https://github.com/muldrik}{muldrik} Митцев \\ Максим \href{https://github.com/maxmartynov08}{maxmartynov08} Мартынов \\ Семен \href{https://github.com/SmnTin}{SmnTin} Паненков
\end{flushright}
\ \\
\ \\
\ \\
\ \\
\ \\
\ \\}
\date{осень 2020}
\begin{document}

\clearpage
%% temporary titles
% command to provide stretchy vertical space in proportion
\newcommand\nbvspace[1][3]{\vspace*{\stretch{#1}}}
% allow some slack to avoid under/overfull boxes
\newcommand\nbstretchyspace{\spaceskip0.5em plus 0.25em minus 0.25em}
% To improve spacing on titlepages
\newcommand{\nbtitlestretch}{\spaceskip0.6em}
\pagestyle{empty}
\begin{center}
\bfseries
\nbvspace[1]
\Huge
{\nbtitlestretch\huge
КОНСПЕКТ ЛЕКЦИЙ ПО МАТЕМАТИЧЕСКОМУ АНАЛИЗУ}

\nbvspace[1]
\normalsize

СПбГУ, МКН, СП, 1 курс\\
ЛЕКТОР: ХРАБРОВ АЛЕКСАНДР ИГОРЕВИЧ
\nbvspace[1]
\\
\Large СОСТАВИТЕЛИ:\\[0.5em]
\footnotesize АНДРЕЙ \href{https://github.com/K-dizzled}{K-dizzled} КОЗЫРЕВ, НИКИТА  \href{https://github.com/muldrik}{muldrik} МИТЦЕВ \\ МАКСИМ \href{https://github.com/maxmartynov08}{maxmartynov08} МАРТЫНОВ, СЕМЕН \href{https://github.com/SmnTin}{SmnTin} ПАНЕНКОВ

\nbvspace[2]

\includegraphics[width=4.0in]{./images/matan_kills.png}
\nbvspace[3]
\normalsize

\large
ОСЕНЬ 2020
\nbvspace[1]
\end{center}
\fi
%\maketitle
%%%%%%%%%%%%%%%%%%%%%%%%%%%%%%%%%%
%                                %
%                                %
%        Table of content        %
%                                %
%                                %
%%%%%%%%%%%%%%%%%%%%%%%%%%%%%%%%%%
\tableofcontents
%\listoftheorem-nons[ignore={lemma},show={conj, theorem-non}]
% To show Table of contents with
% definitions and lemmas
%----------------------------------------
% \listoftheorem-nons[ignoreall,show={lemma}]
% \listoftheorem-nons[ignoreall,show={conj}]
\newpage
\chapter{Первый семестр. Первая четверть}
\section{Множества}
\begin{conj} Множество - набор уникальных элементов \end{conj}

Множества - большие буквы $A, B,\dots$ \\
Элементы множеств - маленькие буквы $a, b,\dots$ \\
$x \in A - x$ пренадлежит $A$ \\
$x \notin A - x$ не пренадлежит $A$ \\
$\mathbb{N} = \{1, 2, 3, \dots\} \\
\mathbb{Z, Q} = \{{{m}\over{n}} : m \in \mathbb{Z}, n \in\mathbb{N}\} \\
\mathbb{R}$ - вещественные числа \\
$\mathbb{C}$ - комплексные числа \\
\begin{theorem-non} Правила Де Моргана \end{theorem-non}
    \begin{itemize}
        \item[] $A \; \setminus \; (\bigcup\limits_{\alpha \in I} B_{\alpha}) 
        = \bigcap\limits_{\alpha \in I}(A \setminus B_{\alpha})$

        \item[] $A \; \setminus \; (\bigcap\limits_{\alpha \in I} B_{\alpha}) 
        = \bigcup\limits_{\alpha \in I}(A \setminus B_{\alpha})$
    \end{itemize}
\begin{proof}
    Докажем для первой формулы. Вторая доказывается аналогично. \\
    $x \in A \; \setminus \; (\bigcup\limits_{\alpha \in I} B_{\alpha}) 
    \Longleftrightarrow \begin{cases}
        x \in A \\
        x \notin \bigcup\limits_{\alpha \in I} B_{\alpha}
    \end{cases}
    \Longleftrightarrow \begin{cases}
        x \in A \\
        x \notin B_{\alpha} \; \; $при всех$ \; \alpha
    \end{cases} 
    \Longleftrightarrow x \in A \; \setminus \; B_{\alpha}$ при всех $\alpha \in I
    \Longleftrightarrow x \in \bigcap\limits_{\alpha \in I}(A \setminus B_{\alpha})$ 
\end{proof} \newpage
\begin{theorem-non} Операции над множествами \end{theorem-non}
\begin{itemize}
    \item $A \cup B = \{x: x \in A $ или $ x \in B\}$
    \item $A \cap B = \{x: x \in A, x  \in B\}$
    \item $A \; \setminus \; B = \{x: x \in A, x  \notin B\}$
    \item $A \bigtriangleup B = (A \; \setminus \; B) \cup (B \; \setminus \; A)$
\end{itemize}
\notice - $\bigtriangleup, \cup, \cap$ - комммутативны, ассоциативны
\begin{conj} 
    Декартово произведение множеств 
    $A \times B = \{\langle a, b \rangle : a \in A; b \in B \}$ 
\end{conj}
\begin{theorem-non} \end{theorem-non}
    \begin{itemize}
        \item[] $A \cap \bigcup\limits_{\alpha \in I} B_{\alpha} =
        \bigcup\limits_{\alpha \in I}(A \cap B_{\alpha}) $

        \item[] $A \cup \bigcap\limits_{\alpha \in I} B_{\alpha} =
        \bigcap\limits_{\alpha \in I}(A \cup B_{\alpha}) $
    \end{itemize}
\begin{proof}
        $x \in A \cap \bigcup\limits_{\alpha \in I} B_{\alpha}
        \Longleftrightarrow \begin{cases}
            x \in A \\
            x \in \bigcup\limits_{\alpha \in I} B_{\alpha}
        \end{cases} \Longleftrightarrow \begin{cases}
            x \in A \\
            x \in B_{\alpha}$ для некоторых $\alpha \in I
        \end{cases}\vspace{0.5cm} \Longleftrightarrow 
        x \in A \cap B_{\alpha}$ для некоторых $\alpha \in I 
        \Longleftrightarrow 
        x \in \bigcup\limits_{\alpha \in I}(A \cap B_{\alpha})$
    \end{proof}
\begin{conj} 
    Упорядоченная пара $ \langle a, b \rangle $ - пара  ``пронумерованных'' элементов
\end{conj}
    $  \langle a, b \rangle $ = $  \langle c, d \rangle \rotatebox[origin=c]{150}{$\Longleftrightarrow$}$ 
\begin{scriptsize}
\estiloJava
\begin{lstlisting}[caption={}, label=]
    ((a == c) && (b == d))
\end{lstlisting}
\end{scriptsize}
\section{Отношения}
\begin{conj} 
    Область определения: 
    $\delta_{R} = \{x \in A: \exists y \in B, $ т.ч.$ \langle x, y \rangle  \in \mathbb{Z} \} $ 
\end{conj}

\begin{conj} 
    Область значений: 
    $\rho_{R} = \{y \in B: \exists x \in A, $ т.ч.$ \langle x, y \rangle  \in \mathbb{Z} \} $ 
\end{conj}
$\delta_{R^{-1}} = \rho_{R} \\
\rho_{R^{-1}} = \delta_{R}$

\begin{conj} 
    Композиция отношений 
\end{conj}

\begin{itemize}
    \item[] $R_1 \subset A \times B, \quad R_2 \subset B \times C, \quad R_1 \circ R_2 \subset A \times C$
\end{itemize}
\subsection*{Пример}
\begin{itemize}
    \item $\langle x, y \rangle \in R$, если x — отец y
    \item $\langle x, y \rangle \in R \circ R$, если x — дед y
    \item $\langle x, y \rangle \in R^{-1} \circ R$, если x — брат y
    \item $\delta R$ — все, у кого есть сыновья
\end{itemize}
\begin{conj} 
    Бинарным отношением $R$ называется подмножество элементов декартова произведения двух
    множеств $R \subset A \times B$
\end{conj}

\begin{itemize}
    \item[] Элементы $x \in A, y \in B$ находятся в отношении, если $  \langle x, y \rangle \in R $ (то же, что $xRy$)
    \item[] Обратное отношение $R^{-1} \subset B \times A$ 
\end{itemize}

\begin{conj}
    Отношение называется:
\end{conj}
\begin{itemize}
    \item Рефлексивным, если $xRx \; \forall x$
    \item Симметричным, если $xRy \Longrightarrow yRx$
    \item Транзитивным, если $xRy, yRz \Longrightarrow xRz$
    \item Иррефлексивным, если $\neg xRx \forall x$
    \item Антисимметричным, если $xRy, yRx \Longrightarrow x = y$
\end{itemize}

\begin{conj}
    $R$ является отношением
\end{conj}
\begin{itemize}
    \item[1.] Эквивалентности, если оно рефлексивно, симметрично и транзитивно
    \item[2.] Нестрогого частичного порядка, если оно рефлексивно, антисимметрично и транзитивно
    \item[3.] Нестрогого полного порядка, если выполняется п. $2 + \forall x, y$ либо $xRy$, либо $yRx$
    \item[4.] Строгого частичного порядка, если оно иррефлексивно и транзитивно
    \item[5.] Строгого полного порядка, если выполняется п. $4 + \forall x$, y либо $xRy$, либо $yRx$
\end{itemize}

\subsection*{Пример}
\begin{itemize}
    \item $x \equiv y \; (mod \; m)$ — отношение эквивалентности
    \item $X$ - множество$, 2^X$ — множество всех его подмножеств
    \item $\forall x, y \in 2^x : \langle x, y \rangle \in R, $ если $ x \subsetneq y$ — отношение строгого частичного порядка
    \item Лексикографический порядок на множестве пар натуральных чисел — отношение нестрогого полного порядка
\end{itemize}

\begin{conj}
    Отображение $f: A  \longrightarrow B$ 
\end{conj}
\begin{itemize}
    \item инъективно, если $f(x_1) = f(x_2) \Longleftrightarrow x_1 = x_2$
    \item сюръективно, если $\rho_f = B$
    \item биективно, если $f$ инъективно и сюръективно
\end{itemize}
\section{Аксиомы вещественных чисел}
\begin{conj}
    Вещественные числа - алгебраическая структура, над которой определены 
    операции сложения ``+'' и умножения ``$\cdot$'' $(\mathbb{R} * \mathbb{R} \Longrightarrow \mathbb{R})$
\end{conj}
\newpage
\begin{conj}
    Аксиомы вещественных чисел:
\end{conj}
\begin{itemize}
    \item[$A_1$] Ассоциативность сложения \\
     $x + (y + z) = (x + y) + z$
    \item[$A_2$] Коммутативность сложения \\
     $x + y = y + x$
    \item[$A_3$] Существование нуля \\
     $\exists 0 \in \mathbb{R} : \forall x \in \mathbb{R} \; x + 0 = x$
    \item[$A_4$] Существование обратного элемента по сложению \\
     $\forall x \in \mathbb{R} \; \exists (-x) \in \mathbb{R} : x + (-x) = 0$
    \item[$M_1$] Ассоциативность умножения \\
     $x(y \cdot z) = (x \cdot y)z$
    \item[$M_2$] Коммутативность умножения \\
     $xy = yx$
    \item[$M_3$] Существование единицы \\
     $\exists 1 \in \mathbb{R} : \forall x \in \mathbb{R} \; x \cdot 1 = x$
    \item[$M_4$] Существование обратного элемента по умножению \\
     $\forall x \in \mathbb{R} \; \exists x^{-1} \in \mathbb{R} : x \cdot x^{-1} = 1$
    \item[$M_A$] Дистрибутивность \\
     $(x + y) \cdot z = x \cdot z + y \cdot z$ 
\end{itemize}
\nocite - Вышеперечисленные аксиомы бразуют поле \vspace{0.5cm} \\
\textbf{Бинарное отношение} ``$\leqslant$'' \\
Аксиомы порядка, задающие отношение порядка на множестве вещественных чисел:
\begin{itemize}
    \item[$O_1$] $x \leqslant x \quad \forall x$
    \item[$O_2$] $x \leqslant y $  и  $ y \leqslant x \Longrightarrow x = y$ 
    \item[$O_3$] $x \leqslant y $  и  $ y \leqslant z \Longrightarrow x \leqslant z$ 
    \item[$O_4$] $\forall x, y \in \mathbb{R} : x \leqslant y $ или $ y \leqslant x$
    \item[$O_4$] $x \leqslant y \Longrightarrow x + z \leqslant y + z \quad \forall z$ 
    \item[$O_4$] $0 \leqslant x $ и $ 0 \leqslant y \Longrightarrow 0 \leqslant xy$  
\end{itemize}
\begin{theorem-non}
    Аксиома полноты
\end{theorem-non}
$A, B \subset \mathbb{R} : A \neq \varnothing, B \neq \varnothing, \forall a \in A \; \forall b \in B \; a \leqslant b$ \\
Тогда $\exists c \in \mathbb{R} : a \leqslant c \leqslant b \; \forall a \in A \; \forall b \in B$
\begin{theorem-non}
    Принцип Архимеда
\end{theorem-non}
Согласно принципу Архимеда: $\forall x \in \mathbb{R}$ и $\forall y_{>0} \in \mathbb{R} \; \exists n \in \mathbb{N} : x < ny$
\begin{proof}  
    \quad \\ $A = \{a \in \mathbb{R} : \exists n \in \mathbb{N} : a < ny\}, A \neq \varnothing$ т.к. $0 \in A$ \\
    $B = \mathbb{R} \; \setminus \; A$ \\
    Пусть $A \neq \mathbb{R}$, тогда $B \neq \varnothing$ Покажем, что $a \leqslant b$, если $a \in A, b \in B$ \\
    Пойдем от противного. Если $b < a < ny \Longrightarrow b < ny \Longrightarrow b \in A$ - противоречие \\
    Таким образом, по аксиоме полноты $\exists c \in \mathbb{R} : a \leqslant c \leqslant b \quad \forall a \in A, \forall b \in B$ \\
    Предположим, что $c \in A$. Тогда $c < ny$ для некоторого $n \in \mathbb{N} \Longrightarrow c + y < (n + 1)y \Longrightarrow \\ 
    c + y \in A \Longrightarrow c + y \leqslant c \Longrightarrow y \leqslant 0$. Это противоречит условию. \\
    Пусть $c \in B$. Так как $y > 0, c - y < c$. Так как $B$ - дополненние $A$ и $c - y \neq c, \; c - y \in A
    \Longrightarrow c - y < ny \Longrightarrow c < (n + 1)y \Longrightarrow c \in A$. Снова пришли к противоречию. \\
    Значит $c \notin A, c \notin B \Longrightarrow c$ не существует $\Longrightarrow B = \varnothing \Longrightarrow A = \mathbb{R}$ 
\end{proof}
\textbf{\textit{Следствие:}}
    \begin{itemize}
        \item[] $\forall \varepsilon_{> 0} \; \exists n \in \mathbb{N}: {{1}\over{n}} < \varepsilon$
        \begin{proof}
           \quad \\ $x = 1, y = \varepsilon \Longrightarrow \exists n \in N: 1 < n\varepsilon$
        \end{proof} 
    \end{itemize}
\section{Принцип математической индукции}
\begin{conj}
    Принцип математической индукции
\end{conj}
$P_n$ -последовательность утверждений 
\begin{enumerate}
    \item $P_1$ - верно
    \item $\forall n \in \mathbb{N}$ из $P_n$ следует $P_{n+1}$
\end{enumerate}
Тогда $P_n$ верно при всех $n \in \mathbb{N}$
\begin{theorem-non}
    В конечном множестве вещественных чисел есть наибольший и наименбший элемент
\end{theorem-non}
\begin{proof}
    \quad \\
    Докажем для максимума. Для минимума рассуждения аналогичны \\
    Будем доказывать утверждение по индукции \\
    Для $n = 1$ - очевидно \\
    Переход $X_n \longrightarrow x_{n+1}$ \\
    Рассмотрим произвольное множество из $n$ элементов $X_n = \{x_1, x_2, x_3, \dots x_n\}$, где максимальным элементом 
    является $x_i$. Пусть в наше множество был добавлен элемент $X_{n+1}$. В таком случае, если $X_{n+1}$ > $X_{i}$, то новый максимум равен
    $X_{n+1}$, иначе - максимумом по-прежнему является $X_{i}$. Таким образом, в любом конечном множестве вещественных чисел существует максимальный
    элемент.     
\end{proof}
\newpage
\textbf{\textit{Следствия:}}
\begin{enumerate}
    \item Во всяком непустом множестве натуральных чисел есть наименьший элемент  
    \begin{proof}
        \quad \\
        Пусть $A$ - множество натуральных чисел, не содержащее наименьшего элемента.
        Докажем по индукции, что для любого $n \in \mathbb{N}$ мы имеем $\mathbb{N}_n \cap A = \varnothing$ \\
        $\N_n = \{k \in \N | k \leqslant \N \}$ \\
        Для $n = 1$ утверждение очевидно. \\
        Переход $n \longrightarrow n+1$ \\
        Предположим для $\mathbb{N}_n \cap A = \varnothing$ \\
        Тогда если для $\mathbb{N}_{n+1} \cap A \neq \varnothing$, то наименьший элемент множества $A$ - это $n+1$ \\
        Значит $\mathbb{N}_{n+1} \cap A = \varnothing$
   \end{proof}
   \item Во всяком конечном непустом множестве натуральных чисел есть наибольший элемент
   \begin{proof}
        \quad \\
        Из натуральных чисел строим целые. Множество чисел $A \subseteq \Z$ называется огианиченным сверху и имеет наибольший элемент
        если $\exists c > a, \forall a \in A, c \in \Z$
    \end{proof}
\end{enumerate}
\subsection*{Рациональные и иррациональные числа в интервале}
\begin{enumerate}
    \item Если $x, y \in \mathbb{R}, x < y$, то $\exists r \in \mathbb{Q}: x < r < y$ 
    \begin{proof}
        \quad \\ Пусть $x < 0, y > 0$. Тогда $\exists r = 0 \in \mathbb{Q}: x < r < y$ \\
        Пусть $x \geqslant 0, y > 0, \varepsilon = x - y$. Тогда $\exists n \in \mathbb{N}: {{1}\over{n}} < \varepsilon$ \\
        По принципу Архимеда найдется такое число $m$, что ${{m-1}\over{n}} \leqslant x < {{m}\over{n}}$ \vspace{0.2cm} \\
        Предположим, что ${{m-1}\over{n}} \leqslant x < y \leqslant {{m}\over{n}}$. Тогда мы получим, что ${{1}\over{n}} \geqslant y - x = \varepsilon$.
        Пришли к противоречию\\
        Следовательно, $\exists m \in \mathbb{N} : x < {{m}\over{n}} < y$ \\
        Случай $y \leqslant 0$ аналогичен предыдущему
    \end{proof} 
    \item Если $x, y \in \mathbb{R}, x < y$, то существует иррациональное число $r: x < r < y$ 
    \begin{proof}
        \quad \\ $x - \sqrt{2} < y - \sqrt{2} \Longrightarrow \exists R_{\in \Q} \in (x - \sqrt{2}, y - \sqrt{2}) \Longrightarrow
        x < R + \sqrt{2} < y \; $(Предыдущий пункт)$ \; \Longrightarrow \\ r$ - иррациональное
    \end{proof} 
    \item Если $x \geqslant 1$, то $\exists n \in \mathbb{N}: x - 1 < n \leqslant x$ 
\end{enumerate}
\section{Супремум и инфимум}
\begin{conj}
    \quad \\
    \begin{itemize}
        \item[] $x$ - верхняя граница множества $A$, если $\forall a \in A: a \leqslant x$
        \item[] $y$ - нижняя граница множества $A$, если $\forall a \in A: y \leqslant a$ 
        \item[] Множество ограничено снизу, если существует какая-нибудь нижняя граница
        \item[] Множество ограничено сверху, если существует какая-нибудь верхняя граница
    \end{itemize}
\end{conj}
\begin{conj}
    \quad \\
    Пусть $A$ - ограниченное сверху множество, тогда $sup A$ - наименьшая из его верхних границ
\end{conj}
\begin{conj}
    \quad \\
    Пусть $A$ - ограниченное снизу множество, тогда $inf A$ - наибольшая из его нижних границ
\end{conj}
\begin{theorem-non}
    \quad \\
    \begin{enumerate}
        \item Если $A \subset \R, A \neq \varnothing $ и $ A $ ограничено снизу, то существует единственный $inf A$
        \item Если $A \subset \R, A \neq \varnothing $ и $ A $ ограничено сверху, то существует единственный $sup A$
    \end{enumerate}
\end{theorem-non}
\begin{proof}
    \quad \\
    Докажем (2) \\
    Пусть $B$ - множество всех верхних границ множества $A$, т.е. $\forall a \in A, b \in B: a \leqslant b$ \\
    Тогда по аксиоме полноты всегда найдется такой $c: a \leqslant c \leqslant b$ \\
    $c - sup A$ по определению \\
    Докажем, что $c$ - единсвтенный \\
    Пусть $\exists c_1, c_2 - sup A$ \\
    Тогда если $c_1 < c_2$, то $c_2 \neq sup A$ \\
    Если $c_1 > c_2$,то $c_1 \neq sup A$ \\
    Следовательно, $c_1 = c_2 = sup A \Longrightarrow sup A$ - единсвтенный  
\end{proof}
\follow
\begin{enumerate}
    \item $B \subset A, B \neq \varnothing $ и $ A $ ограничено снизу. Тогда $inf B \geqslant inf A$
    \item $B \subset A, B \neq \varnothing $ и $ A $ ограничено сверху. Тогда $sup B \leqslant sup A$
\end{enumerate}
\begin{proof}
    \quad \\
    Докажем (1) \\
    Пусть $a = inf A$. Тогда $a$ - нижняя граница $A \Longrightarrow \forall x \in
    A : a \leqslant x \Longrightarrow \forall x \in B : a \leqslant x \Longrightarrow \\
    a$ - нижняя граница $B \Longrightarrow a \leqslant inf B$  
\end{proof}
\notice - Теорема неверна без аксиомы полноты \\
\begin{itemize}
    \item[] $A =\{x \in \Q : x^2 < 2\} \Longrightarrow$ в множестве рациональных чисел у $A$ нет супремума
\end{itemize}
\begin{theorem-non}
    \quad \\
    \begin{enumerate}
        \item $a = inf A \Longleftrightarrow 
        \begin{cases}
            a \leqslant x \quad \forall x \in A \\
            \forall \varepsilon > 0 \quad \exists x \in A : x < a + \varepsilon
        \end{cases}$ 
        \item $b = sup A \Longleftrightarrow 
        \begin{cases}
            b \geqslant x \quad \forall x \in A \\
            \forall \varepsilon > 0 \quad \exists x \in A : x > b - \varepsilon
        \end{cases}$ 
    \end{enumerate}
\end{theorem-non}
\notice
\begin{itemize}
    \item Если $A$ неограничено сверху, то $sup A = +\infty$
    \item Если $A$ неограничено снизу, то $inf A = -\infty$
\end{itemize}
\section{Теорема о вложенных отрезках}
\begin{theorem-non}
    \quad \\
    Если $[a_1, b_1] \supset [a_2, b_2] \supset [a_3, b_3] \supset \dots$ \\
    То $\exists c\in \R : c \in [a_n, b_n] \forall n \in \N$
\end{theorem-non}
\begin{tikzpicture}
    \draw (-.5,0)--(5.5,0);
    \draw[color=black] (0, 0) node {\bfseries[} node[below=9pt]{$a_{n}$};
    \draw[color=black] (5, 0) node {\bfseries]} node[below=9pt]{$b_{n}$};
    \draw[color=black] (1.5, 0) node {\bfseries[} node[below=9pt]{$a_{n+1}$};
    \draw[color=black] (4, 0) node {\bfseries]} node[below=9pt]{$b_{n+1}$};
\end{tikzpicture}
\begin{proof}
    \quad \\
    $A = \{a_1, a_2, a_3, \dots\} \\
    B = \{b_1, b_2, b_3, \dots\} \\
    a_i \leqslant b_j, \forall i,j \in \N \\
    \forall i \leqslant j : a_i \leqslant a_j \leqslant b_j \leqslant b_i, \forall i \geqslant j : a_i \geqslant a_j \geqslant b_j \geqslant b_i$ \\
    По аксиоме полноты $\forall i, j \in \N \; \exists c \in \R: a_i \leqslant c \leqslant b_j \Longrightarrow \forall i \in \N : a_i \leqslant c \leqslant b_i$ \\
\end{proof}
\notice
\begin{enumerate}
    \item Теорема неверна для полуинтервалов \\
    Пример: $\bigcap\limits_{n = 1}^{\infty}(0; {{1}\over{n}}] = \varnothing$
    \item Теорема неверна для лучей \\
    Пример: $\bigcap\limits_{n = 1}^{\infty}(n; +\infty) = \varnothing$
    \item Теорема неверна без аксиомы полноты \\
    Пример: число $\pi$ \\
    $[3;\; 4] \supset [3,1;\; 3,2] \supset [3,14;\; 3,15] \supset \dots$ \\
    Пересечение не содержит рациональных чисел
\end{enumerate}
\section{Метрические пространства и подпространства}
    \begin{conj}
        $X$ - множество $\rho : X \times X \longrightarrow [0; + \infty)$ - метрика(расстояние)
        если: 
        \begin{enumerate}
            \item $\rho(x, x) = 0 \quad \forall x \in X$
            \item если $\rho(x, y) = 0$, то $x = y$
            \item $\rho(x, y) = \rho(y, x) \quad \forall x, y \in X$
            \item $\rho(x, y) + \rho(y, z) \geqslant \rho(x, z) \quad \forall x, y, z \in X$
        \end{enumerate}
    \end{conj}
\subsection*{Примеры}
\begin{enumerate}
    \item Дискретная метрика
        \begin{itemize}
            \item[] $\rho (x, x) = 0$
            \item[] $\rho (x, y) = 1$, если $x \neq y$
        \end{itemize}
    \item $\R \quad \rho (x, y) = \abs{x - y}$
    \item $\R^2 \quad$ обычное расстрояние
    \item Манхэттенская метрика 
    \begin{itemize}
        \item[] $(x', y') = A'$
        \item[] $(x, y) = A$
        \item[] $\rho (A, A') = \abs{x - x'} + \abs{y - y'}$  
    \end{itemize}
    \item Французская железнодорожная метрика \\
    \begin{center}
        \begin{tikzpicture}
            \node (p) {P} node (a) at (-2,1) {A} node (b) at (2,2) {B} node (c) at (1,1) {A}; 
            \draw (p) -- (a); \draw[red,thick] (p) -- (c); \draw[red,thick] (c) -- (b);
        \end{tikzpicture}
        Если $P, A$ и $B$ на луче, то $\rho(AB) = AB$ \\
        \quad \quad \quad Если нет, то $\rho(A, B) = \rho(AP) + \rho(B, P)$
    \end{center}
    \item Расстояние на сфере
\end{enumerate}
\begin{conj}
    Метрическое пространство $(X, \rho), X$ - множество, $\rho$ - метрика на нем
\end{conj}
\begin{conj}
    Подпространство метрического пространства. \\
    $(X, \rho)$ - метрическое пространство, $Y \subset X$ \\
    $(Y, \rho \vert_{Y \times Y})$ - подпространство метрического пространства
    $(X, \rho)$, где $Y$ - подмножество $X$, а $\rho \vert_{Y \times Y}$ - сужение $\rho$ на $Y \times Y$
\end{conj}
\begin{conj} 
    Открытый шар \vspace*{0.5cm} \\
    $B\callout{r}{радиус}(\calloutup{a}{центр шара}) := {x \in X: \rho(x, a) < r}; \quad r > 0$
\end{conj}
\begin{conj} 
    Замкнутый шар \vspace*{0.5cm} \\
    $\overline{B_r}(a) := {x \in X: \rho(x, a) \leqslant r}; \quad r \geqslant 0$ \\
    $B_r(a) \subset \overline{B_r}(a)$
\end{conj}
\begin{itemize}
    \item \underline{Окрестность} точки $a$ - открытый шар $B_r(a)$
\end{itemize}
    \subsection*{Примеры} 
    \begin{enumerate}
        \item Дискретная метрика на $X$
        \begin{itemize}
            \item[] $B_{1/2}(a) = {a}$
            \item[] $B_2(a) = X$ 
        \end{itemize}
        \item $\rho(x, y) = |x - y| \quad B_r(a) = (a - r, a + r)$
        \item Манхэттенская метрика
    
        \begin{tikzpicture}
            \draw[help lines] (0,0) grid (2, 2);
            \draw[dotted] (0,1) coordinate (A) -- (1,2) coordinate (B)
            (1,2) -- (2,1);
            \draw[dotted] (0,1) coordinate (A) -- (1,0) coordinate (C)
            (1,0) -- (2,1);
            \fill[red] ((1,1) circle (2pt);
        \end{tikzpicture} \quad $B_r(a)$
    \end{enumerate}
    \subsection*{Свойства}
    \begin{enumerate}
        \item $B_r(a) \cap B_R(a) = B_{min\{r, R\}}(a)$
        \item Если $x \neq y$, то найдется $r > 0$, такой, что 
        $\overline{B_r}(x) \cap \overline{B_r}(y) = \varnothing$
        \begin{proof}
            \quad \\
            $r := {{\rho(x, y)}\over{3}}$. Пойдем от противного \\
            Пусть $c \in \overline{B_r}(x) \cap \overline{B_r}(y) \Longrightarrow
            \begin{cases}
                \rho(x, c) \leqslant r \\
                \rho(y, c) \leqslant r
            \end{cases} \Longrightarrow \rho(x, y) \leqslant \rho(x, c) + \rho(y, c) 
            \leqslant 2r = {{2}\over{3}}\rho(x, y)$ - противоречие
        \end{proof}
    \end{enumerate}
\section{Открытые множества}
\begin{conj}
    Множество $A$ называется открытым, если $A \subset $ метрическому пространству $X$ и $\forall a \in A \; \exists r_{>0} : B_r(a) \subset A$
\end{conj}
\begin{theorem-non}
    Свойства открытых множеств:
    \begin{enumerate}
        \item $\varnothing, X$ - открытые множества 
        \item Объединение любого количества открытых множеств - открытое множество
        \item Пересечение конечного числа открытых множеств - открытое множество
        \item Открытый шарик - открытое множество
    \end{enumerate}
    \begin{proof}
        \quad \\
        \begin{enumerate}
            \item $B_r(a) \subset X$; Для пустого множества нечего проверять, так как там даже точек то нет
            \item $A_{\alpha} \; \alpha \in I$ - открытые множества. $A = \bigcup\limits_{\alpha \in I} A_{\alpha}$ \\
            Возьмем $a \in A$. Тогда $a \in A_{\beta}$ для какого-то $\beta \in I \Longrightarrow A_{\beta}$ - открытое множество 
            $\Longrightarrow B_r(a) \subset A_{\beta}$ для некоторого $r_{>0} \Longrightarrow$ \\
            $B_r(a) \subset A_{\beta} \subset \bigcup\limits_{\alpha \in I} A_{\alpha} = A$
            \item $A_1, A_2, \dots , A_n$ - открытые множества. $A = \bigcap\limits_{k = 1}^{n} A_k$
            Возьмем $a \in A$. Тогда $a \in A_k$ при $k = \{1, 2, \dots , n\} \Longrightarrow
            B_{r_k}(a) \subset A_k$ для некоторого $r_k > 0$ \\
            $r := min\{r_1, r_2, \dots , r_k\} \Longrightarrow B_r(a) \subset B_{r_k}(a) \subset A_k 
            \Longrightarrow B_r(a) \subset \bigcap\limits_{k=1}^n A_k = A$
            \item Рассмотрим $B_R(a)$. Возьмем $b \in B_R(a)$ \\
            $r := R-\rho(a, b) > 0$.
            Докажем, что $x \in B_r(b):$ \\
            $\rho(x, b) < r \Longrightarrow \rho(x,a) \leqslant \rho(x, b) + \rho(b,a) < r + \rho(b,a) = R$
        \end{enumerate}
    \end{proof}
    \notice \\
    В пункте №3 конечность существенна
    $\bigcap\limits_{n=1}^{\infty} B_{1/n}(0) = \bigcap\limits_{n=1}^{\infty}(-{{1}\over{n}}; {{1}\over{n}}) = \{0\}$ Интервал $(-r; \; r)$
\end{theorem-non}
\subsection*{Пример}
$\R \quad \rho(x, y) = \abs{x-y}$ \\
$Y = [0; \; 2)$ \\
Шары в $(Y, \rho)$: \\
\begin{tikzpicture}
    \draw (-.5,0)--(5.5,0);
    \draw[color=black] (0, 0) node {\bfseries[} node[below=9pt]{$0$};
    \draw[color=black] (4, 0) node {\bfseries)} node[below=9pt]{$2$};
\end{tikzpicture} \\
$B_1^Y(0) = \{x \in [0; \;2) : \abs{x - 0} < 1\} = [0; \; 1)$
\section{Внутренние точки. Внутренность множества}
\begin{conj}
    $(X, \beta)$ - метрическое пространство $A \subset X$ \\
    $a \in A, \; a$ - \underline{внутренняя точка множества}, если $B_r(a) \subset A$ для некоторого $r > 0$
    (Открытое множество - такое множество, у которого все точки внутренние) \\
    \underline{Внутренность множества} - множество всех его внутренних точек. Обозначается как $Int A$ \\
\end{conj}
\begin{theorem-non}
    Свойства внутренности: 
    \begin{enumerate}
        \item $Int A \subset A$
        \item $Int A = \bigcup \{G: G \subset A $ и $G$ - открытое$\} =: B$
        \begin{proof}
            \quad \\
            $\bullet \quad Int A \supset B$ \\
            Возьмем $b \in B$. Тогда найдется открытое $G_{\circ} \subset A$, такое, что $b \in G_{\circ} \Longrightarrow$\\
            $\exists r_{>0}$, такой, что $B_r(b) \subset G_{\circ} \subset A \Longrightarrow b$ - внутренняя точка $A$ \\
            $\bullet \quad Int A \subset B$ \\
            Возьмем $a \in Int A \Longrightarrow a$ - внутренняя точка $\Longrightarrow $ открытое множество $ B_r(a) \subset A$ для некоторого $r_{>0}
            \Longrightarrow a \in B_r(a) \subset A$ \\
            $a \in B_r(a) \subset B \Longrightarrow a \in B$
        \end{proof}
        \item $Int A$ - самое большое (по включению) открытое множество, содержащееся в $A$
        \item $Int A$ - открытое множество
        \item $Int A = A \Longleftrightarrow A$ - открытое 
        \item $A \subset B \Longrightarrow Int A \subset Int B$
        \begin{proof}
            Пусть $a \in Int A \Longrightarrow B_r(a) \subset A$ для 
            некоторого $r_{>0} \Longrightarrow a$ - внутренняя точка $B$
        \end{proof}
        \item $Int(A \cap B) = Int A \cap Int B$
        \begin{proof}
            \quad \\
            ``$\subset$'' : $A \cap B \subset A \Longrightarrow Int(A \cap B) \subset Int A$. Это следует из предыдущего пункта. Аналогично для $B$\\ 
            ``$\supset$'' : Пусть $c \in Int A \cap Int B \Longrightarrow 
            \begin{cases}
                c$ - внутренняя точка $A \\
                c$ - внутренняя точка $B 
            \end{cases} \Longrightarrow 
            \begin{cases}
                B_{r_1}(c) \subset A \\
                B_{r_2}(c) \subset B
            \end{cases}$ \vspace{0,2cm} \\ для некоторых $r_1, r_2 > 0 \Longrightarrow 
            B_r(c) \subset A \cap B,$ где $r = min\{r_1, \; r_2\} \Longrightarrow \\ c$ - внутренняя точка $A \cap B$
        \end{proof}
        \item $Int(Int A) = Int A$
        \begin{proof}
            $Int A$ - открытое множество, а внутренность открытого множества совпадает с ним
        \end{proof}
    \end{enumerate}
\end{theorem-non}
\section{Замкнутые множества. Замыкание множества}
\begin{conj}
    $(X, \beta)$ - метрическое пространство $A \subset X$ \\
    $A \subset X \quad A$ - замкнутое, если $X \; \setminus \; A$ -  открытое
\end{conj}
\begin{theorem-non}
    Свойства замкнутых множеств:
    \begin{enumerate}
        \item $\varnothing, X$ - замкнутое множества 
        \item Пересечение любого количества замкнутых множеств - замкнутое множество
        \item Объединение конечного числа замкнутых множеств - замкнутое множество
        \item Замкнутый шарик - замкнутое множество
    \end{enumerate}
    \begin{proof}
        \quad \\
        \begin{enumerate}
            \item[2.] $A_{\alpha} \; \alpha \in I$ - замкнутые множества. $A \overset{?}{\Longrightarrow} \bigcap\limits_{\alpha \in I} A_{\alpha}$ - замкнутое \\
            $\rotatebox[origin=c]{-30}{$\Longrightarrow$} \qquad \qquad \qquad \qquad \qquad \qquad \qquad \qquad \qquad \qquad \rotatebox[origin=c]{30}{$\Longrightarrow$}$ \vspace{0,2cm}\\
            $X \; \setminus \; A$ - открытое $\Longrightarrow \bigcup\limits_{\alpha \in I}(X \; \setminus \; A_{\alpha}) = X \; \setminus \; \bigcap\limits_{\alpha \in I} A_{\alpha}$ - открытое множество 
            \item[3.] $A_1, A_2, \dots , A_n$ - замкнутые множества. $\Longrightarrow X \setminus A_1, X \setminus A_2, \dots , X \; \setminus \; A_n$ - открытые множества \\ 
            $\Longrightarrow \bigcap\limits_{k = 1}^{n} (X \setminus A_k)$ - открытое множество \\
            $\bigcap\limits_{k = 1}^{n} (X \setminus A_k) = X \setminus \bigcup\limits_{k = 1}^{n} A_k \Longrightarrow \bigcup\limits_{k = 1}^{n} A_k$ - замкнутое
            \item[4.] $\overline{B_R}(a)$ - замкнутый шар\\
            Докажем, что $X \; \setminus \; \overline{B_R}(a)$ - открыто
            \begin{proof}
                \quad \\
                $\overline{B_R}(a) = \{x \in X: \rho(x, a) > R\}$ \\
                Возьмем $b \in X \; \setminus \; \overline{B_R}(a) \Longrightarrow \rho(b, a) > R$ \\
                $r := \rho(b, a) - R$ \\
                Докажем, что $B_r \subset X \; \setminus \; B_R(a) \Longleftrightarrow B_r(b) \cap \overline{B_R}(a) = \varnothing$ \\
                От противного. Пусть есть общая точка $c \in B_r(b) \cap \overline{B_R}(a) \Longrightarrow 
                \begin{cases}
                    \rho(c, b) < r \\
                    \rho(c, a) \leqslant R
                \end{cases} \Longrightarrow \vspace{0,2cm}\\
                \rho(c, b) \leqslant \rho(a, c) \leqslant \rho(c, b) < R + r = \rho(a,b)$ \qquad (Так как $\rho(a, c) \leqslant R$ и $\rho(c, b) < r$) \\
                Противоречие.
            \end{proof} 
        \end{enumerate}
    \end{proof}
    \notice \\
    В пункте №3 конечность существенна
    $\bigcup\limits_{n=1}^{\infty}[-1 + {{1}\over{n}}; 1 - {{1}\over{n}}] = (-1; 1)$ \\ Интервал $(-\infty; \; -1] \cup [1; +\infty)$
\end{theorem-non}
\begin{conj}
    Замыкание множества $A$ - пересечение всех замкнутых множеств, содержащих $A$. Обозначаетя как $Cl A$ \\
    $Cl A = \bigcap \{ F: F $ - замкнутое и $ F \supset A \}$ 
\end{conj}
\begin{theorem-non}
    \quad \\
    $X \setminus Cl A = Int(X \setminus A)$ \\
    $X \setminus Int A = Cl(X \setminus A)$
\end{theorem-non}
\begin{proof}
    \quad \\
    $x \in X \setminus Cl A \Longleftrightarrow x \notin Cl A \Longleftrightarrow x \notin F_{\circ}$, где $F_{\circ} \supset A \\
    \Longleftrightarrow 
    \begin{cases}
        x \in X \setminus F_{\circ} =: G_{\circ} $ - открытое$ \\
        G_{\circ} \subset X \setminus A
    \end{cases} \Longleftrightarrow x \in Int(X \setminus A)$ 
\end{proof}
\follow
\quad \\
$Cl A = X \setminus Int(X \setminus A)$ \\
$Int A = X \setminus Cl(X \setminus A)$
\begin{theorem-non}
    Свойства замыкания \\
    \begin{enumerate}
        \item $Cl A$ - замкнутое множество 
        \item $Cl A \supset A$
        \item $A$ - замкнуто $\Longleftrightarrow A = Cl A$
        \begin{proof}
            $A$ - замкнуто $\Longleftrightarrow X \setminus A 
            \Longleftrightarrow X \setminus A = Int(X \setminus A) \Longleftrightarrow \\ A = \underbrace{X \setminus Int(X \setminus A)}\limits_{Cl A}$
        \end{proof}
        \item Если $A \subset B$, то $Cl A\subset Cl B$
        \begin{proof}
            $A \subset B \Longleftrightarrow X \setminus A \supset X \setminus B \Longrightarrow Int(X \setminus A) \supset Int(X \setminus B) \Longrightarrow \\
            \underbrace{X \setminus Int(X \setminus A)}_{Cl A} \subset \underbrace{X \setminus Int(X \setminus B)}_{Cl B}$
        \end{proof}
        \item $Cl(A \cup B) = Cl A \cup Cl B$
        \begin{proof}
            $Cl(A \cup B) = X \setminus Int(\underbrace{X\setminus(A \cup B)}_{(X\setminus A)\cap(X\setminus B)}) = 
            X \setminus Int((X\setminus A) \cap (X\setminus B)) = \vspace{0,2cm} \\ X \setminus (Int(X\setminus A) \cap Int(X\setminus B)) =
            (X \setminus Int(X\setminus A)\cup (X \setminus Int(X\setminus B) = Cl A \cup Cl B$
        \end{proof}
        \item $Cl Cl A = Cl A$
        \begin{proof}
            $Cl A$ - замкнуто + замыкание замкнутого множества - само множество
        \end{proof}
    \end{enumerate}
\end{theorem-non}
\begin{theorem-non}
    $x \in Cl A \Longleftrightarrow$ для любого $r > 0: B_r(x) \cap A \neq \varnothing$
\end{theorem-non}
\begin{proof}
    $x \in Cl A \Longleftrightarrow x \in X \setminus Int(X \setminus A) \Longleftrightarrow
    x \notin Int(X \setminus A) \Longleftrightarrow$ для любого $r > 0: B_r(x)$ не целиком содержится 
    в $X \setminus A \Longleftrightarrow$ для любого $r > 0: B_r(x) \cap A \neq \varnothing$ 
\end{proof}
\follow 
\quad Если $\mathcal{U}$ - открытое и $\mathcal{U} \cap A = \varnothing$, то $\mathcal{U} \cap Cl A = \varnothing$
\begin{proof}
    Пусть $x \in \mathcal{U} \cap Cl A \Longrightarrow x \in \mathcal{U}$ - открытое $\exists r > 0 \quad B_r(x) \subset \mathcal{U}\\
    x \in \mathcal{U} \cap Cl A \Longrightarrow x \in Cl A \Longrightarrow B_r(x) \cap A \neq \varnothing \Longrightarrow \mathcal{U} \cap A \neq \varnothing$ - противоречие
\end{proof}
\section{Предельные точки. Связь с замыканием множества}
\begin{conj}

    Проколотая окрестность точки $a - B_r(a) \setminus {a}$ \\
    Обозначается как $\overset{\circ}{\mathcal{U}}_a$
\end{conj}
    \begin{conj}
        Предельная точка множества \\
        $a$ - предельная точка множества $A$, если любая $\overset{\circ}{\mathcal{U}_a}\cap A \neq \varnothing$\\
        $A'$ - множество предельных точек $A$
    \end{conj} 
    \begin{theorem-non}
        Свойства:
        \begin{enumerate}
            \item $Cl(A) = A\cup A'$
            \begin{proof}
                $x \in Cl(A) \Longleftrightarrow B_r(x)\cap A \neq \varnothing \forall r>0$ (*) \\
                Пусть $x \notin A$ . Тогда (*) равносильно $B_r(x)\setminus \{x\} \cap A \neq \varnothing \Longleftrightarrow x\in A'$
            \end{proof}
            \item $A \subset B \Longrightarrow A' \subset B'$
            \begin{proof}
                $x \in A' \Longrightarrow B_r(x)\setminus \{x\} \cap A  \neq \varnothing \Longrightarrow B_r(x)\setminus \{x\} \cap B \neq \varnothing \Longrightarrow x \in B'$
            \end{proof}
            \item $(A\cup B)' = A' \cup B'$
            \begin{proof}
                $A \cup B \supset A \Longrightarrow (A\cup B)' \supset A' \Longrightarrow (A\cup B) \supset A'\cup B'$ \\
                Обратное включение. Пусть $x\in (A\cup B)'$ и $x \notin B' \Longrightarrow (B_r(x)\setminus \{x\})\cap (A\cup B) \neq \varnothing \Longrightarrow (B_r(x)\setminus \{x\})\cap A \neq \varnothing$ ИЛИ $(B_r(x)\setminus \{x\})\cap B \neq \varnothing$. \\
                Второе неверно из $x\notin B'$, следовательно $x \in A'$
            \end{proof}
            \item $A$ замкнутно $\Longrightarrow A\supset A'$
            \begin{proof}
                $A$ - замкнуто $\Longrightarrow A = Cl(A) = A \cup A' \Longleftrightarrow A \supset A'$
            \end{proof}
        \end{enumerate}
    \end{theorem-non}
    
\section{Открытые и замкнутые множества в пространстве и подпространстве}
    
    \begin{theorem-non}
        $(X, d)$ -метр пространство $Y \subset X$. Тогда:
        \begin{enumerate}
            \item $A \subset Y$ открыто в $Y \Longleftrightarrow$ найдется открытое множество $G \subset X$, т.ч. $A=G\cap Y$
            \begin{proof}
                $a \in A \Longrightarrow \exists r_a>0 : B_{r_a}^Y(a)$ (т.е. шары в $Y$) $\subset A$. Далее
                \begin{center}
                    $G:= \underset{a\in A}{\cup} B_{r_a}^X(a) = \underset{a\in A}{\cup} \{x \in X: d(x, a)<r_a\} \Longrightarrow$
                \end{center}
                $G$ - открытое (объединение любого числа открытых - открытое)
                Доказать: $G\cap Y = A$
                $G \supset A, Y \supset A \Longrightarrow G\cap Y \supset A$. Докажем обратное включение.
                \begin{center}
                    $B_{r_a}^X(a)\cap Y = B_{r_a}^Y \subset A$ \vspace*{0,2cm} \\
                    $G\cap Y = \underset{a\in A}{\cup}(B_{r_a}^X(a) \cap Y) \subset A \Longrightarrow G\cap Y \subset A$
                \end{center}
                Доказали "$\Longrightarrow$". Теперь докажем "$\Longleftarrow$"\\
                $G$ - открыто в $Y$. Доказать, что $A := G\cap Y$ - открыто в $Y$
                $a \in A \Longrightarrow a \in G$, $G$ - открыто$ \Longrightarrow \exists r>0 : B_r^X(a) \subset G \Longrightarrow B_r^X(a)\cap Y = B_r^Y(a) \subset G\cap Y$, то есть $A$ открыто в $Y$.
            \end{proof}
            \item $A \subset Y$ замкнутое в $Y \Longleftrightarrow$ найдется замкнутое множество $F \in X $, т. ч $A = F\cap Y$
            \begin{proof}
                $A$ - замкнуто в $Y \Longleftrightarrow Y\setminus A$ - открыто в $Y \Longleftrightarrow \\
                \exists$ открытое $G \in X : Y\setminus A = G\cap Y \Longleftrightarrow \\
                F := X\setminus G$ - замкнутое в $X$, при этом $A = Y\setminus (G\cap Y) = Y\cap (X\setminus G)$ \vspace*{0,2cm}\\
                \framebox[1.02\width]{С первого взгляда неочевидный переход, но следует из вложенности $Y$ и $G$ в $X = Y\cap F$.} \par
            \end{proof}
        \end{enumerate}
    \end{theorem-non}
    
\section{Предел числовой последовательности и предел последовательности в метрическом пространтстве}

    \begin{conj}
        Предел числовой последовательности \\
        $x_1, x_2, x_3 ... \in R$. $a=\lim x_n$ если вне любого интервала, содержащего $a$, содержится лишь конечное число членов последовательности.
    \end{conj}
    \notice -
    Можно рассматривать симметричные интервалы (если есть несимметричный, для удобства его можно расширить или сузить до симметричного)
    
    \begin{conj}
        Предел последовательности в метрическом пространстве
        $(X, d)$ - метрическое пространство, $x_1, x_2 ... \in X$. $a=\lim x_n$ если вне любого шара $B_\varepsilon (a)$ содержится лишь конечное число членов последовательности.
    \end{conj}
    
    \notice -
    Верно также для любого открытого множества, содержащего $a$
    
    \notice -
    Cуществование предела зависит от пространства (в $R_+ x_n = 1/n$ не имеет предела)
    
    \begin{theorem-non}
        Свойства:
        \begin{enumerate}
            \item Если $a=\lim x_n$ и из $x_n$ выкинули какое-то число членок так, чтобы осталось бесконечное число членов, то у оставшейся последовательности тот же предел
            \item Если $a=\lim x_n$ и к последовательности добавить конечное число членов, то $a$ - все еще предел
            \item Добавление, замена или выкидывание конечного количества членом не меняет предел и его наличие (то же самое другими словами)
            \item Перестановка членов не влияет на предел последовательности
            \item Если $a=\lim x_n$ и $a=\lim y_n$, то если их перемешать, то у новой последовательности тоже предел $a$
            \item Если $a=\lim x_n$, тогда у последовательности, в которой $x_n$ встречается с конечной кратностью, тот же предел (написать один и тот же элемент много раз подряд)
        \end{enumerate}
    \end{theorem-non}
    \begin{conj}
        $a=\lim x_n$, если
    \[ \forall \varepsilon>0 \exists N : \forall n\geq N d(x_n, a)<\varepsilon\]
    \end{conj}
    \begin{conj}
        $A \subset X, (X, d)$ - метрическое пространство
        $A$ - ограничено, если $A$ целиком содержится в каком-нибудь шаре
    \end{conj}
    
    \begin{theorem-non}
        \quad \\
        \begin{enumerate}
            \item Предел единственный
            \begin{proof}
                \quad \\
                Пусть $a\neq b \Longrightarrow \exists B_{r_1}(a), B_{r_2}(b) : B_{r_1} \cap B_{r, 2} = \varnothing$. \\
                Вне $B_{r_1}(a)$ конечное число членов \\
                Вне $B_{r_2}(b)$ конечное число членов \\
                Тогда в последовательности конечное число членов. Противоречие.
            \end{proof}
            \item Если последовательность имеет предел, то она ограничена
            \begin{proof}
                \quad \\
                Возьмем $\varepsilon=1$. Тогда $\exists N: \forall n\geq N\ x_n \in B_1(a)$. \\
                Тогда $r:= \max\{d(a, x_1), d(a, x_2),..., d(a, x_N)\}+1$
            \end{proof}
            \item $a=\lim x_n \Longleftrightarrow \lim d(x_n, a)=0$
            \begin{proof}
                \quad \\
                $\lim d(x_n, a)=0 \Longleftrightarrow \forall \varepsilon > 0 \exists N : \forall n \geq N d(x_n, a) < \varepsilon \Longleftrightarrow \lim x_n = a$
            \end{proof}
            \item Если $a=\lim x_n$ и $b = \lim y_n$, то $\lim d(x_n, y_n) = d(a, b)$
            \begin{proof}
                \quad \\
                $d(a, b) \leq d(a, x_n)+d(x_n, y_n)+d(y_n, b)$
                $d(x_n, y_n) \leq d(a, x_n)+d(a, b)+d(b, y_n) \Longrightarrow$
                $|d(x_n, y_n)-d(a, b)|\leq d(x_n, a)+d(y_n, b)$ \\
                Справа каждая меньше $\varepsilon /2$, тогда слева стремится к нулю
            \end{proof}
        \end{enumerate}
    \end{theorem-non}


    \section{Связь между пределами и предельными точками}
    
    \begin{theorem-non}
        $a$ - предельная точка $A \Longleftrightarrow$ найдется последовательность точек $x\neq a \in A : \lim x_n = a$. Супер очевидно из соответствующих определений, но распишу
        \begin{proof}
            \quad \\
            ``$\Longleftarrow$'': \quad Пусть $x_n \in A$ и $\lim x_n = a$. \\
            Тогда в $B_r(a)\setminus \{a\}$ содержится бесконечное количество точек из $x_n$, так как $\exists N : \forall n \geq N x_n \in B_r(a)$ \\
            ``$\Longrightarrow$'': \quad
            $r_1 = 1 \Longrightarrow \exists x_1 \in B_1(a), r_2 = \min\{1/2, d(a, x_1)\}, r_3=\min\{1/3, d(a, x_2)\}...$ \\
            $\forall \varepsilon>0\ \exists N: 1/N<\varepsilon \Longrightarrow \forall n\geq N\ d(x_n, a) < 1/n \leq 1/N < \varepsilon$
        \end{proof}
    \end{theorem-non}
    
    
    
    \begin{theorem-non}
        Если $x_n \in A$ и $a=\lim x_n$, то $a \in Cl(A)$
    \end{theorem-non}
    \begin{proof}
        Либо $a\in A$, тогда $a \in Cl(A)$, иначе $x_n \neq a$, тогда по теореме 1. $a \in A' \Longrightarrow a \in Cl(A)$
    \end{proof}

    \section{Предльный переход в неравенствах}

    
    \begin{theorem-non}
        Предельный переход в неравенстве. $x_n, y_n \in \mathbb{R}$ \\
        $x_n \leq y_n\ \forall n, a=\lim x_n, b=\lim y_n \Longrightarrow a\leq b$
    \end{theorem-non}
        
    \begin{proof}
        Пусть $a>b$ \\
        $\varepsilon = \frac{a+b}{2}$ \\
        $\exists N_1: \forall n\geq N_1\ x_n\in (a-\varepsilon, a+\varepsilon)$ \\
        $\exists N_2: \forall n\geq N_2\ y_n \in (b-\varepsilon, b+\varepsilon)$ \\
        $n:=\max\{N_1, N_2\}$ \\
        $y_n \leq x_n$. Противоречие
    \end{proof}
    
    \notice - неверно для строгого знака ($-1/n, 1/n$)
    
    \follow \; Если $x_n \leq b \forall n, \lim x_n = a \Longrightarrow a\leq b$
    \begin{proof}
        $y_n:=b$, далее из теоремы 1
    \end{proof}
    \newpage
    \follow \; Если $x_n \geq a \forall n, \lim x_n = b \Longrightarrow a \leq b$

    \begin{proof}
        $y_n:=a$, далее из теоремы 1
    \end{proof}
    
    \follow \; $x_n \in [a, b], \lim x_n = c \Longrightarrow c \in [a, b]$. 
    Следует из предыдущих
    
    \section{Теорема о двух милиционерах}
    
    
    \begin{theorem-non}
        Теорема о сжатой последовательности (о двух милиционерах) \\
        $x_n \leq y_n \leq z_n\ \forall n\in N, \lim x_n = \lim z_n = a \Longrightarrow \lim y_n = a$
    \end{theorem-non}
    \begin{proof}
        \[
    \begin{drcases}
    \lim x_n = a \Longrightarrow \forall \varepsilon>0\ \exists N_1: x_n\in (a-\varepsilon, a+\varepsilon)\\
    \lim z_n = a \Longrightarrow \forall \varepsilon>0\  \exists N_2: z_n\in (a-\varepsilon, a+\varepsilon)
    \end{drcases}
    \Longrightarrow x_n > a-\varepsilon, z_n<a+\varepsilon 
    \] \\
    При $n\geq \max\{N_1, N_2\}\ a-\varepsilon<x_n\leq y_n \leq z_n<a+\varepsilon \Longrightarrow a-\varepsilon<y_n<a+\varepsilon$\\
    \end{proof}
    
    \follow \; $|y_n|\leq z_n\ \forall n, \lim z_n = 0 \Longrightarrow \lim y_n = 0$
    
    \begin{proof}
        $x_n:=-z_n \Longrightarrow x_n \leq |y_n| \leq z_n,\ x_n \to 0,\ z_n \to 0 \Longrightarrow y_n \to 0$
    \end{proof}
    
    \section{Монотонные последовательности}
    
    \begin{conj}
        \quad \\
        \begin{itemize}
            \item[] $x_n$ монотонно возрастает(убывает), если $\forall n\ x_n\leq(\geq ) x_{n+1}$
            \item[] $x_n$ монотонна, если она монотонно возрастает или монотонно убывает
        \end{itemize}
    \end{conj}
    
    \begin{theorem-non}
        Если последовательность монотонно возрастает(убывает) и ограничена сверху(снизу), то она имеет предел.
    \end{theorem-non}
    \begin{proof}
        $x_n$ такова, что $x_1\leq x_2\leq x_3...$ и ограничена сверху. Тогда у нее есть $sup:=S$. Докажем, что $\lim x_n = S$. \\
        $\forall \varepsilon>0\ \ S-\varepsilon$ не является верхней границей $\Longrightarrow \exists x_N>s-\varepsilon \Longrightarrow \forall n\geq N\ S-\varepsilon < x_n < S+\varepsilon \Longrightarrow$ S - предел
    \end{proof}

    \follow \; Если последовательность монотонна, то она имеет предел тогда и только тогда, когда она ограничена.

    ``$\Longleftarrow$'' По доказанной теореме \\
    ``$\Longrightarrow$'' Из свойств предела
    
    \section{Топологическое пространство}
    
    \begin{conj}
        $X$ - множество. Топология, это набор подмножеств $\Omega \subset X$, называющихся открытыми, таких что: 
        \begin{enumerate}
            \item $\varnothing, X$ - открытые
            \item Объединение любого количество открытых - открыто
            \item Пересечение конечного числа открытых - открыто\\
        \end{enumerate}
    \end{conj}
    
    \subsubsection*{Примеры}
    \begin{itemize}
        \item[] $\{\varnothing, X\}$
        \item[] $X = [0, +\infty), \Omega = (a, +\infty), a\geq 0\}$
    \end{itemize}  

    \begin{conj}
        Замкнутое множество - дополнение открытого
    \end{conj}
    \begin{conj}
        $a$ - внутренняя точка множетсва $A$, если существует открытое множество $U$, т. ч. $a \in U, U\subset A$
    \end{conj}
    \begin{conj}
        Внутренность $Int\ A$ - объединение всех открытых множеств, содержащихся в $A$. Равносильно - множество всех внутренних точек
    \end{conj}
    \begin{conj}
        Замыкание $Cl\ A$ - пересечение всех замкнутых множеств, содержищих $A$
    \end{conj}
    \begin{conj}
        $a = \lim x_n$, если вне любого открытого множества, содержащего точку $a$ находится лишь конечное число членов последовательности \\
        $\forall U \ni a\ \exists N\ \forall n\geq N\ x_n \in U$
    \end{conj}
    \begin{conj}
        Хаусдорфовость \\
        $\forall a, b \in X \ \exists U, V$ - открытые множества, такие что $a\in U,\ b\in V,\ U\cap V = \varnothing$.
    \end{conj}
    \begin{conj} 
        Если хаусдорфовость выполняется, то предел единственный.
        \begin{proof}
            Если $a, b$ - пределы, то $\exists U, V : a\in U,\ b\in V,\ U\cap V = \varnothing \Longrightarrow $ Вне $U$ лежит конечное количество членов, вне $V$ тоже, тогда и в $X$ конечное число членов. Противоречие
        \end{proof}
    \end{conj}

    \section{Векторное пространство. Пространство $R^d$. Скалярное произведение. Неравенство Коши-Буняковского}
    
    \begin{conj}
        $X$ - векторное пространство (над полем $\mathbb{R}$), если:
        Определена операции "+": $X\times X \to X$
        "*": $\mathbb{R}\times X \to X$ \\
        \begin{enumerate}
            \item Сложение коммутативно и ассоциативно
            \item Существует $\overrightarrow{0}$
            \item Существует обратный элемент $x+(-x)=\overrightarrow{0}$
            \item $(\alpha \beta)x = \alpha(\beta x)\ \forall \alpha, \beta \in \mathbb{R}\ \ \forall x\in X$
            \item $(\alpha+\beta)x = \alpha x + \beta x$
            \item $\alpha(x+y) = \alpha x + \alpha y$
        \end{enumerate}
    \end{conj}
    \vspace*{0,5cm}

    \begin{conj}
        \quad \\
        $R^d = \{ \langle x_1, x_2,...,x_d\rangle  \}: x_i \in \mathbb{R}$
    
        $\langle x_1,...,x_d\rangle +\langle y_1,...,y_d\rangle =\langle x_1+y_1,..., x_d + y_d\rangle $
    
        $\alpha \langle x_1,...,x_d\rangle  = \langle \alpha x_1,..., \alpha x_d\rangle $
    \end{conj}
    
    \begin{conj}
        Скалярное произведение $\langle \bullet, \bullet\rangle X\times X \to \mathbb{R}$
    
        \begin{enumerate}
            \item $\langle x, x\rangle \geq 0,\ \langle x, x\rangle =0 \Longleftrightarrow x=\overrightarrow{0}$
            \item $\langle x, y\rangle = \langle y, x\rangle$
            \item $\langle x+y, z\rangle = \langle x, z\rangle + \langle y, z \rangle $
            \item $\langle \alpha x, y\rangle = \alpha \langle x, y \rangle $
        \end{enumerate}
    \end{conj}
    
    \begin{conj}
        Неравенство Коши-Буняковского: $\langle x, y \rangle^2 \leq \langle x, x \rangle \langle y, y \rangle$
    \end{conj} 
    
    \begin{proof}
    
    $f(t):=\langle x+ty, x+ty\rangle = \langle x, x+ty \rangle + \langle ty, x+ty \rangle =\langle x, x \rangle + t\langle x, y \rangle + t\langle y, x \rangle + t^2\langle y, y \rangle = \langle x, x \rangle + 2t\langle x, y \rangle + t^2\langle y, y \rangle \geq 0$. Это всегда неотрицательно, тогда дискриминант неположителен.

    \[4t^2\langle x, y \rangle^2 - 4t^2\langle x, x \rangle \langle y, y \rangle \leq 0 \Longrightarrow \langle x, x \rangle^2 \leq \langle x, x \rangle \langle y, y \rangle\]
    \end{proof}

    \section{Норма}
    
    \begin{conj}
        Норма $||\bullet || : X\to \mathbb{R}$\\
        \begin{enumerate}
            \item $||x|| \geq 0$, $||x|| = 0 \Longleftrightarrow x = \overleftarrow{0}$
            \item $||\alpha x|| = |\alpha|*||x||$
            \item $||x+y|| \leq ||x||+||y||$
        \end{enumerate}
    \end{conj}
    
    \subsection*{Примеры}
    $X = \mathbb{R},\ ||x||:=|x|$
    
    $X = \mathbb{R}^d,\ ||x||:=|x_1|+|x_2|+...+|x_d|$
    \newpage
    \begin{theorem-non}
        Если $\langle \bullet, \bullet \rangle$ - скалярное произведение в $X$, то $||x||:= \sqrt{\langle x, x \rangle}$ - норма.\\
        $||\alpha x|| = \sqrt{\langle \alpha x, \alpha x\rangle} = \sqrt{\alpha^2 \langle x, x\rangle} = |\alpha|\sqrt{\langle x, x \rangle}$\\
        $||x+y||^2 = \langle x+y, x+y \rangle = \langle x,x\rangle + 2\langle x, y\rangle + \langle y, y\rangle \Longrightarrow ||x||^2 + 2\langle x, y\rangle + ||y||^2 \overset{?}{\leq} ||x||^2 + 2||x||*||y|| + ||y||^2$ \vspace*{0,001cm}\\
        $2\langle x, y \rangle \leq 2||x||*||y|| = \sqrt{\langle x, x\rangle}\sqrt{\langle y, y\rangle\ }$ - верно по неравенству Коши Буняковского
    \end{theorem-non}
    
    \begin{theorem-non}
        Свойства норм:
        \begin{enumerate}
            \item $||x-y||=||(x-z)+(z-y)|| \leq ||x-z||+||z-y||$
            \item $d(x, y):=||x-y||$ - метрика
            \item $|\ ||x||-||y||\ | \leq ||x-y||$ \\
            $|x|| = ||(x-y)+y|| \leq ||x-y||+||y||$ \\
            $||y|| = ||(y-x)+x|| \leq ||y-x||+||x||=||x-y||+||x||$ \\
            $||x-y||\geq ||x||-||y||$ \\
            $||x-y||\geq -(||x||-||y||)$
        \end{enumerate}
    \end{theorem-non}
    \begin{theorem-non}
        $X$ - нормированное пространство. Тогда норма порождена некоторым скалярным произведением тогда и только тогда, когда
    
        $2(||x||^2+||y||^2)=||x+y||^2+||x-y||^2$ - тождество параллелограмма.
    
        Доказательства не будет. Автор принял Линал
    \end{theorem-non}
    \section{Арифметические свойства пределов последовательности}
    $X$ - нормированное пространство \\
    $x_n,\; y_n \in X \quad \lambda_n \in \mathbb{R}$ \\
    $\lim x_n = x_0 \quad \lim y_n = y_0 \quad \lim \lambda_n = \lambda_0$ 
    \begin{theorem-non} Арифметические свойства пределов в нормированном пространстве \end{theorem-non}
    \begin{enumerate}
        \item $lim (x_n + y_n) = x_0 + y_0$
        \begin{proof}
                \begin{gather*}
                    ||x_n + y_n - (x_0 + y_0) || = || (x_n - x_0) + (y_n - y_0)|| \leqslant ||x_n - x_0|| + ||y_n - y_0|| \\
                    \lim x_n = x_0 \Rightarrow \forall \varepsilon > 0 \quad \exists N_1 : \forall n \geqslant N_1 \quad ||x_n - x_0|| < \frac{\varepsilon}{2} \\
                    \lim y_n = y_0 \Rightarrow \forall \varepsilon > 0 \quad \exists N_2 : \forall n \geqslant N_2 \quad ||y_n - y_0|| < \frac{\varepsilon}{2} 
                \end{gather*}
                Тогда при $n \geqslant max\{N_1, N_2\} \quad ||x_n + y_n - (x_0 + y_0)|| \leqslant ||x_n - x_0|| + ||y_n - y_0|| < \varepsilon$
        \end{proof}
        \item $\lim (x_n - y_n) = x_0 - y_0$
        \begin{proof}
                Аналогично первому пункту.
        \end{proof}
        \item $\lim \lambda_nx_n = \lambda_0x_0$
        \begin{proof}
                \begin{gather*}
                || \lambda_nx_n - \lambda_0x_0 || = ||(\lambda_nx_n - \lambda_nx_0) + (\lambda_nx_0 - \lambda_0x_0)|| \leqslant \\
                \leqslant ||\lambda_nx_n - \lambda_nx_0|| + ||\lambda_nx_0 - \lambda_0x_0|| = |\lambda_n|*||x_n - x_0|| + |\lambda_n - \lambda_0|*||x_0||
                \end{gather*} 
                Так как у $\lambda_n$ есть предел, она ограничена, то есть $|\lambda_n| \leqslant M$. \\
                Итого получаем:
                \[ || \lambda_nx_n - \lambda_0x_0 || \leqslant M*||x_n - x_0|| + ||x_0||*|\lambda_n - \lambda_0|| \]
                \begin{gather*}
                    \lim x_n = x_0 \Rightarrow \forall \varepsilon > 0 \quad \exists N_1 : \forall n \geqslant N_1 \quad ||x_n - x_0|| < \frac{\varepsilon}{2M} \\
                    \lim \lambda_n = \lambda_0 \Rightarrow \forall \varepsilon > 0 \quad \exists N_2 : \forall n \geqslant N_2 \quad |\lambda_n - \lambda_0| < \frac{\varepsilon}{2||x_0||+1}
                \end{gather*}
                При $n \geqslant max\{N_1, N_2\}$
                \[|| \lambda_nx_n - \lambda_0x_0 || \leqslant M*||x_n - x_0|| + ||x_0||*|\lambda_n - \lambda_0|| < M * \frac{\varepsilon}{2M} + ||x_0|| * \frac{\varepsilon}{2||x_0||+1} < \varepsilon\]
        \end{proof}
        \item $\lim ||x_n|| = ||x_0||$
        \begin{proof}
                \[ ||x_n|| - ||x_0|| = ||(x_n - x_0) + x_0|| - ||x_0|| \leqslant ||x_n - x_0|| + ||x_0|| - ||x_0|| = ||x_n - x_0|| \to 0 \]
        \end{proof}
        \item Если в $X$ есть скалярное произведение, то $\lim<x_n, y_n> = <x_0, y_0>$
        \begin{proof}
            \begin{gather*}
                <x_n, y_n> - <x_0, y_0> = <x_n, y_n> - <x_n, y_0> + <x_n, y_0> - <x_0, y_0> =  \\
                = <x_n, y_n - y_0> + <x_n - x_0, y_0> \\
                | <x_n, y_n> - <x_0, y_0> | \leqslant | <x_n, y_n - y_0> | + | <x_n - x_0, y_0> | \leqslant \\
                \leqslant ||x_n||*||y_n-y_0||+||x_n-x_0||*||y_0||
            \end{gather*}
            Так как у $x_n$ есть предел, она ограничена, то есть $||x_n|| \leqslant M$. \\
            Итого получаем:
            \begin{gather*}
                | <x_n, y_n> - <x_0, y_0> | \leqslant M*\underbrace{||y_n - y_0||}_{\to 0}+||y_0||*\underbrace{||x_n - x_0||}_{\to 0}
            \end{gather*}
        \end{proof}
    \end{enumerate}
    \begin{theorem-non} Арифметические свойства пределов числовых последовательностей \end{theorem-non}
    $x_n,\; y_n \in \mathbb{R} \quad \lim x_n = x_0 \quad \lim y_n = y_0$
    \begin{enumerate}
        \item $\lim (x_n \pm y_n) = x_0 \pm y_0$
        \item $\lim (x_ny_n) = x_0y_0$
        \item $\lim |x_n| = |x_0|$
        \item Если $y_0 \neq 0$ и $y_n \neq 0\;\; \forall n$, то $\lim \frac{x_n}{y_n} = \frac{x_0}{y_0}$
        \begin{proof}
            Докажем, что $\lim\frac{1}{y_n} = \frac{1}{y_0}$:
            \[ | \frac{1}{y_n} - \frac{1}{y_0} | = \frac{|y_n - y_0|}{|y_n||y_0|} \]
            Так кая $y_0 = \lim y_n$, найдется такое $N_1$, что $\forall n \geqslant N_1 \quad |y_n| \in (\frac{|y_0}{2}, \frac{3|y_0|}{2}) \Rightarrow |y_n| > \frac{|y_0|}{2}$ \\
            При $n >= N_1$ получаем, что
            \begin{gather*}
                \frac{|y_n - y_0|}{|y_n||y_0|} < \frac{|y_n - y_0|}{\frac{|y_0|}{2}|y_0|} \\
                lim y_n = y_0 \Rightarrow \forall \varepsilon > 0 \quad \exists N_2 : \forall n \geqslant N_2 \quad |y_n - y_0| < \frac{\varepsilon*y_0^2}{2}
            \end{gather*}
            Тогда если $n \geqslant max\{N_1, N_2\}$, то $|\frac{1}{y_n} - \frac{1}{y_0}| < \varepsilon$.
            Теперь, когда мы знаем, что $lim\frac{1}{y_n} = \frac{1}{y_0}$, доказать исходное равенство легко:
            \[ \lim \frac{x_n}{y_n} = \lim (x_n * \frac{1}{y_n}) = \lim x_n * \lim \frac{1}{y_n} = \frac{x_0}{y_0} \]
        \end{proof}
    \end{enumerate}
    
    \section{Покоординатная сходимость в $\mathbb{R}^d$}
    \[ x_n = <x_n^{(1)}, \dots, x_n^{(d)}> \]
    $x_n$ покоординатно сходится к $x_0$, если \\
    \begin{gather*}
        \begin{cases}
            \lim x_n^{(1)} = x_0^{(1)} \\
            \dots \\
            \lim x_n^{(d)} = x_0^{(d)}
        \end{cases}
    \end{gather*}
    
    \begin{theorem-non} \end{theorem-non}
    $x_n$ покоординатно сходится к $x_0 \Longleftrightarrow x_n$ сходится к $x_0$ по норме в $\mathbb{R}^d$ 
    
    $||a|| = \sqrt{a_1^2 + \dots + a_d^2}$ - норма
    
    \begin{proof}
        \[ ||x_n - x_0|| = \sqrt{(x_n^{(1)} - x_0^{(1)})^2 + \dots + (x_n^{(d)} - x_0^{(d)})^2} \]
        Заметим следующее: 
        \[ \sqrt{(x_n^{(1)} - x_0^{(1)})^2 + \dots + (x_n^{(d)} - x_0^{(d)})^2} \geqslant \sqrt{(x_n^{(k)} - x_0^{(k)})^2} = |x_n^{(k)} - x_0^{(k)}| \]
        \[  \sqrt{(x_n^{(1)} - x_0^{(1)})^2 + \dots + (x_n^{(d)} - x_0^{(d)})^2} \leqslant |x_n^{(1)} - x_0^{(1)}| + \dots + |x_n^{(d)} - x_0^{(d)}| \]
        Итого получаем
        \[ |x_n^{(k)} - x_0^{(k)}| \leqslant ||x_n - x_0|| \leqslant |x_n^{(1)} - x_0^{(1)}| + \dots + |x_n^{(d)} - x_0^{(d)}| \]
        Докажем $"\Rightarrow"$:
        \[ \lim x_n = x_0 \Rightarrow ||x_n - x_0|| \to 0 \Rightarrow  |x_n^{(k)} - x_0^{(k)}| \to 0 \Rightarrow \lim x_n^{(k)} = x_0^{(k)} \]
        Докажем $"\Leftarrow"$:
        \[ \lim x_n^{(k)} = x_0^{(k)} \Rightarrow |x_n^{(k)} - x_0^{(k)}| \to 0 \Rightarrow \sum_{k = 1}^d |x_n^{(k)} - x_0^{(k)}| \to 0 \Rightarrow ||x_n - x_0|| \to 0 \Rightarrow \lim x_n = x_0  \]
    \end{proof}
    
    \section{Бесконечные пределы}
    \begin{itemize}
        \item \underline{$x_n \in \mathbb{R} \quad \lim x_n = +\infty$}
        
        Вне любого луча $(u, +\infty)$ находится лишь конечное число членов.
        
        $\forall u\quad \exists N: \forall n \geqslant N \quad x_n > u$
        \item \underline{$x_n \in \mathbb{R} \quad \lim x_n = -\infty$}
        
        Вне любого луча $(-\infty, u)$ находится лишь конечное число членов.
        
        $\forall u\quad \exists N: \forall n \geqslant N \quad x_n < u$
        
        \item \underline{$x_n \in \mathbb{R} \quad \lim x_n = \infty$}
        
        В любом интервале $(u, v)$ находится лишь конечное число членов.
        
        $\forall u\quad \exists N: \forall n \geqslant N \quad |x_n| > u$
    \end{itemize} 
    \vspace{0.7cm}
    \underline{Замечание 1}: Если $\lim x_n = +\infty$ или $\lim x_n = -\infty$, то $\lim x_n = \infty$. Обратное неверно (контрпример - $x_n = (-1)^nn$).
    
    \underline{Замечание 2}: Если $\lim x_n = \infty$, то ${x_n}$ не ограничена. Обратное неверно (контрпример - $x_n = n$(если $n$ четно) и $x_n = 0$ иначе).
    \vspace{0.3cm}
    
    \begin{theorem-non} Единственность предела в $\overline{\mathbb{R}}$
    \end{theorem-non}
    Если $\lim x_n = a \in \overline{\mathbb{R}}$ и $\lim x_n = b \in \overline{\mathbb{R}}$, то $a = b$.
    \begin{proof}
        Пусть $a < b$. 
        
        Если $a,\;b \in \mathbb{R}$, то $a = b$ (должно быть доказано где-то раньше).
        
        Если $a \in \mathbb{R}$ и $b = +\infty$, то в $(a - 1, a + 1)$ и $(a + 1, +\infty)$ должно содержаться бесконечное число членов последовательности, но это невозможно. 
        
        Аналогично для случая  $a = -\infty$ и $b \in \mathbb{R}$.
        
        Если $a = \infty$ и $b = \infty$, то либо $a = b = +\infty$, либо $a = b = -\infty$.
    \end{proof}
    \begin{theorem-non} О стабилизации знака в $\overline{\mathbb{R}}$  \end{theorem-non}
    Если $\lim x_n = a \in \overline{\mathbb{R}}$ и $a \neq 0$, то, начиная с некоторого номера, $x_n$ и $a$ одного знака. 
    \begin{proof}
        Не, ну это очевидно.
    \end{proof}
    \begin{theorem-non} О предельном переходе в неравенстве в $\overline{\mathbb{R}}$ \end{theorem-non}
    \begin{enumerate}
        \item Если $\lim x_n = +\infty$ и $x_n \leqslant y_n \;\forall n$, то $\lim y_n = +\infty$.
        \begin{proof}
            Мы знаем что,
            \[ \forall u\quad \exists N: \forall n \geqslant N \quad x_n > u  \]
            Так как $x_n \leqslant y_n \;\forall n$, то нам подойдет тоже $N$:
            \[ \forall n \geqslant N \quad y_n \geqslant x_n > u  \]
        \end{proof}
        \item Если $\lim y_n = -\infty$ и $x_n \leqslant y_n \;\forall n$, то $\lim x_n = -\infty$.
        \begin{proof}
            Аналогично первому пункту.
        \end{proof}
        \item Если $x_n \leqslant y_n \;\forall n$ и $\lim x_n = a \in \overline{\mathbb{R}},\; \lim y_n = b \in \overline{\mathbb{R}}$, то $a \leqslant b$
        \begin{proof} \quad \\
        \begin{itemize}
            \item $a, b \in R$, доказано ранее
            \item $a = -\infty$, то $a \leqslant b$ всегда
            \item $a = +\infty$, то по первому пункту $b = +\infty$
            \item $b = +\infty$, то $a \leqslant b$ всегда
            \item $b = -\infty$, то по второму пункту $a = -\infty$
        \end{itemize}
        \end{proof}
    \end{enumerate}
    
    \section{Бесконечно большие и малые последовательности}
    \begin{itemize}
        \item $x_n$ называется бесконечно большой, если $\lim x_n = \infty$
        \item $x_n$ называется бесконечно малой, если $\lim x_n = 0$
        \item $x_n$ называется сходящайся, если она имеет конечный предел
    \end{itemize}
    \vspace{0.7cm}
    \begin{theorem-non} Связь между бесконечно большими и бесконечно малыми\end{theorem-non}
    $x_n \neq 0\; \forall n$ \\
    $x_n$ - б.б. $\Leftrightarrow \frac{1}{x_n}$ - б.м.
    \begin{proof}
        $x_n$ - б.б. $\Leftrightarrow \forall u > 0\quad \exists N : \forall n \geqslant N\quad |x_n| > u$ $\Leftrightarrow$ 
        
        $\Leftrightarrow \forall \varepsilon > 0\quad \exists N : \forall n \geqslant N\quad |x_n| > \frac{1}{\varepsilon} \Leftrightarrow \frac{1}{|x_n|} < \varepsilon \Leftrightarrow \frac{1}{x_n}$ - б.м.
    \end{proof}
    \begin{theorem-non} О действиях с бесконечно малыми \end{theorem-non}
    \begin{enumerate}
        \item Сумма / разность б.м. это б.м.
        \begin{proof}
            Предел суммы / разности это сумма / разность пределов. 
        \end{proof}
        \item Произведение б.м. и ограниченной это б.м.
        \begin{proof}
            $y_n$ - ограниченная $\Rightarrow |y_n| \leqslant M$ 
            
            $x_n$ - б.м. $\Rightarrow \forall \varepsilon > 0\quad \exists N : \forall n \geqslant N\quad |x_n| < \frac{\varepsilon}{M}$
            
            $|x_ny_n| \leqslant M|x_n| < \varepsilon$
        \end{proof}
    \end{enumerate}
    
    \section{Арифметические действия в $\overline{\mathbb{R}}$}
    \begin{theorem-non} Об арифметических операциях с $\infty$ \end{theorem-non}
    \begin{enumerate}
        \item $x_n \to +\infty,\; y_n$ - ограниченная снизу $\Rightarrow x_n + y_n \to +\infty$
        \begin{proof}
            $y_n$ - ограниченная снизу $\Rightarrow y_n \geqslant a$ 
            
            $x_n \to +\infty \Rightarrow \forall u \quad \exists N: \forall n \geqslant N \quad x_n > u - a$ 
            
            $\Rightarrow x_n + y_n > u - a + a = u$
        \end{proof}
        \item $x_n \to -\infty,\; y_n$ - ограниченная сверху $\Rightarrow x_n + y_n \to -\infty$
        \begin{proof}
            Аналогично предыдущему пункту.
        \end{proof}
        \item $x_n \to \infty,\; y_n$ - ограниченная $\Rightarrow x_n \pm y_n \to \infty$
        \begin{proof}
            Аналогично первому пункту.
        \end{proof}
        \item $x_n \to \pm \infty,\; y_n \geqslant c > 0 \Rightarrow x_ny_n \to \pm \infty$
        \begin{proof}
            $x_n \to +\infty \Rightarrow \forall u \quad \exists N: \forall n \geqslant N \quad x_n>\frac{u}{c}$
            
            $y_n \geqslant c > 0 \Rightarrow x_ny_n \geqslant cx_n > u$
            
            Случай $x_n \to -\infty$ рассматривается аналогично.
        \end{proof}
        \item $x_n \to \pm \infty,\; y_n \leqslant c < 0 \Rightarrow x_ny_n \to \mp \infty$
        \begin{proof}
            Аналогично предыдущему пункту.
        \end{proof}
        \item $x_n \to \infty,\; |y_n| \geqslant c > 0 \Rightarrow x_ny_n \to \infty $
        \begin{proof}
            Аналогично четвертому пункту.
        \end{proof}
        \item $x_n \to a \neq 0,\; y_n \neq 0 \to 0 \Rightarrow \frac{x_n}{y_n} \to \infty$
        \begin{proof}
            $\lim \frac{y_n}{x_n} = 0 \Rightarrow \frac{y_n}{x_n}$ - б.м. $\Rightarrow \frac{x_n}{y_n}$ - б.б. $\Rightarrow \lim \frac{x_n}{y_n} = \infty$ 
        \end{proof}
        \item $x_n$ - ограниченная, $y_n \to \infty \Rightarrow \frac{x_n}{y_n} \to 0$
        \begin{proof}
            $y_n \to \infty \Rightarrow \frac{1}{y_n}$ - б.м. $\Rightarrow x_n * \frac{1}{y_n}$ - б.м.
        \end{proof}
        \item $x_n \to \infty,\; y_n \neq 0$ - ограниченная $\Rightarrow \frac{x_n}{y_n} \to \infty$
        \begin{proof}
            $y_n$ - ограниченная $\Rightarrow |y_n| \leqslant M$
            
            $x_n \to \infty \Rightarrow \forall u > 0 \quad \exists N : \forall n \geqslant N \quad |x_n| > uM \Rightarrow |\frac{x_n}{y_n}| \geqslant |\frac{x_n}{M}| > u$
        \end{proof}
    \end{enumerate}
    \vspace{0.7cm}
    Запрещенные операции:
    \begin{itemize}
        \item $+\infty \pm (\mp\infty)$
        \item $-\infty \pm (\pm\infty)$
        \item $\pm \infty * 0$
        \item $\frac{0}{0}$
        \item $\frac{\pm \infty}{\pm \infty}$
    \end{itemize}
    \vspace{0.3cm}
    Почему эти операции запрещенные? Разберем на примере:
    
    $\lim x_n = \lim y_n = +\infty$ \\
    $x_n - y_n$  может иметь любой предел в $\overline{\mathbb{R}}$, а может его вообще не иметь:
    \begin{itemize}
        \item $x_n = n + a,\; y_n = n \Rightarrow x_n - y_n = a \to a$
        \item $x_n = 2n,\; y_n = n \Rightarrow x_n - y_n = n \to +\infty$
        \item $x_n = n + (-1)^n,\; y_n = n \Rightarrow x_n - y_n = (-1)^n$ - предела не имеет
    \end{itemize}
    
    \section{Неравенство Бернулли}
    \[ (1 + x)^n \geqslant 1 + nx \quad x > -1,\; n \in \mathbb{N} \]
    \begin{proof}
        Индукция по $n$.
        
        База $n = 1: (1 + x) = 1 + x$
        
        Переход $n \to n + 1: (1 + x)^{n + 1} = \underbrace{(1 + x)}_{> 0}\underbrace{(1 + x)^n}_{assumption} \geqslant (1 + x)(1 + nx) = 1 + (n + 1)x + nx^2 \geqslant 1 + (n + 1)x$
    \end{proof}
    \underline{Замечание 1:} В неравенсте Бернулли почти всегда строгий знак, равенство достигается только в случаях, когда $n = 1$ или $x = 0$.
    
    \underline{Замечание 2:} $(1 + x)^p \geqslant 1 + px \quad x > -1$ верно при всех $p \geqslant 1$ и $p \leqslant 0$. Какая-то жесткая тема. Дали без доказателства.
    \vspace{0.5cm}
    
    \textbf{Следствие.} 
    \begin{enumerate}
        \item Если $a > 1$, то $\lim a^n = +\infty$.
        \begin{proof}
            $a > 1 \Rightarrow a = 1 + x \quad x > -1$
            
            $a^n = (1 + x)^n \geqslant 1 + xn \to +\infty$
        \end{proof}
        \item Если $|a| < 1$, то $\lim a^n = 0$.
        \begin{proof}
            Считаем, что $a \neq 0$.
            
            $|\frac{1}{a}| > 1 \Rightarrow \lim |\frac{1}{a}|^n = +\infty \Rightarrow |\frac{1}{a}|^n$ - б.б. $\Rightarrow |a^n|$ - б.м. $\Rightarrow a^n$ - б.м.
        \end{proof}
    \end{enumerate}
    
    \section{Определение экспоненты}
    Рассмотрим последовательность $x_n = (1 + \frac{a}{n})^n$, где $a \in \mathbb{R}$
    \begin{theorem-non}
    $x_n$ монотонно возрастает, начиная с $n > -a$ и ограничена сверху
    \end{theorem-non}
    \begin{proof} \quad \\
        \begin{enumerate}
        \item Монотонное возрастание (если $a < 0$, то с номера $n = -a + 1$)
        \begin{equation*}
            \begin{split}
                \frac{x_n}{x_{n - 1}} &= \frac{(1 + \frac{a}{n})^n}{(1 + \frac{a}{n - 1})^{n - 1}} \\ 
                &= \frac{\frac{(n + a)^n}{n^n}}{\frac{(n - 1 + a)^{n - 1}}{(n - 1)^{n - 1}}} \\ 
                &= \frac{(n - 1)^{n - 1}}{n^n} * \frac{(n + a)^n}{(n - 1 + a)^{n - 1}} \\
                &= \frac{(n - 1)^n * (n + a)^n}{n^n * (n - 1 + a)^n } * \frac{n - 1 + a}{n - 1} \\
                &= (\frac{n^2 - n + an - a}{n^2 - n + an})^n * \frac{n - 1 + a}{n - 1} \\ 
                &= \underbrace{(1 - \frac{a}{n(n - 1 + a)})^n}_{\geqslant 1 - \frac{na}{n(n - 1 + a)} \;by\; Bernoulli's\; inequality} * \frac{n - 1 + a}{n - 1} \\
                &\geqslant\frac{n - 1}{n - 1 + a} * \frac{n - 1 + a}{a} = 1
            \end{split}
        \end{equation*}
        \item Ограниченность сверху
        
        $y_n = (1 - \frac{a}{n})^n$ монотонно возрастает при $n > a$ 
        
        $x_ny_n = (1 + \frac{a}{n})^n * (1 - \frac{a}{n})^n = (1 - (\frac{a}{n})^2)^n \leqslant 1$
        
        $y_n \geqslant c > 0$, начиная с некоторого номера $\Rightarrow 1 \geqslant x_ny_n \geqslant cx_n \Rightarrow x_n \leqslant \frac{1}{c}$, начиная с некоторого номера $\Rightarrow x_n$ - ограниченная
        \end{enumerate}
    \end{proof}
    \textbf{Следствие.} Существует конечный $\lim (1 + \frac{a}{n})^n$ 
    \begin{conj} \quad \\
        \begin{enumerate}
            \item $exp\,a := \lim (1 + \frac{a}{n})^n$
            \item $e := \lim (1 + \frac{1}{n})^n \approx 2,71828$
        \end{enumerate}
    \end{conj}
    \underline{Замечание:} Последовательность $x_n = (1 + \frac{a}{n})^n$ при $a \neq 0$ \underline{строго} монотонно возрастает c $n > -a$. В доказателсьстве пользовались неравенством Бернулли, при $a \neq 0$ в нем строгий знак.
    \vspace{0.5cm}
    
    \textbf{Следствие.} Последовательность $z_n = (1 + \frac{1}{n})^{n+1}$ строго убывает и стремиться к $e$
    \begin{proof}
        $z_n = \underbrace{(1 + \frac{1}{n})}_{\to 1} * \underbrace{(1 + \frac{1}{n})^n}_{\to e} \to e$
        
        $z_n = \frac{n + 1}{n}^{n+1} = \frac{1}{(\frac{n}{n+1})^{n+1}} = \frac{1}{(1 - \frac{1}{n+1})^{n+1}}$
        
        Последовательность $(1 - \frac{1}{n+1})^{n+1}$ строго возрастает, следовательно, обратная к ней строго убывает. 
    \end{proof}
    
    \section{Свойства экспоненты}
    \begin{enumerate}
        \item Для любого $a \in \mathbb{R} \quad exp\,a > 0$
        \item $exp\,0 = 1,\; exp\,1 = e$
        \item Если $a \leqslant b$, то $exp\,a \leqslant exp\,b$
        \begin{proof}
            $0 < 1 + \frac{a}{n} \leqslant 1 + \frac{b}{n}$ при $n > -a \Rightarrow \underbrace{(1 + \frac{a}{n})^n}_{\to exp\,a} \leqslant \underbrace{(1 + \frac{b}{n})^n}_{\to exp\,b}$ при $n > -a$
        \end{proof}
        \item $exp\,a \geqslant 1 + a$
        \begin{proof}
            По неравенству Бернулли:
            
            $\underbrace{(1 + \frac{a}{n})^n}_{\to exp\,a} \geqslant 1 + n * \frac{a}{n} = 1 + a$ при $n > -a$
        \end{proof}
        \item $exp\,a * exp\,(-a) \leqslant 1$
        \begin{proof}
            $\underbrace{(1 + \frac{a}{n})^n}_{\to exp\,a} * \underbrace{(1 - \frac{a}{n})^n}_{\to exp\,(-a)} = (1 - (\frac{a}{n})^2)^n \leqslant 1$
        \end{proof}
        \item $exp\,a \leqslant \frac{1}{1 - a}$ при $a < 1$
        \begin{proof}
            С помощью двух предыдущих пунктов
            
            $exp\,a \leqslant\frac{1}{exp\,(-a)} \leqslant \frac{1}{1 - a}$
        \end{proof}
        \item $(1 + \frac{1}{n})^n < e < (1 + \frac{1}{n})^{n + 1}$ при всех $n$
        \begin{proof}
            $(1 + \frac{1}{n})^n < (1 + \frac{1}{n+1})^{n+1} \leqslant \underbrace{(1 + \frac{1}{m})^{m}}_{\to e}$ при $m
            \geqslant n + 1 \Rightarrow (1 + \frac{1}{n})^n < e$
            
            $(1 + \frac{1}{n})^{n + 1} > (1 + \frac{1}{n+1})^{n + 2} \geqslant \underbrace{(1 + \frac{1}{m})^{m + 1}}_{\to e}$ при $m \geqslant n + 1 \Rightarrow (1 + \frac{1}{n})^{n + 1} > e$
        \end{proof}
        В частности, подставив $n = 1$ и $n = 5$ получаем, что $2 < e < 3$
    \end{enumerate}
    
    \section{Формула для экспоненты суммы}
    \textbf{Лемма.}
        Если $\lim a_n = a \in \mathbb{R}$, то $\lim (1 + \frac{a_n}{n})^n = exp\,a$    
    \begin{proof}
        Последовательность $a_n$ ограничена $\Rightarrow a_n \leqslant M,\; a \leqslant M$ и $M > 0$
        
        $A := 1 + \frac{a}{n} \leqslant 1 + \frac{M}{n} \quad B:= 1 + \frac{a_n}{n} \leqslant 1 + \frac{M}{n}$
        
        Надо доказать, что $\lim(A^n - B^n) = 0$
        \begin{equation*}
            \begin{split}
                |A^n - B^n| &= |A - B|(A^{n-1} + A^{n-2}B + \dots + B^{n-1}) \\ &\leqslant |A-B|n(1 + \frac{M}{n})^{n-1} \\
                &\leqslant |A-B|n(1 + \frac{M}{n})^n \\ 
                &= \frac{|a-a_n|}{n}n(1 + \frac{M}{n})^n \\
                &= |a - a_n|(1 + \frac{M}{n})^n \leqslant \underbrace{|a-a_n|}_{\to 0}*exp\,M
            \end{split}
        \end{equation*}
    \end{proof}
    \begin{theorem-non}
        $exp\,(a+b) = exp\,a * exp\,b$
    \end{theorem-non}
    \begin{proof}
        \[ \underbrace{(1 + \frac{a}{n})^n}_{\to\,exp\,a} * \underbrace{(1 + \frac{b}{n})^n}_{\to\,exp\,b} = (1 + \frac{a+b}{n} + \frac{ab}{n^2})^n = \underbrace{(1 + \frac{a + b + \frac{ab}{n}}{n})^n}_{a + b + \frac{ab}{n} := a_n \to\,a + b} = \underbrace{(1 + \frac{a_n}{n})^n}_{\to\,exp\,(a+b)} \]
    \end{proof}
    
    \section{Сравнение скорости возрастания последовательностей}
    \begin{theorem-non}
        Пусть $x_n > 0$ и $\lim \frac{x_{n+1}}{x_n} < 1$. Тогда $x_n \to\,0$
    \end{theorem-non}
    \begin{proof} \quad \\\\
        $l := \lim \frac{x_{n+1}}{x_n}$. Начиная с некоторого номера $m \;\; \frac{x_{n+1}}{x_n} < \frac{1 + l}{2} =: q < 1$
        
        При $n \geqslant m$ 
        \[ 0 < x_n < \frac{x_n}{x_{n-1}} * \frac{x_{n-1}}{x_{n-2}} * \frac{x_{n-2}}{x_{n-3}} * \dots * \frac{x_{m+1}}{x_m} * x_m < q^{n-m}x_m = q^n * \frac{x_m}{q^m} \]
        \[ 0 < x_n < q^n * \frac{x_m}{q^m} \to 0 \Rightarrow x_n \to 0  \]
    \end{proof}
    \textbf{Следствие.}
    \begin{enumerate}
        \item $\lim \frac{n^k}{a^n} = 0$ при $a > 1$ (показательная функция растет быстрее полиномиальной)
        \begin{proof}
            $x_n = \frac{n^k}{a^n}$
            \[ \frac{x_{n+1}}{x_n} =
            \frac{(n+1)^ka^n}{a^{n+1}n^k} = 
             (\frac{n+1}{n})^k * \frac{a^n}{a^{n+1}} = \frac{1}{a} * (1 + \frac{1}{n})^k \to \frac{1}{a} < 1 \Rightarrow x_n \to 0\]
        \end{proof}
        \item $\lim \frac{a^n}{n!} = 0$ (факториал растет быстрее показательной)
        \begin{proof}
            $x_n = \frac{a^n}{n!}$
            \[ \frac{x_{n+1}}{x_n} =  \frac{a^{n+1}n!}{(n+1)!a^n}
            = a\frac{n!}{(n+1)!} = \frac{a}{n+1} \to 0 < 1 \Rightarrow x_n \to 0 \]
        \end{proof}
        \item $\lim \frac{n!}{n^n} = 0$
        \begin{proof}
            $x_n = \frac{n!}{n^n}$
            \[ \frac{x_{n+1}}{x_n} = \frac{(n+1)!n^n}{(n+1)^{n+1}n!} = \frac{(n+1)n^n}{(n+1)^{n+1}} = (\frac{n}{n+1})^n = \frac{1}{(\frac{n + 1}{n})^n} = \frac{1}{(1 + \frac{1}{n})^n} \to \frac{1}{e} < 1 \Rightarrow x_n \to 0 \]
        \end{proof}
    \end{enumerate}

    \section{Теорема Штольца (для неопределённости $\frac \infty \infty$)}

    \begin{theorem-non} Штольца № 1 \end{theorem-non}
        Пусть $(y_n)$ строго возрастает и $\lim y_n = +\infty$. 
        Тогда если $\lim \frac{x_n - x_{n - 1}}{y_n - y_{n - 1}} = l \in
        \overline{\mathbb{R}}$, то $\lim \frac{x_n}{y_n} = l$.
    \begin{proof}
        \textbf{Ключевой случай $l = 0$:}
    
        Пусть \[\varepsilon_n := 
        \frac{x_n - x_{n - 1}}{y_n - y_{n - 1}} \rightarrow 0\]
    
        Зафиксируем $\varepsilon > 0$ и найдём $m$, т.ч. $
        \abs{\varepsilon_n} < \varepsilon$ при $n \geq m$.
    
        \[x_n - x_m = (x_n - x_{n - 1}) + (x_{n - 1} - x_{n - 2})
        + ... + (x_{m + 1} - x_m) = \sum_{k=m+1}^n \varepsilon_k
        \cdot (y_k - y_{k - 1})\]
        \[\abs{x_n - x_m} \leq \sum_{k=m+1}^n \abs{\varepsilon_k}
        \cdot (y_k - y_{k - 1}) < \sum_{k=m+1}^n \varepsilon
        \cdot (y_k - y_{k - 1}) = \varepsilon \cdot \sum_{k=m+1}^n 
        (y_k - y_{k - 1}) = \varepsilon \cdot (y_n - y_m) <
        \varepsilon y_n\]
    
        Можно считать, что $y_m > 0$ (по теореме о стабилизации знака).
    
        Заметим, что $\abs{x_m}$ фиксировано, а $y_n \rightarrow +\infty$
        $\Rightarrow$ $\lim \frac{\abs{x_m}}{y_n} = 0$ и
        $\frac{\abs{x_m}}{y_n} < \varepsilon$, начиная с некоторого номера.
    
        \[\abs{x_n} \leq \abs{x_m} + \abs{x_n - x_m} < 
        \abs{x_m} + \varepsilon y_n \Rightarrow \abs{\frac{x_n}{y_n}} <
        \frac{\abs{x_m}}{y_n} + \varepsilon < 2 \varepsilon\]
        начиная с некоторого номера $\Rightarrow \lim \abs{\frac{x_n}{y_n}} = 0 = l$.
    
        \textbf{Случай $l \in \mathbb{R}$:}
        \[\widetilde{x_n} := x_n - l \cdot y_n, \widetilde{x_n} -
        \widetilde{x_{n-1}} = x_n - x_{n-1} - l \cdot (y_n - y_{n-1})\]
        \[\frac{\widetilde{x_n} - \widetilde{x_{n-1}}}
        {y_n - y_{n - 1}} = \frac{x_n - x_{n - 1}}{y_n - y_{n - 1}}  - l
        \rightarrow 0 \xRightarrow[]{l = 0} \frac{\widetilde{x_n}}{y_n}
        \rightarrow 0 \Rightarrow \frac{\widetilde{x_n}}{y_n} =
        \frac{x_n - l \cdot y_n}{y_n} = \frac{x_n}{y_n} - l
        \rightarrow 0 \Rightarrow \frac{x_n}{y_n} \rightarrow l\]
    
        \textbf{Случай $l = +\infty$:}
    
        \[\frac{x_n - x_{n - 1}}{y_n - y_{n - 1}} \rightarrow +\infty
        \Rightarrow \frac{x_n - x_{n - 1}}{y_n - y_{n - 1}} > 1\] 
        начиная с некоторого номера \\
        $\Rightarrow x_n - x_{n - 1} > y_n - y_{n - 1} > 0 \Rightarrow
        x_n$ строго возрастает с нек. номера $m \Rightarrow$\\
        $\Rightarrow x_n - x_m > y_n - y_m \Rightarrow
        x_n > y_n + (x_m - y_m) \rightarrow +\infty \Rightarrow
        x_n \rightarrow +\infty$
    
        Рассмотрим
        \[\frac{y_n - y_{n - 1}}{x_n - x_{n - 1}} \rightarrow 0
        \xRightarrow[]{l = 0} \frac{y_n}{x_n} \rightarrow 0 \Rightarrow
        \frac{x_n}{y_n} \rightarrow +\infty\] (а не $\infty$, т.к. 
        $x_n > 0, y_n > 0$ с нек. номера)
    
        \textbf{Случай $l = -\infty$}
    
        Пусть $\widetilde{x_n} := -x_n$.
        \[\frac{x_n - x_{n-1}}{y_n - y_{n - 1}} \rightarrow -\infty
        \Rightarrow \frac{\widetilde{x_n} - \widetilde{x_{n-1}}}
        {y_n - y_{n - 1}} = -\frac{x_n - x_{n-1}}{y_n - y{n - 1}}
        \rightarrow +\infty \Rightarrow -\frac{x_n}{y_n} =
        \frac{\widetilde{x_n}}{y_n} \rightarrow +\infty
        \Rightarrow \frac{x_n}{y_n} \rightarrow -\infty\] 
    
    \end{proof}
    
    \textbf{Следствие.}
    \[\text{Если } \lim a_n = a \in \overline{\mathbb{R}}
    \text{, то } \lim \frac{a_1 + a_2 + ... + a_n}{n} = a\] 
    \begin{proof}
        \[x_n := \sum_{k=1}^n a_k, \quad y_n := n \nearrow +\infty\]
        \[\lim \frac{x_n - x_{n-1}}{y_n - y_{n-1}} = \lim{a_n}{n - (n - 1)}
        = \lim a_n = a \Rightarrow \lim \frac{a_1 + a_2 + ... + a_n}{n} = a\]
    \end{proof}
    
    Пример. Найти предел:
    \[m \in \mathbb{N}, \quad \frac{1}{n^{m + 1}} \cdot \sum_{k=1}^{n} k^m\]
    
    \[x_n := \sum_{k=1}^{n} k^m, \quad y_n := n^{m + 1} \nearrow +\infty\]
    \begin{gather*}
        \lim \frac{y_n - y_{n - 1}}{x_n - x_{n - 1}} =
        \lim \frac{n^{m + 1} - (n - 1)^{m + 1}}{n^m} =
        \lim \frac{n^{m+1} - (n^{m+1} + \sum_{k = 1}^{m + 1} 
        (C_{m+1}^k (-1)^k n^{m+1-k})}{n^m}) = \\
        = \lim \sum_{k = 1}^{m + 1} 
        ((-1)^{k+1} \cdot \frac{C_{m+1}^k}{n^{k - 1}}) =
        \lim C_{m + 1}^1 + \lim \sum_{k = 2}^{m + 1} 
        ((-1)^{k+1} \cdot \frac{C_{m+1}^k}{n^{k - 1}}) =
        (m + 1) + 0 = m + 1
    \end{gather*}
    \[\lim\frac{x_n - x_{n - 1}}{y_n - y_{n-1}} = \frac1{m + 1}
    \Rightarrow \lim\frac{x_n}{y_n} = \frac1{m + 1}\]
    
    \section{Теорема Штольца (для неопределённости $\frac 0 0$)}
    
    \begin{theorem-non}Штольца № 2\end{theorem-non}
    \[0<y_n<y_{n-1} \text{ и } \lim x_n = \lim y_n = 0
    \text{ Тогда если }\lim\frac{x_n - x_{n-1}}{y_n - y_{n-1}} = l \in
    \overline{\mathbb{R}}\text{, то }\lim \frac{x_n}{y_n} = l\]
    
    \begin{proof}
        \textbf{Случай $l = 0$:}
    
        Пусть \[\varepsilon_n := 
        \frac{x_n - x_{n - 1}}{y_n - y_{n - 1}} \rightarrow 0\]
    
        Зафиксируем $\varepsilon > 0$ и найдём $m$, т.ч. $
        \abs{\varepsilon_n} < \varepsilon$ при $n \geq m$.
        \[x_n - x_m = \sum_{k=m+1}^n(x_k - x_{k-1}) =
        \sum_{k=m+1}^n \varepsilon_k (y_k - y_{k-1}) \Rightarrow
        \abs{x_n - x_m} \leq\] \[\leq \sum_{k=m+1}^n -\abs{\varepsilon_k}
        (y_k - y_{k-1}) < \varepsilon \sum_{k=m+1}^n (y_k - y_{k-1}) =
        \varepsilon (y_m - y_n)\]
        \[(x_n - x_m) < \varepsilon (y_m - y_n)\]
        \[\text{Устремим } n \text{ к } +\infty \Rightarrow
        \abs{x_n - x_m} \rightarrow \abs{-x_m} = x_m, \quad
        \varepsilon (y_m - y_n) \rightarrow \varepsilon y_m \Rightarrow\]
        \[\Rightarrow \text{по пред. переходу в нер., при }
        m \geq \text{нек. } N \quad \abs{x_m} < \varepsilon y_m
        \Rightarrow \abs{\frac{x_m}{y_m}} < \varepsilon
        \Rightarrow \lim \frac{x_m}{y_m} = 0\]
    
        \textbf{Случай $l \in \overline{\mathbb{R}}$:}
        Так же, как в теореме Штольца № 1
        \[\widetilde{x_n} := x_n - l \cdot y_n, \widetilde{x_n} -
        \widetilde{x_{n-1}} = x_n - x_{n-1} - l \cdot (y_n - y_{n-1})\]
        \[\frac{\widetilde{x_n} - \widetilde{x_{n-1}}}
        {y_n - y_{n - 1}} = \frac{x_n - x_{n - 1}}{y_n - y_{n - 1}}  - l
        \rightarrow 0 \xRightarrow[]{l = 0} \frac{\widetilde{x_n}}{y_n}
        \rightarrow 0 \Rightarrow \frac{\widetilde{x_n}}{y_n} =
        \frac{x_n - l \cdot y_n}{y_n} = \frac{x_n}{y_n} - l
        \rightarrow 0 \Rightarrow \frac{x_n}{y_n} \rightarrow l\]
    
        \textbf{Случай $l = +\infty$:}
        \[\frac{x_n - x_{n - 1}}{y_n - y_{n - 1}} \rightarrow +\infty
        \Rightarrow \frac{x_{n - 1} - x_n}{y_{n - 1} - y_{n}} =
        \frac{x_n - x_{n - 1}}{y_n - y_{n - 1}} > 1 \text{ начиная
        с некоторого номера} \Rightarrow\] \[\Rightarrow
        x_{n - 1} - x_n > y_{n - 1} - y_n > 0 \Rightarrow x_n
        \text{ строго убывает} \Rightarrow \lim \frac{y_n - y_{n - 1}}
        {x_n - x_{n - 1}} = 0 \xRightarrow[]{l = 0} \lim \frac{y_n}{x_n} = 0
        \Rightarrow \] \[\Rightarrow \frac{x_n}{y_n} = +\infty\]
    
        \textbf{Случай $l = -\infty$:}
        Так же, как в теореме Штольца № 1
    
        Пусть $\widetilde{x_n} := -x_n$.
        \[\frac{x_n - x_{n-1}}{y_n - y_{n - 1}} \rightarrow -\infty
        \Rightarrow \frac{\widetilde{x_n} - \widetilde{x_{n-1}}}
        {y_n - y_{n - 1}} = -\frac{x_n - x_{n-1}}{y_n - y{n - 1}}
        \rightarrow +\infty \Rightarrow -\frac{x_n}{y_n} =
        \frac{\widetilde{x_n}}{y_n} \rightarrow +\infty
        \Rightarrow \frac{x_n}{y_n} \rightarrow -\infty\] 
    
    \end{proof}
    
    \section{Подпоследовательности.
    Теорема о стягивающихся отрезках}
    
    \begin{conj}
    Последовательность $(x_n)$, $n_1 < n_2 < n_3 < ...$ Тогда
    $(x_{n_k})$ - подпоследовательность.
    \end{conj}
    \textbf{Замечание.} $n_k \geq k$ (по индукции)
    
    \textbf{Свойства:}
    \begin{enumerate}
        \item Если последовательность имеет предел, то подпоследовательность
        имеет тот же предел.
        \item Пусть две подпоследовательности в объединении дают исходную
        последовательность. Если подпоследовательности имеют одинаковый
        предел, то исходная последовательность имеет тот же предел.
    \end{enumerate}
    
    \begin{theorem-non}О стягивающихся отрезках.\end{theorem-non}
    \[\text{Пусть }[a_1; b_1] \supset [a_2; b_2] \supset [a_3; b_3] 
    \supset ... \text{ и } \lim (b_n - a_n) = 0\]
    Тогда существует единственная точка $c$, принадлежащая всем отрезкам
    и $\lim a_n = \lim b_n = c$.
    \[\text{Т.е. } \bigcap_{n = 1}^{+\infty} [a_n; b_n] = {c}\]
    
    \begin{proof}
        По теореме о вложенных отрезках $\bigcap_{n = 1}^{+\infty} [a_n; b_n]
        \neq \varnothing$.
        \[\text{Пусть } c,d \in \bigcap_{n = 1}^{+\infty} [a_n; b_n]
        \Rightarrow c, d \in [a_n; b_n] \forall n; \text{ НУО, } d \geq c\]
        \[0 \leq d - c \leq b_n - a_n \rightarrow 0 \Rightarrow c = d
        \text{, иначе } \exists n : b_n - a_n < \varepsilon = d - c\]
        \[0 \leq c - a_n \leq b_n - a_n \rightarrow 0
        \xRightarrow[]{\text{2 мил.}}
        c - a_n \rightarrow 0 \Rightarrow \lim a_n = c\]
        \[0 \leq b_n - c \leq b_n - a_n \rightarrow 0
        \xRightarrow[]{\text{2 мил.}}
        b_n - c \rightarrow 0 \Rightarrow \lim b_n = c\]
    \end{proof}
    
    \section{Теорема Больцано-Вейерштрасса в $\mathbb{R}$}
    \begin{theorem-non}
        Из любой ограниченной последовательности можно выделить
        сходящуюся подпоследовательность.
    \end{theorem-non}
    \begin{proof}
    ${x_n}$ ограничено $\Rightarrow x_n \in [a; b]$
    
    В каком-то из отрезков $[a; \frac{a + b}{2}]$ и $[\frac{a + b}{2}; b]$
    содержится бесконечное число членов послед.\\
    Назовём этот отрезок $[a_1; b_1]$.
    
    В каком-то из отрезков $[a_1; \frac{a_1 + b_1}{2}]$ и 
    $[\frac{a_1 + b_1}{2}; b_1]$
    содержится бесконечное число членов послед.\\
    Назовём этот отрезок $[a_2; b_2]$.
    
    В каком-то из отрезков $[a_2; \frac{a_2 + b_2}{2}]$ и 
    $[\frac{a_2 + b_2}{2}; b_2]$
    содержится бесконечное число членов послед.\\
    Назовём этот отрезок $[a_3; b_3]$.
    \[...\]
    \[[a; b] \supset [a_1; b_1] \supset [a_2; b_2] \supset
    [a_3; b_3] \supset ...\]
    \[b_n - a_n = \frac{b - a}{2^n} \rightarrow 0\]
    
    Тогда по теореме о стягивающихся отрезках $\lim a_n = \lim b_n = c$
    
    Выберем подпоследовательность. Берём $[a_1; b_1]$, в нём есть
    какой-то член последовательности, назовём его $x_{n_1}$.
    
    В $[a_2; b_2]$ содержится бесконечное число членов последовательности
    $\Rightarrow$ есть член последовательности с номером, большим $n_1$.
    Обозначим его $x_{n_2}$, тогда $n_2 > n_1$.
    \[...\]
    $x_{n_k} \in [a_k; b_k], n_1 < n_2 < n_3 < ...$, значит построили
    подпоследовательность.
    
    \[a_k \rightarrow c, \,\, b_k \rightarrow c \quad a_k \leq x_{n_k} \leq b_k
    \xRightarrow[]{\text{2 мил.}} \lim x_{n_k} = c \]
    \end{proof}
    
    \section{Аналог теоремы Больцано–Вейерштрасса для неограниченной 
    последовательности. Частичные пределы. Теорема о характеристике 
    частичных пределов.}
    
    \begin{theorem-non}\end{theorem-non}
    \begin{enumerate}
        \item Неограниченная монотонная последовательность стремится
        к $+\infty$ или к $-\infty$.
        \item Из любой неограниченной последовательности можно выделить
        подпоследовательность, стремящуюся к $+\infty$ или к $-\infty$.
    \end{enumerate}
    \begin{proof}.
    \begin{enumerate}
        \item Пусть $(x_n)$ возрастает. $(x_n)$ неограничена $\Rightarrow$
        никакое $u$ не является верхней границей $\Rightarrow \exists m : x_m
        x_m > u \Rightarrow u < x_m \leq x_{m + 1} \leq x_{m + 2} \leq \dots
        \Rightarrow x_n > u$, начиная с некоторого номера $\Rightarrow
        \lim x_n = +\infty$
    
        \item Пусть $(x_n)$ неограничена сверху.
        
        $1$ не является верхней границей $\Rightarrow \exists x_{n_1} > 1$;\\
        $\max\{2, x_1, x_2, \dots, x_{n_1}\}$ не является верхней границей
        $\Rightarrow \exists x_{n_2} > \max\{\dots\} \Rightarrow x_{n_2} > 2,\\
        n_2 > n_1$;\\
        $\max\{3, x_1, x_2, \dots, x_{n_2}\}$ не является верхней границей
        $\Rightarrow \exists x_{n_3} > \max\{\dots\} \Rightarrow x_{n_3} > 3,\\
        n_3 > n_2$;\\
        и т.д.
    
        Итого, $x_{n_k} > k$ и $n_1 < n_2 < \dots \Rightarrow (x_{n_k})$
        -- подпоследовательность $(x_n)$ и $\lim x_{n_k} = +\infty$ по
        предельному переходу в неравенстве.
    \end{enumerate}
    \end{proof}
    
    \begin{conj}
    $a$ -- частичный предел последовательности $(x_n)$, если найдётся
    подпоследовательность $x_{n_k} \rightarrow a$.
    \end{conj}
    \begin{theorem-non}
    $a$ -- частичный предел последовательности $\Leftrightarrow$
    в любой окрестности точки $a$ найдётся бесконечное число членов
    последовательности.
    \end{theorem-non}
    \begin{proof} $ $
    
        ''$\Longrightarrow$'':
    
        Если $a = \lim x_{n_k}$ и $U_a$ -- окрестность точки $a$, то
        все $x_{n_k}$ кроме конечного числа лежат в $U_a \Rightarrow$
        в $U_a$ лежит бесконечное число членов последовательности $(x_n)$.
    
        ''$\Longleftarrow$'':
    
        Будем строить подпоследовательность, имеющую предел $a$.
    
        В $B_{1}(a)$ найдётся бесконечное число членов последовательности,
        возьмём какой-то и назовём его $x_{n_1}$.\\
        В $B_{1/2}(a)$ найдётся бесконечное число членов
        последовательности, значит найдётся член $(x_n)$ с индексом, большим
        $n_1$, назовём его $x_{n_2}$.\\
        В $B_{1/3}(a)$ найдётся бесконечное число членов
        последовательности, значит найдётся член $(x_n)$ с индексом, большим
        $n_2$, назовём его $x_{n_3}$.\\
        $\dots$
    
        $n_1 < n_2 < n_3 < \dots$\\
        $x_{n_k} \in B_{1/k}(a) \Rightarrow \rho(x_{n_k}, a) < \frac1k
        \Rightarrow \rho(x_{n_k}, a) \rightarrow 0 \Rightarrow
        \lim x_{n_k} = a$
    
    \end{proof}
    
    \section{Фундаментальные последовательности. Критерий Коши.}
    
    \begin{conj}
    Фундаментальная последовательность (сходящаяся в себе,
    последовательность Коши)
    \end{conj}
    Пусть $(X, \rho)$ -- метрическое пространство. $x_n \in X$.
    $x_n$ -- фундаментальная последовательность, если $\forall
    \varepsilon > 0 \,\, \exists N : \forall n, m \geq N \,\,
    \rho(x_n, x_m) < \varepsilon$
    
    \textbf{Свойства:}
    \begin{enumerate}
        \item Сходящаяся последовательность фундаментальна.
        
        \textbf{Доказательство:}\\
        Пусть $\lim x_n := a$. Зафиксируем $\varepsilon > 0$.
        Тогда $\exists N :\\ \forall n \geq N \,\,\, \rho(x_n, a) < 
        \frac{1}{2} \varepsilon$\\ 
        $\forall m \geq N \,\,\, \rho(x_m, a) < \frac{1}{2} \varepsilon$\\
        $\Rightarrow \rho(x_n, x_m) \leq \rho(x_n, a) + \rho(x_m, a) <
        \varepsilon$
    
        \item Фундаментальная последовательность ограничена
        
        \textbf{Доказательство:}\\
        Берём $\varepsilon = 1$. Тогда $\exists N : \forall n, m \geq N \,\,
        \rho(x_n, x_m) < 1 \Rightarrow$\\
        $\Rightarrow \forall n \geq N \,\, \rho(x_n, x_N) < 1
        \Leftrightarrow x_n \in B_1(x_N)$\\
        $R := \max\{\rho(x_1, x_N), \rho(x_2, x_N), \dots, 
        \rho(x_{N-1}, x_N)\} \Rightarrow \forall n \,\, x_n \in B_R(x_N)$
    
        \item Если у фундаментальной последовательности есть сходящаяся
        подпоследовательность, то фундаментальная последовательность
        имеет тот же предел.
    
        \textbf{Доказательство:}\\
        Пусть $\lim x_{n_k} = a$. Зафиксируем $\varepsilon > 0$.\\
        $\exists K : \forall k \geq K \quad \rho(x_k, a) <
        \frac{1}{2} \varepsilon$\\
        $\exists N : \forall n, m \geq N \quad \rho(x_n, x_m) <
        \frac{1}{2} \varepsilon$\\
        Возьмём $N \geq 0$ и подберём такое $k$, что $k \geq N$\\
        и $n_k \geq N$ (например, $k \geq \max{N, K}$ подходит)\\
        Тогда $\rho(x_n, x_{n_k}) < \frac{1}{2} \varepsilon$
        (т.к. $n_k \geq N$)\\
        И тогда $\rho(x_{n_k}, a) < \frac{1}{2} \varepsilon$
        (т.к. $k \geq K$)\\
        $\Rightarrow \rho(x_n, a) \leq \rho(x_n, x_{n_k}) +
        \rho(x_{n_k}, a) < \varepsilon \Rightarrow \lim x_n = a$
    \end{enumerate}
    
    \begin{theorem-non}Критерий Коши\end{theorem-non}
    Числовая последовательность имеет предел $\Leftrightarrow$
    она фундаментальна.
    
    \begin{proof} $ $
    
    ''$\Longrightarrow$'':\\
    По свойству 1.
    
    ''$\Longleftarrow$'':\\
    фундаментальность $\xRightarrow[]{\text{св-во 2}}$
    ограниченность $\xRightarrow[]
    {\text{Больцано–Вейерштрасса}}$\\
    $\begin{rcases*}
        \Rightarrow \text{сущ. сходящаяся подпосл.}\\
        \quad\quad\text{фундаментальность}
    \end{rcases*}$
    $\xRightarrow[]{\text{св-во 3}}$ существует конечный предел.
    
    \end{proof}
    
    \section{Теорема Больцано–Вейерштрасса в $\mathbb{R}^d$.
    Полнота $\mathbb{R}^d$ }
    
    \begin{conj}
    Полнота метрического простраства
    \end{conj}
    Пусть $(X, \rho)$ -- метрическое пространство.
    $X$ - полное, если любая фундаментальная последовательность
    в нём имеет предел.
    
    \begin{theorem-non}
        $\mathbb{R}^d$ - полное пространство.
    \end{theorem-non}
    \begin{proof} $ $
    
        Возьмём фундаментальную последовательность $(x_n)$.
        $x_n = (x_n^{(1)}, x_n^{(2)}, \dots, x_n^{(d)})$
    
        \[\forall \varepsilon > 0 \,\, \exists N :
        \forall n, m \geq N \,\, \rho(x_n, x_m) <
        \varepsilon \Rightarrow\]
        \[\Rightarrow \abs{x_n^{(k)} -
        x_m^{(k)}} \leq \sqrt{(x_n^{(1)} -
        x_m^{(1)})^2 + (x_n^{(2)} - x_m^{(2)})^2 +
        \dots + (x_n^{(d)} - x_m^{(d)})^2} < \varepsilon
        \Rightarrow\] \[ \Rightarrow
        \text{числовая послед. } x_n^{(k)}
        \text{ фундаментальна } \Rightarrow \text
        {у неё есть конечный предел}\] \[\lim x_n^{(k)}
        = a_k \Rightarrow \lim x_n = a, \quad a = 
        (a_1, a_2, \dots, a_d) \]
        Т.к. в $\mathbb{R}^d$ покоординатная и
        сходимость по метрике -- одно и то же.
    
    \end{proof}
    
    \begin{theorem-non}
    Больцано–Вейерштрасса в $\mathbb{R}^d$.
    \end{theorem-non}
    \begin{proof}
    Пусть векторная последовательность
    $x_n = (x_n^{(1)}, x_n^{(2)}, \dots, x_n^{(d)})$ ограничена.
    Это равносильно тому, что все её координатные последовательности
    ограничены.
    
    Выделим из первой координатной последовательности сходящуюся
    подпоследовательность $(x_{n_{1, k}}^{(1)})$. Тогда получим
    подпоследовательность $(x_{n_{1, k}})$, первая координатная
    последовательность которой сходится, а остальные ограничены.
    
    Тогда в ней можно выделить такую подпоследовательность
    $(x_{n_{2, k}})$ так, чтобы вторая координатная последовательность
    сходилась.
    
    Повторим так ещё $d - 2$ раз и получим то, что в векторной 
    подпоследовательности $(x_{n_{k}})$, где $n_k = n_{d, k}$, 
    любая координатная последовательность сходится $\Rightarrow$
    $(x_{n_{k}})$ тоже сходится, т.к. в $\mathbb{R}^d$ покоординатная и
    сходимость по метрике -- одно и то же.
    
    \end{proof}
    
    
    \section{Верхний и нижний пределы. Связь между частичными пределами и  
    верхним и нижним пределами.}
    
    \begin{conj}
    Нижний и верхний пределы
    \end{conj}
    $x_n$ - числовая последовательность.
    
    $\underline{\lim} x_n := \liminf x_n := \lim \inf_{k \geq n} x_k$ -- 
    нижний предел.
    
    $\overline{\lim} x_n := \limsup x_n := \lim \sup_{k \geq n} x_k$ -- 
    верхний предел.
    
    $y_n := \inf_{k \geq n} x_k = \inf\{x_n, x_{n+1}, x_{n+2}, \dots\}$
    $\quad y_n \leq y_{n+1}$\\
    $z_n := \sup_{k \geq n} x_k = \sup\{x_n, x_{n+1}, x_{n+2}, \dots\}$
    $\quad z_n \geq z_{n+1}$
    
    \begin{theorem-non}
        $\underline{\lim}$ и $\overline{\lim}$ существуют в 
        $\overline{\mathbb{R}}$ и $\underline{\lim} \leq \overline{\lim}$
    \end{theorem-non}
    \begin{proof} $ $
    
    Про $\underline{\lim}$: $y_n \leq y_{n+1} \Rightarrow (y_n)$ --
    возрастающая последовательность $\Rightarrow$ у неё есть предел в
    $\overline{\mathbb{R}}$.
    
    Про $\overline{\lim}$: $z_n \geq z_{n+1} \Rightarrow (z_n)$ --
    убывающая последовательность $\Rightarrow$ у неё есть предел в
    $\overline{\mathbb{R}}$.
    
    Про неравенство $\underline{\lim} \leq \overline{\lim}$:
    $y_n \leq z_n$, $y_n \rightarrow \underline{\lim}$,
    $z_n \rightarrow \overline{\lim} \Rightarrow$ по предельному
    переходу в неравенстве $\underline{\lim} \leq \overline{\lim}$.
    \end{proof}
    
    \begin{theorem-non}\end{theorem-non}
    \begin{enumerate}
        \item $\overline{\lim}$ -- наибольший частичный предел
        \item $\underline{\lim}$ -- наименьший частичный предел
        \item $\exists \lim \in \overline{\mathbb{R}} \Leftrightarrow
        \overline{\lim} = \underline{\lim}$ и в этом случае 
        $\lim = \overline{\lim} = \underline{\lim}$ 
    \end{enumerate}
    \begin{proof} $ $
    
        \begin{enumerate}
            \item $a := \overline{\lim} \, x_n$
            
            Рассмотрим \textbf{случай} $a \in \mathbb{R}$
    
            Докажем, что $a$ -- частичный предел.
    
            $a = \lim z_n$, $z_n = \sup_{k \geq n} x_k$, $z_n \searrow a$
    
            Будем строить некоторую подпоследовательность $(x_{n_k})$.\\
            Найдётся $n_k \geq n_{k-1} : x_{n_k} > a - \frac{1}{k}$.
            Пусть не нашлось $\Rightarrow x_n \leq a - \frac{1}{k} \forall
            n \geq n_{k-1} \Rightarrow \sup \{x_{n_{k-1}}, x_{n_{k-1} + 1},
            \dots \} \leq a - \frac{1}{k} \Rightarrow a \leq z_{n_{k - 1}} 
            \leq a - \frac{1}{k}$. Противоречие
    
            $a - \frac{1}{k} \rightarrow a$, $z_{n_k} \rightarrow a$,
            $a - \frac{1}{k} < x_{n_k} \leq z_{n_k} \xRightarrow[]
            {\text{2 мил.}} x_{n_k} \rightarrow a$
    
            Докажем, что $a$ -- наибольший частичный предел.
    
            Пусть $b$ - частичный предел $\Rightarrow b = \lim x_{n_k}$.
            Но $x_{n_k} \rightarrow b$, $z_{n_k} \rightarrow a \Rightarrow$
            по предельному переходу $b \leq a$. 
    
            Рассмотрим \textbf{случай} $a = -\infty$.
    
            Тогда $z_n \rightarrow -\infty$, но $z_n = 
            \sup\{x_n, x_{n+1},\dots\} \geq x_n \Rightarrow x_n
            \rightarrow -\infty$.
    
            Рассмотрим \textbf{случай} $a = +\infty$.
    
            Тогда $z_n = +\infty \Rightarrow \sup {x_1, x_2, \dots} =
            +\infty \Rightarrow x_n$ не ограничена сверху $\Rightarrow$
            в ней найдётся подпоследовательность, стремящаяся к $+\infty$.
    
            \item Доказывается аналогично
            
            \item 
            ''$\Longrightarrow$'':
            
            Если $a = \lim x_n$, то все подпоследовательности стремятся
            к $a \Rightarrow$ все частичные пределы равны $a \Rightarrow
            \overline{\lim} x_n = \underline{\lim} x_n = \lim x_n = a$.
    
            ''$\Longleftarrow$'':
    
            $y_n \rightarrow a$, $z_n \rightarrow a$, $y_n \leq x_n \leq z_n$
            $\xRightarrow[]{\text{2 мил.}} x_n \rightarrow a \Rightarrow
            \lim x_n = \overline{\lim}\, x_n = \underline{\lim}\, x_n = a$
        \end{enumerate}
    \end{proof}
    
    \textbf{Замечание.} Арифметики для верхних и нижних пределов нет.
    
    Пример.
    \[x_n = (-1)^n, \quad y_n = (-1)^{n + 1} \Rightarrow
    \underline{\lim}\, x_n = \underline{\lim}\, y_n = -1\]
    \[x_n + y_n = 0 \Rightarrow \underline{\lim}\, (x_n + y_n) = 
    \underline{\lim}\, (x_n + y_n) = 0\]
    \[\underline{\lim}\, x_n + \underline{\lim}\, y_n = -2 < 0 =
    \underline{\lim}\, (x_n + y_n)\]
    
    \section{Характеристика верхних и нижних пределов с помощью 
    $N$ и $\varepsilon$. Сохранение неравенств.}
    
    \begin{theorem-non}\end{theorem-non}
    \begin{enumerate}
        \item $a = \underline{\lim}\, x_n \in \mathbb{R} \Leftrightarrow
        \begin{cases}
            \forall \varepsilon > 0 \,\, \exists N : \forall n \geq N
            \,\, x_n > a - \varepsilon\\
            \forall \varepsilon > 0 \,\, \forall N \,\, \exists n \geq N
            : x_n < a + \varepsilon\\
        \end{cases}$
        \item $b = \overline{\lim}\, x_n \in \mathbb{R} \Leftrightarrow
        \begin{cases}
            \forall \varepsilon > 0 \,\, \exists N : \forall n \geq N
            \,\, x_n < b + \varepsilon \quad \circled{1}\\
            \forall \varepsilon > 0 \,\, \forall N \,\, \exists n \geq N
            : x_n > b - \varepsilon \quad \circled{2}\\
        \end{cases}$
    \end{enumerate}
    \begin{proof} $ $
    
        2. Докажем \circled{1} $\Leftrightarrow \forall \varepsilon > 0\,\, 
        \exists N : z_N < b + \varepsilon$
    
        ''$\Longrightarrow$'':
        \[\forall \varepsilon > 0 \,\, \exists N : \forall n \geq N
        \,\, x_n < b + \varepsilon \Rightarrow
        \forall \varepsilon > 0 \,\, \exists N : \forall n \geq N
        \,\, x_n < b + \frac{\varepsilon}{2} \Rightarrow\] \[\Rightarrow
        z_N = \sup \{x_N, x_{N + 1}, \dots\} \leq b + \frac{\varepsilon}{2}
        < b + \varepsilon \Rightarrow \forall \varepsilon > 0\,\, 
        \exists N : z_N < b + \varepsilon\]
    
        ''$\Longleftarrow$'':
        \[\text{Зафиксируем } \varepsilon > 0
        \Rightarrow \exists N : z_N < b + \varepsilon \Leftrightarrow
        \sup\{x_N, x_{N + 1}, \dots\} < b + \varepsilon \Rightarrow
        x_n < b + \varepsilon \,\, \forall n \geq N\]
    
        Докажем \circled{2} $\Leftrightarrow \forall \varepsilon > 0 \,\,
        \forall N \,\, z_N > b - \varepsilon$
    
        ''$\Longrightarrow$'':
        \[\forall \varepsilon > 0 \,\, \forall N \,\, \exists n \geq N
        : x_n > b - \varepsilon \text{ при этом } z_N = \sup\{
        x_N, x_{N+1}, x_{N+2}, \dots\} \Rightarrow \forall
        \varepsilon > 0 \,\, \forall N \,\, z_N > b - \varepsilon\]
    
        ''$\Longleftarrow$'':
        \[\text{Зафиксируем } \varepsilon > 0 \text{ и } N \Rightarrow
        z_N > b - \varepsilon \Leftrightarrow \sup\{ x_N, x_{N+1}, \dots\}
        > b - \varepsilon \Rightarrow \exists n \geq N : x_n > b -
        \varepsilon,\] \[\text{иначе } \forall n \geq N : x_n \leq b -
        \varepsilon \text{ и тогда } \sup\{ x_N, x_{N+1}, \dots\} \leq
        b - \varepsilon \Leftrightarrow z_N \leq b - \varepsilon\]
    
        \circled{1} + \circled{2} $\Leftrightarrow
        \begin{cases}
            \forall \varepsilon > 0\,\, 
            \exists N : z_N < b + \varepsilon\\
            \forall \varepsilon > 0 \,\,
            \forall N \,\, z_N > b - \varepsilon\\
        \end{cases}$
        $\Leftrightarrow$ т.к. $z_n \searrow$
        $\begin{cases}
            \forall \varepsilon > 0\,\, 
            \exists N : \forall n \geq N \,\, z_n < b + \varepsilon\\
            \forall \varepsilon > 0 \,\,
            \forall N \,\, z_N > b - \varepsilon\\
        \end{cases}$
    
        Это и есть определение предела $\Rightarrow b = 
        \overline{\lim}\, x_n$
    
        В обратную сторону, первая строка следует из определения предела,
        вторая строка следует из того, что $(z_n) \searrow$. Более того,
        $(z_n) \searrow$, $\lim z_n = b \Rightarrow z_n \geq b$
    
    \end{proof}
    
    \begin{theorem-non}\end{theorem-non}
    Если $x_n \leq y_n$, то 
    $\underline{\lim}\, x_n \leq \underline{\lim}\, y_n$ и  
    $\overline{\lim}\, x_n \leq \overline{\lim}\, y_n$
    
    \begin{proof} $ $
    
    $x_n \leq y_n \Rightarrow \inf\{x_n, x_{n + 1}, ...\} \leq
    \inf\{y_n, y_{n + 1}, ...\} \Rightarrow$ по пред. переходу 
    $\underline{\lim}\, x_n \leq \underline{\lim}\, y_n$
    
    Аналогично для $\overline{\lim}\, x_n \leq \overline{\lim}\, y_n$.
    \end{proof}
    
    \section{Сходимость рядов. Необходимое условие сходимости рядов. Примеры.}
    
    \begin{conj}Ряд\end{conj}
    $x_n \in \mathbb{R}, \quad \sum_{n=1}^{+\infty} x_n$ -- ряд.
    
    \begin{conj}Частичная сумма ряда\end{conj}
    $S_n := \sum_{k=1}^{n} x_k$
    
    \begin{conj}Сумма ряда\end{conj}
    Cумма ряда -- $\lim S_n$, если он существует.
    
    \begin{conj}Сходимость ряда\end{conj}
    Ряд сходится, если $\exists \lim S_n \in \mathbb{R}$\\
    В противном случае ряд расходится.
    
    \begin{theorem-non}
    Необходимое условие сходимости
    \end{theorem-non}
    Если $\sum_{n = 1}^{+\infty} x_n$ сходится, то $\lim x_n = 0$.
    
    \begin{proof}
        Если ряд сходится, то $S := \lim S_n \in \mathbb{R}$. Тогда
        $x_n = S_n - S_{n - 1} \Rightarrow \lim x_n = \lim S_n - 
        \lim S_{n - 1} = S - S = 0$
    \end{proof}
    
    \textbf{Примеры:}
    \begin{enumerate}
        \item Геометрическая прогрессия $1 + q + q^2 + \dots$
        $\sum_{n = 0}^{+\infty} q^n$
    
        При $\abs{q} < 1$ $S_n = 1 + q + q^2 + \dots + q^{n - 1} =
        \frac{1 - q^n}{1 - q} \rightarrow \frac{1}{1 - q}$
    
        При $\abs{q} > 1$ ряд расходящийся, т.к. не выполнено
        необходимое условие.
    
        \item $1 - 1 + 1 - 1 + 1 - 1 + \dots$
        
        $S_{2n} = 0, \,\, S_{2n + 1} = 1 \Rightarrow$ предела нет.
    
        \item Гармонический ряд $1 + \frac{1}{2} + \frac{1}{3} + \dots$
        $\sum_{n = 1}^{+\infty} \frac{1}{n}$
    
        $H_n := \sum_{k = 1}{n} \frac{1}{k}$ -- гармонические числа.
        $H_n$ монотонно возрастает.
        \[H_{2^n} = 1 + \frac{1}{2} + \underbrace{\left(\frac{1}{3} + 
        \frac{1}{4}\right)}_{> 2 \cdot \frac{1}{4} = \frac{1}{2}} +
        \underbrace{\left(\frac{1}{5} + \frac{1}{6} + \frac{1}{7} + 
        \frac{1}{8}\right)}_{> 4 \cdot \frac{1}{8} = \frac{1}{2}} +
        \dots + \underbrace{\left(\frac{1}{2^{n - 1} + 1} + 
        \frac{1}{2^{n-1} + 2} + \dots + \frac{1}{2^n} \right)}_
        {> 2^{n-1} \cdot \frac{1}{2^n} = \frac{1}{2}} >\]
        \[> 1 + \underbrace{\frac{1}{2} + \frac{1}{2} + ... + \frac{1}{2}}_
        {n \text{ шт.}} = 1 + \frac{n}{2} \Rightarrow \text{ частичные
        суммы сколь угодно большие } \Rightarrow \lim H_n = +\infty\]
    
        Гармонический ряд -- расходящийся ряд, члены которого стремятся к $0$.
    
        \item \[\frac{1}{1 \cdot 2} + \frac{1}{2 \cdot 3} + \frac{1}{3 \cdot 4}
        + \dots \quad\quad\sum_{n=1}^{+\infty} \frac{1}{n\cdot(n+1)}\]
        \[\frac{1}{k\cdot(k+1)} = \frac{1}{k} - \frac{1}{k + 1} \Rightarrow
        S_n = \frac{1}{1 \cdot 2} + \frac{1}{2 \cdot 3} + \frac{1}{3 \cdot 4}
        + \dots + \frac{1}{n\cdot(n+1)} = 1 - \frac{1}{n+1} \rightarrow 1\]
    
    
    \end{enumerate}
\section{Простейшие свойства сходящихся рядов.}

\begin{enumerate}
    \item Сумма ряда единственна
    
    \begin{proof}
        Утверждение про единственность предела частичных сумм
    \end{proof}

    \item Расстановка скобок не меняет суммы ряда (если она была)
    
    \begin{proof}
        $\underbracket[0pt][2pt]{x_1}_{S_1} + (x_2 + x_3 +
        \underbracket[0pt][2pt]{x_4}_{S_4}) + (x_5 + 
        \underbracket[0pt][2pt]{x_6}_{S_6}) + (x_7 + x_8 + 
        \underbracket[0pt][2pt]{x_9}_{S_9})...$

        Т.е. из последовательности частичных сумм просто выбрали
        другую подпоследовательность, ну таким образом, если предел был,
        то он такой же и остался.
    \end{proof}

    \textbf{Замечание.} Он расстановки скобок сумма ряда могла
    появиться.

    Пример. Ряд $1 - 1 + 1 - 1 + 1 - 1 + 1 - 1 + \dots$ расходится.
    Но при расстановке следующим образом скобок:
    $(1 - 1) + (1 - 1) + (1 - 1) + (1 - 1) + \dots$ получаем, что ряд
    имеет сумму $0$.

    \item Добавление/отбрасывание конечного числа членов не влияет на
    сходимость, но влияет на сумму.

    \begin{proof}
        Рассмотрим отбрасывание.

        Ряд $x_1 + x_2 + x_3 + \dots$, частичная сумма которого
        $S_n$, переделали в $x_{k+1} + x_{k+2} + x_{k+3} + \dots$,
        частичная сумма которого $\widetilde{S}_n := x_{k+1} + x_{k+2}
        + \dots + x_{k+n} = S_{k + n} - S_{k}$. Т.к. $k$ фиксировано
        отсюда видно, что если $S_n$ (не) имеет предел, то и
        $\widetilde{S}_n$ (не) имеет предел, и наоборот.

        Добавление - просто обратная операция.
    \end{proof}

    \item Если $\sum_{n = 1}^{+\infty} a_n$ и $\sum_{n = 1}^{+\infty} b_n$
    сходятся, то $\sum_{n = 1}^{+\infty} (a_n \pm b_n)$ сходится и
    $\sum_{n = 1}^{+\infty} (a_n \pm b_n) =\vspace*{0,2cm} \\ = \sum_{n = 1}^{+\infty} a_n
    \pm \sum_{n = 1}^{+\infty} b_n$

    \item Если $\sum_{n = 1}^{+\infty} a_n$ сходится, то
    $\sum_{n = 1}^{+\infty} c a_n$ сходится и $\sum_{n = 1}^{+\infty} 
    c a_n = c \cdot \sum_{n = 1}^{+\infty} a_n$ 
\end{enumerate}
\ifdefined\niveldos\else
\end{document} 
\fi