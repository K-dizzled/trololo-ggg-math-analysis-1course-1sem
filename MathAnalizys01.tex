\ifdefined\niveldos\else
\documentclass[12pt,letterpaper]{report}
 
%Russian-specific packages
%--------------------------------------
\usepackage[T2A]{fontenc}
\usepackage[utf8]{inputenc}
\usepackage[russian]{babel}
\usepackage[mathscr]{euscript}
\usepackage{mathrsfs}
\usepackage{amsmath,amsthm,amssymb,latexsym,amsfonts}
\usepackage{showkeys}
\usepackage{pythonhighlight}
\usepackage{mdframed}
\usepackage{lipsum}
\usepackage{soul}
\usepackage{amsmath,amssymb}
\usepackage{parskip}
\usepackage{graphicx}
\usepackage{mathtools}
\usepackage[usestackEOL]{stackengine} 
\usepackage{tocloft}
\usepackage{xcolor}
\usepackage{hyperref}
\usepackage{tikz}
\usepackage{thmtools}
\usepackage{amsthm}
\usepackage{mathtools}

\DeclarePairedDelimiter\abs{\lvert}{\rvert}%
\DeclarePairedDelimiter\norm{\lVert}{\rVert}%

% Swap the definition of \abs* and \norm*, so that \abs
% and \norm resizes the size of the brackets, and the 
% starred version does not.
\makeatletter
\let\oldabs\abs
\def\abs{\@ifstar{\oldabs}{\oldabs*}}
%
\let\oldnorm\norm
\def\norm{\@ifstar{\oldnorm}{\oldnorm*}}
\makeatother

\newtheorem{theorem}{Теорема}
\newtheorem*{theorem-non}{Теорема}
\newtheorem{lemma}[theorem]{Лемма}
\newtheorem{conj}[theorem]{Определение}
\declaretheorem[numbered=no]{definition}

% Definindo novas cores
\definecolor{verde}{rgb}{0.25,0.5,0.35}
\definecolor{jpurple}{rgb}{0.5,0,0.35}
\definecolor{darkgreen}{rgb}{0.0, 0.2, 0.13}
%\definecolor{oldmauve}{rgb}{0.4, 0.19, 0.28}
% Configurando layout para mostrar codigos Java
\usepackage{listings}

\newcommand{\estiloJava}{
\lstset{
    language=Java,
    basicstyle=\ttfamily\small,
    keywordstyle=\color{jpurple}\bfseries,
    stringstyle=\color{red},
    commentstyle=\color{verde},
    morecomment=[s][\color{blue}]{/**}{*/},
    extendedchars=true,
    showspaces=false,
    showstringspaces=false,
    numbers=left,
    numberstyle=\tiny,
    breaklines=true,
    backgroundcolor=\color{cyan!10},
    breakautoindent=true,
    captionpos=b,
    xleftmargin=0pt,
    tabsize=2
}}

\newcommand{\estiloR}{
  \lstset{ %
    language=R,                     % the language of the code
    basicstyle=\footnotesize,       % the size of the fonts that are used for the code
    numbers=left,                   % where to put the line-numbers
    numberstyle=\tiny\color{gray},  % the style that is used for the line-numbers
    stepnumber=1,                   % the step between two line-numbers. If it's 1, each line
                                    % will be numbered
    numbersep=5pt,                  % how far the line-numbers are from the code
    backgroundcolor=\color{white},  % choose the background color. You must add \usepackage{color}
    showspaces=false,               % show spaces adding particular underscores
    showstringspaces=false,         % underline spaces within strings
    showtabs=false,                 % show tabs within strings adding particular underscores
    frame=single,                   % adds a frame around the code
    rulecolor=\color{black},        % if not set, the frame-color may be changed on line-breaks within not-black text (e.g. commens (green here))
    tabsize=2,                      % sets default tabsize to 2 spaces
    captionpos=b,                   % sets the caption-position to bottom
    breaklines=true,                % sets automatic line breaking
    breakatwhitespace=false,        % sets if automatic breaks should only happen at whitespace
    title=\lstname,                 % show the filename of files included with \lstinputlisting;
                                    % also try caption instead of title
    keywordstyle=\color{blue},      % keyword style
    commentstyle=\color{darkgreen},   % comment style
    stringstyle=\color{red},      % string literal style
    escapeinside={\%*}{*)},         % if you want to add a comment within your code
    morekeywords={*,...}          % if you want to add more keywords to the set
}}

\newcommand{\Z}{\mathbb{Z}}
\newcommand{\Q}{\mathbb{Q}}
\newcommand{\N}{\mathbb{N}}
\newcommand{\R}{\mathbb{R}}
\newcommand{\follow}{\textbf{\textit{Следствие:}}}
\newcommand{\notice}{\underline{\textit{Замечание }}}

\definecolor{linkcolor}{HTML}{47528f} % цвет ссылок
\definecolor{urlcolor}{HTML}{47528f} % цвет гиперссылок
\hypersetup{pdfstartview=FitH,  linkcolor=linkcolor,urlcolor=urlcolor, colorlinks=true}
\newcommand{\RomanNumeralCaps}[1]
  {\MakeUppercase{\romannumeral #1}}
\usepackage{titlesec}
\renewcommand{\thesection}{\arabic{section}}
\renewcommand{\listtheoremname}{Определения и теоремы}
\renewcommand\qedsymbol{$\blacksquare$}
\newcommand\oast{\stackMath\mathbin{\stackinset{c}{0ex}{c}{0ex}{\ast}{\bigcirc}}}
\makeatletter
\renewenvironment{proof}[1][\proofname]{%
   \par\pushQED{\qed}\normalfont%
   \topsep6\p@\@plus6\p@\relax
   \trivlist\item[\hskip\labelsep\bfseries#1\@addpunct{.}]%
   \ignorespaces
}{%
   \popQED\endtrivlist\@endpefalse
}
\makeatother
%--------------------------------------
\DeclareMathOperator{\Mr}{M_{\mathbb{R}}}
\addtolength{\oddsidemargin}{-.875in}
	\addtolength{\evensidemargin}{-.875in}
	\addtolength{\textwidth}{1.75in}

	\addtolength{\topmargin}{-.875in}
    \addtolength{\textheight}{1.75in}
%--------------------------------------
\def\calloutsym{%
  \ensurestackMath{%
  \scalebox{1.7}{\color{red}\stackunder[0pt]{\bigcirc}{\downarrow}}}%
}
\def\calloutsymup{%
  \ensurestackMath{%
  \scalebox{1.7}{\color{red}\stackon[0pt]{\bigcirc}{\uparrow}}}%
}
\newcommand\callouttext[1]{%
  \def\stacktype{S}\renewcommand\useanchorwidth{T}\stackText%
  \stackunder{\calloutsym}{\scriptsize\Longstack{#1}}\stackMath%
}
\newcommand\callout[3][2.5pt]{%
  \def\stacktype{L}\stackMath\stackunder[#1]{#2}{\callouttext{#3}}%
}
\newcommand\callouttextup[1]{%
  \def\stacktype{S}\renewcommand\useanchorwidth{T}\stackText%
  \stackon{\calloutsymup}{\scriptsize\Longstack{#1}}\stackMath%
}
\newcommand\calloutup[3][1.5pt]{%
  \def\stacktype{L}\stackMath\stackunder[#1]{#2}{\callouttextup{#3}}%
}
%%%%%%%%%%%%%%%%%%%%%%%%%%%%%%%%%%
%                                %
%                                %
%              Title             %
%                                %
%                                %
%%%%%%%%%%%%%%%%%%%%%%%%%%%%%%%%%%
\title{Конспект лекций по математическому анализу 
\thanks{Атворы: \href{https://github.com/maxmartynov08}{maxmartynov08}, \href{https://github.com/K-dizzled}{K-dizzled}, \href{https://github.com/SmnTin}{SmnTin}, \href{https://github.com/muldrik}{muldrik}}}
\author{Храбров Александр Игоревич}
\date{Первый курс, первый семестр 2020}
\begin{document}
\fi
\maketitle
%%%%%%%%%%%%%%%%%%%%%%%%%%%%%%%%%%
%                                %
%                                %
%        Table of content        %
%                                %
%                                %
%%%%%%%%%%%%%%%%%%%%%%%%%%%%%%%%%%
\tableofcontents
%\listoftheorem-nons[ignore={lemma},show={conj, theorem-non}]
% To show Table of contents with
% definitions and lemmas
%----------------------------------------
% \listoftheorem-nons[ignoreall,show={lemma}]
% \listoftheorem-nons[ignoreall,show={conj}]
\newpage
\chapter{Введение}
\section{Множества}
\begin{conj} Множество - набор уникальных элементов \end{conj}

Множества - большие буквы $A, B,\dots$ \\
Элементы множеств - маленькие буквы $a, b,\dots$ \\
$x \in A - x$ пренадлежит $A$ \\
$x \notin A - x$ не пренадлежит $A$ \\
$\mathbb{N} = \{1, 2, 3, \dots\} \\
\mathbb{Z, Q} = \{{{m}\over{n}} : m \in \mathbb{Z}, n \in\mathbb{N}\} \\
\mathbb{R}$ - вещественные числа \\
$\mathbb{C}$ - комплексные числа \\
\begin{theorem-non} Правила Де Моргана \end{theorem-non}
    \begin{itemize}
        \item[] $A \; \setminus \; (\bigcup\limits_{\alpha \in I} B_{\alpha}) 
        = \bigcap\limits_{\alpha \in I}(A \setminus B_{\alpha})$

        \item[] $A \; \setminus \; (\bigcap\limits_{\alpha \in I} B_{\alpha}) 
        = \bigcup\limits_{\alpha \in I}(A \setminus B_{\alpha})$
    \end{itemize}
\begin{proof}
    Докажем для первой формулы. Вторая доказывается аналогично. \\
    $x \in A \; \setminus \; (\bigcup\limits_{\alpha \in I} B_{\alpha}) 
    \Longleftrightarrow \begin{cases}
        x \in A \\
        x \notin \bigcup\limits_{\alpha \in I} B_{\alpha}
    \end{cases}
    \Longleftrightarrow \begin{cases}
        x \in A \\
        x \notin B_{\alpha} \; \; $при всех$ \; \alpha
    \end{cases} 
    \Longleftrightarrow x \in A \; \setminus \; B_{\alpha}$ при всех $\alpha \in I
    \Longleftrightarrow x \in \bigcap\limits_{\alpha \in I}(A \setminus B_{\alpha})$ 
\end{proof} \newpage
\begin{theorem-non} Операции над множествами \end{theorem-non}
\begin{itemize}
    \item $A \cup B = \{x: x \in A $ или $ x \in B\}$
    \item $A \cap B = \{x: x \in A, x  \in B\}$
    \item $A \; \setminus \; B = \{x: x \in A, x  \notin B\}$
    \item $A \bigtriangleup B = (A \; \setminus \; B) \cup (B \; \setminus \; A)$
\end{itemize}
\notice - $\bigtriangleup, \cup, \cap$ - комммутативны, ассоциативны
\begin{conj} 
    Декартово произведение множеств 
    $A \times B = \{\langle a, b \rangle : a \in A; b \in B \}$ 
\end{conj}
\begin{theorem-non} \end{theorem-non}
    \begin{itemize}
        \item[] $A \cap \bigcup\limits_{\alpha \in I} B_{\alpha} =
        \bigcup\limits_{\alpha \in I}(A \cap B_{\alpha}) $

        \item[] $A \cup \bigcap\limits_{\alpha \in I} B_{\alpha} =
        \bigcap\limits_{\alpha \in I}(A \cup B_{\alpha}) $
    \end{itemize}
\begin{proof}
        $x \in A \cap \bigcup\limits_{\alpha \in I} B_{\alpha}
        \Longleftrightarrow \begin{cases}
            x \in A \\
            x \in \bigcup\limits_{\alpha \in I} B_{\alpha}
        \end{cases} \Longleftrightarrow \begin{cases}
            x \in A \\
            x \in B_{\alpha}$ для некоторых $\alpha \in I
        \end{cases}\vspace{0.5cm} \Longleftrightarrow 
        x \in A \cap B_{\alpha}$ для некоторых $\alpha \in I 
        \Longleftrightarrow 
        x \in \bigcup\limits_{\alpha \in I}(A \cap B_{\alpha})$
    \end{proof}
\begin{conj} 
    Упорядоченная пара $ \langle a, b \rangle $ - пара  ``пронумерованных'' элементов
\end{conj}
    $  \langle a, b \rangle $ = $  \langle c, d \rangle \rotatebox[origin=c]{150}{$\Longleftrightarrow$}$ 
\begin{scriptsize}
\estiloJava
\begin{lstlisting}[caption={}, label=]
    ((a == c) && (b == d))
\end{lstlisting}
\end{scriptsize}
\section{Отношения}
\begin{conj} 
    Область определения: 
    $\delta_{R} = \{x \in A: \exists y \in B, $ т.ч.$ \langle x, y \rangle  \in \mathbb{Z} \} $ 
\end{conj}

\begin{conj} 
    Область значений: 
    $\rho_{R} = \{y \in B: \exists x \in A, $ т.ч.$ \langle x, y \rangle  \in \mathbb{Z} \} $ 
\end{conj}
$\delta_{R^{-1}} = \rho_{R} \\
\rho_{R^{-1}} = \delta_{R}$

\begin{conj} 
    Композиция отношений 
\end{conj}

\begin{itemize}
    \item[] $R_1 \subset A \times B, \quad R_2 \subset B \times C, \quad R_1 \circ R_2 \subset A \times C$
\end{itemize}
\subsection*{Пример}
\begin{itemize}
    \item $\langle x, y \rangle \in R$, если x — отец y
    \item $\langle x, y \rangle \in R \circ R$, если x — дед y
    \item $\langle x, y \rangle \in R^{-1} \circ R$, если x — брат y
    \item $\delta R$ — все, у кого есть сыновья
\end{itemize}
\begin{conj} 
    Бинарным отношением $R$ называется подмножество элементов декартова произведения двух
    множеств $R \subset A \times B$
\end{conj}

\begin{itemize}
    \item[] Элементы $x \in A, y \in B$ находятся в отношении, если $  \langle x, y \rangle \in R $ (то же, что $xRy$)
    \item[] Обратное отношение $R^{-1} \subset B \times A$ 
\end{itemize}

\begin{conj}
    Отношение называется:
\end{conj}
\begin{itemize}
    \item Рефлексивным, если $xRx \; \forall x$
    \item Симметричным, если $xRy \Longrightarrow yRx$
    \item Транзитивным, если $xRy, yRz \Longrightarrow xRz$
    \item Иррефлексивным, если $\neg xRx \forall x$
    \item Антисимметричным, если $xRy, yRx \Longrightarrow x = y$
\end{itemize}

\begin{conj}
    $R$ является отношением
\end{conj}
\begin{itemize}
    \item[1.] Эквивалентности, если оно рефлексивно, симметрично и транзитивно
    \item[2.] Нестрогого частичного порядка, если оно рефлексивно, антисимметрично и транзитивно
    \item[3.] Нестрогого полного порядка, если выполняется п. $2 + \forall x, y$ либо $xRy$, либо $yRx$
    \item[4.] Строгого частичного порядка, если оно иррефлексивно и транзитивно
    \item[5.] Строгого полного порядка, если выполняется п. $4 + \forall x$, y либо $xRy$, либо $yRx$
\end{itemize}

\subsection*{Пример}
\begin{itemize}
    \item $x \equiv y \; (mod \; m)$ — отношение эквивалентности
    \item $X$ - множество$, 2^X$ — множество всех его подмножеств
    \item $\forall x, y \in 2^x : \langle x, y \rangle \in R, $ если $ x \subsetneq y$ — отношение строгого частичного порядка
    \item Лексикографический порядок на множестве пар натуральных чисел — отношение нестрогого полного порядка
\end{itemize}

\begin{conj}
    Отображение $f: A  \longrightarrow B$ 
\end{conj}
\begin{itemize}
    \item инъективно, если $f(x_1) = f(x_2) \Leftrightarrow x_1 = x_2$
    \item сюръективно, если $\rho_f = B$
    \item биективно, если $f$ инъективно и сюръективно
\end{itemize}
\section{Аксиомы вещественных чисел}
\begin{conj}
    Вещественные числа - алгебраическая структура, над которой определены 
    операции сложения ``+'' и умножения ``$\cdot$'' $(\mathbb{R} * \mathbb{R} \rightarrow \mathbb{R})$
\end{conj}
\newpage
\begin{conj}
    Аксиомы вещественных чисел:
\end{conj}
\begin{itemize}
    \item[$A_1$] Ассоциативность сложения \\
     $x + (y + z) = (x + y) + z$
    \item[$A_2$] Коммутативность сложения \\
     $x + y = y + x$
    \item[$A_3$] Существование нуля \\
     $\exists 0 \in \mathbb{R} : \forall x \in \mathbb{R} \; x + 0 = x$
    \item[$A_4$] Существование обратного элемента по сложению \\
     $\forall x \in \mathbb{R} \; \exists (-x) \in \mathbb{R} : x + (-x) = 0$
    \item[$M_1$] Ассоциативность умножения \\
     $x(y \cdot z) = (x \cdot y)z$
    \item[$M_2$] Коммутативность умножения \\
     $xy = yx$
    \item[$M_3$] Существование единицы \\
     $\exists 1 \in \mathbb{R} : \forall x \in \mathbb{R} \; x \cdot 1 = x$
    \item[$M_4$] Существование обратного элемента по умножению \\
     $\forall x \in \mathbb{R} \; \exists x^{-1} \in \mathbb{R} : x \cdot x^{-1} = 1$
    \item[$M_A$] Дистрибутивность \\
     $(x + y) \cdot z = x \cdot z + y \cdot z$ 
\end{itemize}
\nocite - Вышеперечисленные аксиомы бразуют поле \vspace{0.5cm} \\
\textbf{Бинарное отношение} ``$\leqslant$'' \\
Аксиомы порядка, задающие отношение порядка на множестве вещественных чисел:
\begin{itemize}
    \item[$O_1$] $x \leqslant x \quad \forall x$
    \item[$O_2$] $x \leqslant y $  и  $ y \leqslant x \Longrightarrow x = y$ 
    \item[$O_3$] $x \leqslant y $  и  $ y \leqslant z \Longrightarrow x \leqslant z$ 
    \item[$O_4$] $\forall x, y \in \mathbb{R} : x \leqslant y $ или $ y \leqslant x$
    \item[$O_4$] $x \leqslant y \Longrightarrow x + z \leqslant y + z \quad \forall z$ 
    \item[$O_4$] $0 \leqslant x $ и $ 0 \leqslant y \Longrightarrow 0 \leqslant xy$  
\end{itemize}
\begin{theorem-non}
    Аксиома полноты
\end{theorem-non}
$A, B \subset \mathbb{R} : A \neq \varnothing, B \neq \varnothing, \forall a \in A \; \forall b \in B \; a \leqslant b$ \\
Тогда $\exists c \in \mathbb{R} : a \leqslant c \leqslant b \; \forall a \in A \; \forall b \in B$
\begin{theorem-non}
    Принцип Архимеда
\end{theorem-non}
Согласно принципу Архимеда: $\forall x \in \mathbb{R}$ и $\forall y_{>0} \in \mathbb{R} \; \exists n \in \mathbb{N} : x < ny$
\begin{proof}  
    \quad \\ $A = \{a \in \mathbb{R} : \exists n \in \mathbb{N} : a < ny\}, A \neq \varnothing$ т.к. $0 \in A$ \\
    $B = \mathbb{R} \; \setminus \; A$ \\
    Пусть $A \neq \mathbb{R}$, тогда $B \neq \varnothing$ Покажем, что $a \leqslant b$, если $a \in A, b \in B$ \\
    Пойдем от противного. Если $b < a < ny \Longrightarrow b < ny \Longrightarrow b \in A$ - противоречие \\
    Таким образом, по аксиоме полноты $\exists c \in \mathbb{R} : a \leqslant c \leqslant b \quad \forall a \in A, \forall b \in B$ \\
    Предположим, что $c \in A$. Тогда $c < ny$ для некоторого $n \in \mathbb{N} \Longrightarrow c + y < (n + 1)y \Longrightarrow \\ 
    c + y \in A \Longrightarrow c + y \leqslant c \Longrightarrow y \leqslant 0$. Это противоречит условию. \\
    Пусть $c \in B$. Так как $y > 0, c - y < c$. Так как $B$ - дополненние $A$ и $c - y \neq c, \; c - y \in A
    \Longrightarrow c - y < ny \Longrightarrow c < (n + 1)y \Longrightarrow c \in A$. Снова пришли к противоречию. \\
    Значит $c \notin A, c \notin B \Longrightarrow c$ не существует $\Longrightarrow B = \varnothing \Longrightarrow A = \mathbb{R}$ 
\end{proof}
\textbf{\textit{Следствие:}}
    \begin{itemize}
        \item[] $\forall \varepsilon_{> 0} \; \exists n \in \mathbb{N}: {{1}\over{n}} < \varepsilon$
        \begin{proof}
           \quad \\ $x = 1, y = \varepsilon \Longrightarrow \exists n \in N: 1 < n\varepsilon$
        \end{proof} 
    \end{itemize}
\section{Принцип математической индукции}
\begin{conj}
    Принцип математической индукции
\end{conj}
$P_n$ -последовательность утверждений 
\begin{enumerate}
    \item $P_1$ - верно
    \item $\forall n \in \mathbb{N}$ из $P_n$ следует $P_{n+1}$
\end{enumerate}
Тогда $P_n$ верно при всех $n \in \mathbb{N}$
\begin{theorem-non}
    В конечном множестве вещественных чисел есть наибольший и наименбший элемент
\end{theorem-non}
\begin{proof}
    \quad \\
    Докажем для максимума. Для минимума рассуждения аналогичны \\
    Будем доказывать утверждение по индукции \\
    Для $n = 1$ - очевидно \\
    Переход $X_n \longrightarrow x_{n+1}$ \\
    Рассмотрим произвольное множество из $n$ элементов $X_n = \{x_1, x_2, x_3, \dots x_n\}$, где максимальным элементом 
    является $x_i$. Пусть в наше множество был добавлен элемент $X_{n+1}$. В таком случае, если $X_{n+1}$ > $X_{i}$, то новый максимум равен
    $X_{n+1}$, иначе - максимумом по-прежнему является $X_{i}$. Таким образом, в любом конечном множестве вещественных чисел существует минимальный
    элемент.     
\end{proof}
\newpage
\textbf{\textit{Следствия:}}
\begin{enumerate}
    \item Во всяком непустом множестве натуральных чисел есть наименьший элемент  
    \begin{proof}
        \quad \\
        Пусть $A$ - множество натуральных чисел, не содержащее наименьшего элемента.
        Докажем по индукции, что для любого $n \in \mathbb{N}$ мы имеем $\mathbb{N}_n \cap A = \varnothing$ \\
        $\N_n = \{k \in \N | k \leqslant \N \}$ \\
        Для $n = 1$ утверждение очевидно. \\
        Переход $n \longrightarrow n+1$ \\
        Предположим для $\mathbb{N}_n \cap A = \varnothing$ \\
        Тогда если для $\mathbb{N}_{n+1} \cap A \neq \varnothing$, то наименьший элемент множества $A$ - это $n+1$ \\
        Значит $\mathbb{N}_{n+1} \cap A = \varnothing$
   \end{proof}
   \item Во всяком конечном непустом множестве натуральных чисел есть наибольший элемент
   \begin{proof}
        \quad \\
        Из натуральных чисел строим целые. Множество чисел $A \subseteq \Z$ называется огианиченным сверху и имеет наибольший элемент
        если $\exists c > a, \forall a \in A, c \in \Z$
    \end{proof}
\end{enumerate}
\subsection{Рациональные и иррациональные числа в интервале}
\begin{enumerate}
    \item Если $x, y \in \mathbb{R}, x < y$, то $\exists r \in \mathbb{Q}: x < r < y$ 
    \begin{proof}
        \quad \\ Пусть $x < 0, y > 0$. Тогда $\exists r = 0 \in \mathbb{Q}: x < r < y$ \\
        Пусть $x \geqslant 0, y > 0, \varepsilon = x - y$. Тогда $\exists n \in \mathbb{N}: {{1}\over{n}} < \varepsilon$ \\
        По принципу Архимеда найдется такое число $m$, что ${{m-1}\over{n}} \leqslant x < {{m}\over{n}}$ \vspace{0.2cm} \\
        Предположим, что ${{m-1}\over{n}} \leqslant x < y \leqslant {{m}\over{n}}$. Тогда мы получим, что ${{1}\over{n}} \geqslant y - x = \varepsilon$.
        Пришли к противоречию\\
        Следовательно, $\exists m \in \mathbb{N} : x < {{m}\over{n}} < y$ \\
        Случай $y \leqslant 0$ аналогичен предыдущему
    \end{proof} 
    \item Если $x, y \in \mathbb{R}, x < y$, то существует иррациональное число $r: x < r < y$ 
    \begin{proof}
        \quad \\ $x - \sqrt{2} < y - \sqrt{2} \Longrightarrow \exists R_{\in \Q} \in (x - \sqrt{2}, y - \sqrt{2}) \Longrightarrow
        x < R + \sqrt{2} < y \; $(Предыдущий пункт)$ \; \Longrightarrow \\ r$ - иррациональное
    \end{proof} 
    \item Если $x \geqslant 1$, то $\exists n \in \mathbb{N}: x - 1 < n \leqslant x$ 
\end{enumerate}
\section{Супремум и инфимум}
\begin{conj}
    \quad \\
    \begin{itemize}
        \item[] $x$ - верхняя граница множества $A$, если $\forall a \in A: a \leqslant x$
        \item[] $y$ - нижняя граница множества $A$, если $\forall a \in A: y \leqslant a$ 
        \item[] Множество ограничено снизу, если существует какая-нибудь нижняя граница
        \item[] Множество ограничено сверху, если существует какая-нибудь верхняя граница
    \end{itemize}
\end{conj}
\begin{conj}
    \quad \\
    Пусть $A$ - ограниченное сверху множество, тогда $sup A$ - наименьшая из его верхних границ
\end{conj}
\begin{conj}
    \quad \\
    Пусть $A$ - ограниченное снизу множество, тогда $inf A$ - наибольшая из его нижних границ
\end{conj}
\begin{theorem-non}
    \quad \\
    \begin{enumerate}
        \item Если $A \subset \R, A \neq \varnothing $ и $ A $ ограничено снизу, то существует единственный $inf A$
        \item Если $A \subset \R, A \neq \varnothing $ и $ A $ ограничено сверху, то существует единственный $sup A$
    \end{enumerate}
\end{theorem-non}
\begin{proof}
    \quad \\
    Докажем (2) \\
    Пусть $B$ - множество всех верхних границ множества $A$, т.е. $\forall a \in A, b \in B: a \leqslant b$ \\
    Тогда по аксиоме полноты всегда найдется такой $c: a \leqslant c \leqslant b$ \\
    $c - sup A$ по определению \\
    Докажем, что $c$ - единсвтенный \\
    Пусть $\exists c_1, c_2 - sup A$ \\
    Тогда если $c_1 < c_2$, то $c_2 \neq sup A$ \\
    Если $c_1 > c_2$,то $c_1 \neq sup A$ \\
    Следовательно, $c_1 = c_2 = sup A \Longrightarrow sup A$ - единсвтенный  
\end{proof}
\follow
\begin{enumerate}
    \item $B \subset A, B \neq \varnothing $ и $ A $ ограничено снизу. Тогда $inf B \geqslant inf A$
    \item $B \subset A, B \neq \varnothing $ и $ A $ ограничено сверху. Тогда $sup B \leqslant sup A$
\end{enumerate}
\begin{proof}
    \quad \\
    Докажем (1) \\
    Пусть $a = inf A$. Тогда $a$ - нижняя граница $A \Longrightarrow \forall x \in
    A : a \leqslant x \Longrightarrow \forall x \in B : a \leqslant x \Longrightarrow \\
    a$ - нижняя граница $B \Longrightarrow a \leqslant inf B$  
\end{proof}
\notice - Теорема неверна без аксиомы полноты \\
\begin{itemize}
    \item[] $A =\{x \in \Q : x^2 < 2\} \Longrightarrow$ в множестве рациональных чисел у $A$ нет супремума
\end{itemize}
\begin{theorem-non}
    \quad \\
    \begin{enumerate}
        \item $a = inf A \Longleftrightarrow 
        \begin{cases}
            a \leqslant x \quad \forall x \in A \\
            \forall \varepsilon > 0 \quad \exists x \in A : x < a + \varepsilon
        \end{cases}$ 
        \item $b = sup A \Longleftrightarrow 
        \begin{cases}
            b \geqslant x \quad \forall x \in A \\
            \forall \varepsilon > 0 \quad \exists x \in A : x > b - \varepsilon
        \end{cases}$ 
    \end{enumerate}
\end{theorem-non}
\notice
\begin{itemize}
    \item Если $A$ неограничено сверху, то $sup A = +\infty$
    \item Если $A$ неограничено снизу, то $inf A = -\infty$
\end{itemize}
\section{Теорема о вложенных отрезках}
\begin{theorem-non}
    \quad \\
    Если $[a_1, b_1] \supset [a_2, b_2] \supset [a_3, b_3] \supset \dots$ \\
    То $\exists c\in \R : c \in [a_n, b_n] \forall n \in \N$
\end{theorem-non}
\begin{tikzpicture}
    \draw (-.5,0)--(5.5,0);
    \draw[color=black] (0, 0) node {\bfseries[} node[below=9pt]{$a_{n}$};
    \draw[color=black] (5, 0) node {\bfseries]} node[below=9pt]{$b_{n}$};
    \draw[color=black] (1.5, 0) node {\bfseries[} node[below=9pt]{$a_{n+1}$};
    \draw[color=black] (4, 0) node {\bfseries]} node[below=9pt]{$b_{n+1}$};
\end{tikzpicture}
\begin{proof}
    \quad \\
    $A = \{a_1, a_2, a_3, \dots\} \\
    B = \{b_1, b_2, b_3, \dots\} \\
    a_i \leqslant b_j, \forall i,j \in \N \\
    \forall i \leqslant j : a_i \leqslant a_j \leqslant b_j \leqslant b_i, \forall i \geqslant j : a_i \geqslant a_j \geqslant b_j \geqslant b_i$ \\
    По аксиоме полноты $\forall i, j \in \N \; \exists c \in \R: a_i \leqslant c \leqslant b_j \Longrightarrow \forall i \in \N : a_i \leqslant c \leqslant b_i$ \\
\end{proof}
\notice
\begin{enumerate}
    \item Теорема неверна для полуинтервалов \\
    Пример: $\bigcap\limits_{n = 1}^{\infty}(0; {{1}\over{n}}] = \varnothing$
    \item Теорема неверна для лучей \\
    Пример: $\bigcap\limits_{n = 1}^{\infty}(n; +\infty) = \varnothing$
    \item Теорема неверна без аксиомы полноты \\
    Пример: число $\pi$ \\
    $[3;\; 4] \supset [3,1;\; 3,2] \supset [3,14;\; 3,15] \supset \dots$ \\
    Пересечение не содержит рациональных чисел
\end{enumerate}
\chapter{Последовательности вещественных чисел}
\section{Метрические пространства и подпространства}
    \begin{conj}
        $X$ - множество $\rho : X \times X \longrightarrow [0; + \infty)$ - метрика(расстояние)
        если: 
        \begin{enumerate}
            \item $\rho(x, x) = 0 \quad \forall x \in X$
            \item если $\rho(x, y) = 0$, то $x = y$
            \item $\rho(x, y) = \rho(y, x) \quad \forall x, y \in X$
            \item $\rho(x, y) + \rho(y, z) \geqslant \rho(x, z) \quad \forall x, y, z \in X$
        \end{enumerate}
    \end{conj}
\subsection*{Примеры}
\begin{enumerate}
    \item Дискретная метрика
        \begin{itemize}
            \item[] $\rho (x, x) = 0$
            \item[] $\rho (x, y) = 1$, если $x \neq y$
        \end{itemize}
    \item $\R \quad \rho (x, y) = \abs{x - y}$
    \item $\R^2 \quad$ обычное расстрояние
    \item Манхэттенская метрика 
    \begin{itemize}
        \item[] $(x', y') = A'$
        \item[] $(x, y) = A$
        \item[] $\rho (A, A') = \abs{x - x'} + \abs{y - y'}$  
    \end{itemize}
    \newpage
    \item Французская железнодорожная метрика \\
    \begin{center}
        \begin{tikzpicture}
            \node (p) {P} node (a) at (-2,1) {A} node (b) at (2,2) {B} node (c) at (1,1) {A}; 
            \draw (p) -- (a); \draw[red,thick] (p) -- (c); \draw[red,thick] (c) -- (b);
        \end{tikzpicture}
        Если $P, A$ и $B$ на луче, то $\rho(AB) = AB$ \\
        \quad \quad \quad Если нет, то $\rho(A, B) = \rho(AP) + \rho(B, P)$
    \end{center}
    \item Расстояние на сфере
\end{enumerate}
\begin{conj}
    Метрическое пространство $(X, \rho), X$ - множество, $\rho$ - метрика на нем
\end{conj}
\begin{conj}
    Подпространство метрического пространства. \\
    $(X, \rho)$ - метрическое пространство, $Y \subset X$ \\
    $(Y, \rho \vert_{Y \times Y})$ - подпространство метрического пространства
    $(X, \rho)$, где $Y$ - подмножество $X$, а $\rho \vert_{Y \times Y}$ - сужение $\rho$ на $Y \times Y$
\end{conj}
\begin{conj} 
    Открытый шар \vspace*{0.5cm} \\
    $B\callout{r}{радиус}(\calloutup{a}{центр шара}) := {x \in X: \rho(x, a) < r}; \quad r > 0$
\end{conj}
\begin{conj} 
    Замкнутый шар \vspace*{0.5cm} \\
    $\overline{B_r}(a) := {x \in X: \rho(x, a) \leqslant r}; \quad r \geqslant 0$ \\
    $B_r(a) \subset \overline{B_r}(a)$
\end{conj}
\begin{itemize}
    \item \underline{Окрестность} точки $a$ - открытый шар $Br(a)$
\end{itemize}
    \subsection*{Примеры} 
    \begin{enumerate}
        \item Дискретная метрика на $X$
        \begin{itemize}
            \item[] $B_{1/2}(a) = {a}$
            \item[] $B_2(a) = X$ 
        \end{itemize}
        \item $\rho(x, y) = |x - y| \quad B_r(a) = (a - r, a + r)$
        \item Манхэттенская метрика
    
        \begin{tikzpicture}
            \draw[help lines] (0,0) grid (2, 2);
            \draw[dotted] (0,1) coordinate (A) -- (1,2) coordinate (B)
            (1,2) -- (2,1);
            \draw[dotted] (0,1) coordinate (A) -- (1,0) coordinate (C)
            (1,0) -- (2,1);
            \fill[red] ((1,1) circle (2pt);
        \end{tikzpicture} \quad $B_r(a)$
    \end{enumerate}
    \subsection*{Свойства}
    \begin{enumerate}
        \item $B_r(a) \cap B_R(a) = B_{min\{r, R\}}(a)$
        \item Если $x \neq y$, то найдется $r > 0$, такой, что 
        $\overline{Br}(x) \cap \overline{Br}(y) = \varnothing$
        \begin{proof}
            \quad \\
            $r := {{\rho(x, y)}\over{3}}$. Пойдем от противного \\
            Пусть $c \in \overline{Br}(x) \cap \overline{Br}(y) \Longrightarrow
            \begin{cases}
                \rho(x, c) \leqslant r \\
                \rho(y, c) \leqslant r
            \end{cases} \Longrightarrow \rho(x, y) \leqslant \rho(x, c) + \rho(y, c) 
            \leqslant 2r = {{2}\over{3}}\rho(x, y)$ - противоречие
        \end{proof}
    \end{enumerate}
\section{Открытые множества}
\begin{conj}
    Множество $A$ называется открытым, если $A \subset $ метрическому пространству $X$ и $\forall a \in A \; \exists r_{>0} : B_r(a) \subset A$
\end{conj}
\begin{theorem-non}
    Свойства открытых множеств:
    \begin{enumerate}
        \item $\varnothing, X$ - открытые множества 
        \item Объединение любого количества открытых множеств - открытое множество
        \item Пересечение конечного числа открытых множеств - открытое множество
        \item Открытый шарик - открытое множество
    \end{enumerate}
    \begin{proof}
        \quad \\
        \begin{enumerate}
            \item $B_r(a) \subset X$; Для пустого множества нечего проверять, так как там даже точек то нет
            \item $A_{\alpha} \; \alpha \in I$ - открытые множества. $A = \bigcup\limits_{\alpha \in I} A_{\alpha}$ \\
            Возьмем $a \in A$. Тогда $a \in A_{\beta}$ для какого-то $\beta \in I \Longrightarrow A_{\beta}$ - открытое множество 
            $\Longrightarrow B_r(a) \subset A_{\beta}$ для некоторого $r_{>0} \Longrightarrow$ \\
            $B_r(a) \subset A_{\beta} \subset \bigcup\limits_{\alpha \in I} A_{\alpha} = A$
            \item $A_1, A_2, \dots , A_n$ - открытые множества. $A = \bigcap\limits_{k = 1}^{n} A_k$
            Возьмем $a \in A$. Тогда $a \in A_k$ при $k = \{1, 2, \dots , n\} \Longrightarrow
            B_{r_k}(a) \subset A_k$ для некоторого $r_k > 0$ \\
            $r := min\{r_1, r_2, \dots , r_k\} \Longrightarrow B_r(a) \subset B_{r_k}(a) \subset A_k 
            \Longrightarrow B_r(a) \subset \bigcap\limits_{k=1}^n A_k = A$
            \item Рассмотрим $B_R(a)$. Возьмем $b \in B_R(a)$ \\
            $r := R-\rho(a, b) > 0$.
            Докажем, что $x \in B_r(b):$ \\
            $\rho(x, b) < r \Longrightarrow \rho(x,a) \leqslant \rho(x, b) + \rho(b,a) < r + \rho(b,a) = R$
        \end{enumerate}
    \end{proof}
    \notice \\
    В пункте №3 конечность существенна
    $\bigcap\limits_{n=1}^{\infty} B_{1/n}(0) = \bigcap\limits_{n=1}^{\infty}(-{{1}\over{n}}; {{1}\over{n}}) = \{0\}$ Интервал $(-r; \; r)$
\end{theorem-non}
\subsection*{Пример}
$\R \quad \rho(x, y) = \abs{x-y}$ \\
$Y = [0; \; 2)$ \\
Шары в $(Y, \rho)$: \\
\begin{tikzpicture}
    \draw (-.5,0)--(5.5,0);
    \draw[color=black] (0, 0) node {\bfseries[} node[below=9pt]{$0$};
    \draw[color=black] (4, 0) node {\bfseries)} node[below=9pt]{$2$};
\end{tikzpicture} \\
$B_1^Y(0) = \{x \in [0; \;2) : \abs{x - 0} < 1\} = [0; \; 1)$
\section{Внутренние точки. Внутренность множества}
\begin{conj}
    $(X, \beta)$ - метрическое пространство $A \subset X$ \\
    $a \in A, \; a$ - \underline{внутренняя точка множества}, если $B_r(a) \subset A$ для некоторого $r > 0$
    (Открытое множество - такое множество, у которого все точки внутренние) \\
    \underline{Внутренность множества} - множество всех его внутренних точек. Обозначается как $Int A$ \\
\end{conj}
\begin{theorem-non}
    Свойства внутренности: 
    \begin{enumerate}
        \item $Int A \subset A$
        \item $Int A = \bigcup \{G: G \subset A $ и $G$ - открытое$\} =: B$
        \begin{proof}
            \quad \\
            $\bullet \quad Int A \supset B$ \\
            Возьмем $b \in B$. Тогда найдется открытое $G_{\circ} \subset A$, такое, что $b \in G_{\circ} \Longrightarrow$\\
            $\exists r_{>0}$, такой, что $B_r(b) \subset G_{\circ} \subset A \Longrightarrow b$ - внутренняя точка $A$ \\
            $\bullet \quad Int A \subset B$ \\
            Возьмем $a \in Int A \Longrightarrow a$ - внутренняя точка $\Longrightarrow $ открытое множество $ B_r(a) \subset A$ для некоторого $r_{>0}
            \Longrightarrow a \in B_r(a) \subset A$ \\
            $a \in B_r(a) \subset B \Longrightarrow a \in B$
        \end{proof}
        \item $Int A$ - самое большое (по включению) открытое множество, содержащееся в $A$
        \item $Int A$ - открытое множество
        \item $Int A = A \Longleftrightarrow A$ - открытое 
        \item $A \subset B \Longrightarrow Int A \subset Int B$
        \begin{proof}
            Пусть $a \in Int A \Longrightarrow B_r(a) \subset A$ для 
            некоторого $r_{>0} \Longrightarrow a$ - внутренняя точка $B$
        \end{proof}
        \item $Int(A \cap B) = Int A \cap Int B$
        \begin{proof}
            \quad \\
            ``$\subset$'' : $A \cap B \subset A \Longrightarrow Int(A \cap B) \subset Int A$. Это следует из предыдущего пункта. Аналогично для $B$\\ 
            ``$\supset$'' : Пусть $c \in Int A \cap Int B \Longrightarrow 
            \begin{cases}
                c$ - внутренняя точка $A \\
                c$ - внутренняя точка $B 
            \end{cases} \Longrightarrow 
            \begin{cases}
                B_{r_1}(c) \subset A \\
                B_{r_2}(c) \subset B
            \end{cases}$ \vspace{0,2cm} \\ для некоторых $r_1, r_2 > 0 \Longrightarrow 
            B_r(c) \subset A \cap B,$ где $r = min\{r_1, \; r_2\} \Longrightarrow \\ c$ - внутренняя точка $A \cap B$
        \end{proof}
        \item $Int(Int A) = Int A$
        \begin{proof}
            $Int A$ - открытое множество, а внутренность открытого множества совпадает с ним
        \end{proof}
    \end{enumerate}
\end{theorem-non}
\section{Замкнутые множества. Замыкание множества}
\begin{conj}
    $(X, \beta)$ - метрическое пространство $A \subset X$ \\
    $A \subset X \quad A$ - замкнутое, если $X \; \setminus \; A$ -  открытое
\end{conj}
\begin{theorem-non}
    Свойства замкнутых множеств:
    \begin{enumerate}
        \item $\varnothing, X$ - замкнутое множества 
        \item Пересечение любого количества замкнутых множеств - замкнутое множество
        \item Объединение конечного числа замкнутых множеств - замкнутое множество
        \item Замкнутый шарик - замкнутое множество
    \end{enumerate}
    \begin{proof}
        \quad \\
        \begin{enumerate}
            \item[2.] $A_{\alpha} \; \alpha \in I$ - замкнутые множества. $A \overset{?}{\Longrightarrow} \bigcap\limits_{\alpha \in I} A_{\alpha}$ - замкнутое \\
            $\rotatebox[origin=c]{-30}{$\Longrightarrow$} \qquad \qquad \qquad \qquad \qquad \qquad \qquad \qquad \qquad \qquad \rotatebox[origin=c]{30}{$\Longrightarrow$}$ \vspace{0,2cm}\\
            $X \; \setminus \; A$ - открытое $\Longrightarrow \bigcup\limits_{\alpha \in I}(X \; \setminus \; A_{\alpha}) = X \; \setminus \; \bigcap\limits_{\alpha \in I} A_{\alpha}$ - открытое множество 
            \item[3.] $A_1, A_2, \dots , A_n$ - замкнутые множества. $\Longrightarrow X \setminus A_1, X \setminus A_2, \dots , X \; \setminus \; A_n$ - открытые множества \\ 
            $\Longrightarrow \bigcap\limits_{k = 1}^{n} (X \setminus A_k)$ - открытое множество \\
            $\bigcap\limits_{k = 1}^{n} (X \setminus A_k) = X \setminus \bigcup\limits_{k = 1}^{n} A_k \Longrightarrow \bigcup\limits_{k = 1}^{n} A_k$ - замкнутое
            \item[4.] $\overline{B_R}(a)$ - замкнутый шар\\
            Докажем, что $X \; \setminus \; \overline{B_R}(a)$ - открыто
            \begin{proof}
                \quad \\
                $\overline{B_R}(a) = \{x \in X: \rho(x, a) > R\}$ \\
                Возьмем $b \in X \; \setminus \; \overline{B_R}(a) \Longrightarrow \rho(b, a) > R$ \\
                $r := \rho(b, a) - R$ \\
                Докажем, что $B_r \subset X \; \setminus \; B_R(a) \Longleftrightarrow B_r(b) \cap \overline{B_R}(a) = \varnothing$ \\
                От противного. Пусть есть общая точка $c \in B_r(b) \cap \overline{B_R}(a) \Longrightarrow 
                \begin{cases}
                    \rho(c, b) < r \\
                    \rho(c, a) \leqslant R
                \end{cases} \Longrightarrow \vspace{0,2cm}\\
                \rho(c, b) \leqslant \rho(a, c) \leqslant \rho(c, b) < R + r = \rho(a,b)$ \qquad (Так как $\rho(a, c) \leqslant R$ и $\rho(c, b) < r$) \\
                Противоречие.
            \end{proof} 
        \end{enumerate}
    \end{proof}
    \notice \\
    В пункте №3 конечность существенна
    $\bigcup\limits_{n=1}^{\infty}[-1 + {{1}\over{n}}; 1 - {{1}\over{n}}] = (-1; 1)$ \\ Интервал $(-\infty; \; -1] \cup [1; +\infty)$
\end{theorem-non}
\begin{conj}
    Замыкание множества $A$ - пересечение всех замкнутых множеств, содержащих $A$. Обозначаетя как $Cl A$ \\
    $Cl A = \bigcap \{ F: F $ - замкнутое и $ F \supset A \}$ 
\end{conj}
\begin{theorem-non}
    \quad \\
    $X \setminus Cl A = Int(X \setminus A)$ \\
    $X \setminus Int A = Cl(X \setminus A)$
\end{theorem-non}
\begin{proof}
    \quad \\
    $x \in X \setminus Cl A \Longleftrightarrow x \notin Cl A \Longleftrightarrow x \notin F_{\circ}$, где $F_{\circ} \supset A \\
    \Longleftrightarrow 
    \begin{cases}
        x \in X \setminus F_{\circ} =: G_{\circ} $ - открытое$ \\
        G_{\circ} \subset X \setminus A
    \end{cases} \Longleftrightarrow x \in Int(X \setminus A)$ 
\end{proof}
\follow
\quad \\
$Cl A = X \setminus Int(X \setminus A)$ \\
$Int A = X \setminus Cl(X \setminus A)$
\begin{theorem-non}
    Свойства замыкания \\
    \begin{enumerate}
        \item $Cl A$ - замкнутое множество 
        \item $Cl A \supset A$
        \item $A$ - замкнуто $\Longleftrightarrow A = Cl A$
        \begin{proof}
            $A$ - замкнуто $\Longleftrightarrow X \setminus A 
            \Longleftrightarrow X \setminus A = Int(X \setminus A) \Longleftrightarrow \\ A = \underbrace{X \setminus Int(X \setminus A)}\limits_{Cl A}$
        \end{proof}
        \item Если $A \subset B$, то $Cl A\subset Cl B$
        \begin{proof}
            $A \subset B \Longleftrightarrow X \setminus A \supset X \setminus B \Longrightarrow Int(X \setminus A) \supset Int(X \setminus B) \Longrightarrow \\
            \underbrace{X \setminus Int(X \setminus A)}_{Cl A} \subset \underbrace{X \setminus Int(X \setminus B)}_{Cl B}$
        \end{proof}
        \item $Cl(A \cup B) = Cl A \cup Cl B$
        \begin{proof}
            $Cl(A \cup B) = X \setminus Int(\underbrace{X\setminus(A \cup B)}_{(X\setminus A)\cap(X\setminus B)}) = 
            X \setminus Int((X\setminus A) \cap (X\setminus B)) = \vspace{0,2cm} \\ X \setminus (Int(X\setminus A) \cap Int(X\setminus B)) =
            (X \setminus Int(X\setminus A)\cup (X \setminus Int(X\setminus B) = Cl A \cup Cl B$
        \end{proof}
        \item $Cl Cl A = Cl A$
        \begin{proof}
            $Cl A$ - замкнуто + замыкание замкнутого множества - само множество
        \end{proof}
    \end{enumerate}
\end{theorem-non}
\begin{theorem-non}
    $x \in Cl A \Longleftrightarrow$ для любого $r > 0: B_r(x) \cap A \neq \varnothing$
\end{theorem-non}
\begin{proof}
    $x \in Cl A \Longleftrightarrow x \in X \setminus Int(X \setminus A) \Longleftrightarrow
    x \notin Int(X \setminus A) \Longleftrightarrow$ для любого $r > 0: B_r(x)$ не целиком содержится 
    в $X \setminus A \Longleftrightarrow$ для любого $r > 0: B_r(x) \cap A \neq \varnothing$ 
\end{proof}
\follow 
\quad Если $\mathcal{U}$ - открытое и $\mathcal{U} \cap A = \varnothing$, то $\mathcal{U} \cap Cl A = \varnothing$
\ifdefined\niveldos\else
\begin{proof}
    Пусть $x \in \mathcal{U} \cap Cl A \Longrightarrow x \in \mathcal{U}$ - открытое $\exists r > 0 \quad B_r(x) \subset \mathcal{U}\\
    x \in \mathcal{U} \cap Cl A \Longrightarrow x \in Cl A \Longrightarrow B_r(x) \cap A \neq \varnothing \Longrightarrow \mathcal{U} \cap A \neq \varnothing$ - противоречие
\end{proof}
\begin{conj}
    Проколотая окрестность точки $a - B_r(a) \setminus {a}$ \\
    Обозначается как $\overset{\circ}{\mathcal{U}}_a$
\end{conj}
\end{document} 
\fi