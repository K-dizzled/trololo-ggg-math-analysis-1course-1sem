\ifdefined\niveldos\else
\documentclass[12pt,letterpaper]{report}
 
%Russian-specific packages
%--------------------------------------
\usepackage[T2A]{fontenc}
\usepackage[utf8]{inputenc}
\usepackage[russian]{babel}
\usepackage[mathscr]{euscript}
\usepackage{mathrsfs}
\usepackage{amsmath,amsthm,amssymb,latexsym,amsfonts}
\usepackage{showkeys}
\usepackage{pythonhighlight}
\usepackage{mdframed}
\usepackage{lipsum}
\usepackage{soul}
\usepackage{amsmath,amssymb}
\usepackage{parskip}
\usepackage{graphicx}
\usepackage{mathtools}
\usepackage{stackengine} 
\usepackage{tocloft}
\usepackage{xcolor}
\usepackage{hyperref}
\usepackage{thmtools}

\usepackage{amsthm}
\newtheorem{theorem}{Теорема}
\newtheorem{lemma}[theorem]{Лемма}
\newtheorem{conj}[theorem]{Определение}

% Definindo novas cores
\definecolor{verde}{rgb}{0.25,0.5,0.35}
\definecolor{jpurple}{rgb}{0.5,0,0.35}
\definecolor{darkgreen}{rgb}{0.0, 0.2, 0.13}
%\definecolor{oldmauve}{rgb}{0.4, 0.19, 0.28}
% Configurando layout para mostrar codigos Java
\usepackage{listings}

\newcommand{\estiloJava}{
\lstset{
    language=Java,
    basicstyle=\ttfamily\small,
    keywordstyle=\color{jpurple}\bfseries,
    stringstyle=\color{red},
    commentstyle=\color{verde},
    morecomment=[s][\color{blue}]{/**}{*/},
    extendedchars=true,
    showspaces=false,
    showstringspaces=false,
    numbers=left,
    numberstyle=\tiny,
    breaklines=true,
    backgroundcolor=\color{cyan!10},
    breakautoindent=true,
    captionpos=b,
    xleftmargin=0pt,
    tabsize=2
}}

\newcommand{\estiloR}{
  \lstset{ %
    language=R,                     % the language of the code
    basicstyle=\footnotesize,       % the size of the fonts that are used for the code
    numbers=left,                   % where to put the line-numbers
    numberstyle=\tiny\color{gray},  % the style that is used for the line-numbers
    stepnumber=1,                   % the step between two line-numbers. If it's 1, each line
                                    % will be numbered
    numbersep=5pt,                  % how far the line-numbers are from the code
    backgroundcolor=\color{white},  % choose the background color. You must add \usepackage{color}
    showspaces=false,               % show spaces adding particular underscores
    showstringspaces=false,         % underline spaces within strings
    showtabs=false,                 % show tabs within strings adding particular underscores
    frame=single,                   % adds a frame around the code
    rulecolor=\color{black},        % if not set, the frame-color may be changed on line-breaks within not-black text (e.g. commens (green here))
    tabsize=2,                      % sets default tabsize to 2 spaces
    captionpos=b,                   % sets the caption-position to bottom
    breaklines=true,                % sets automatic line breaking
    breakatwhitespace=false,        % sets if automatic breaks should only happen at whitespace
    title=\lstname,                 % show the filename of files included with \lstinputlisting;
                                    % also try caption instead of title
    keywordstyle=\color{blue},      % keyword style
    commentstyle=\color{darkgreen},   % comment style
    stringstyle=\color{red},      % string literal style
    escapeinside={\%*}{*)},         % if you want to add a comment within your code
    morekeywords={*,...}          % if you want to add more keywords to the set
}}

\definecolor{linkcolor}{HTML}{47528f} % цвет ссылок
\definecolor{urlcolor}{HTML}{47528f} % цвет гиперссылок
\hypersetup{pdfstartview=FitH,  linkcolor=linkcolor,urlcolor=urlcolor, colorlinks=true}
\newcommand{\RomanNumeralCaps}[1]
  {\MakeUppercase{\romannumeral #1}}
\usepackage{titlesec}
\renewcommand{\thesection}{\arabic{section}}
\renewcommand{\listtheoremname}{Определения и теоремы}
\renewcommand\qedsymbol{$\blacksquare$}
\newcommand\oast{\stackMath\mathbin{\stackinset{c}{0ex}{c}{0ex}{\ast}{\bigcirc}}}
\makeatletter
\renewenvironment{proof}[1][\proofname]{%
   \par\pushQED{\qed}\normalfont%
   \topsep6\p@\@plus6\p@\relax
   \trivlist\item[\hskip\labelsep\bfseries#1\@addpunct{.}]%
   \ignorespaces
}{%
   \popQED\endtrivlist\@endpefalse
}
\makeatother
%--------------------------------------
\DeclareMathOperator{\Mr}{M_{\mathbb{R}}}
\addtolength{\oddsidemargin}{-.875in}
	\addtolength{\evensidemargin}{-.875in}
	\addtolength{\textwidth}{1.75in}

	\addtolength{\topmargin}{-.875in}
	\addtolength{\textheight}{1.75in}
%%%%%%%%%%%%%%%%%%%%%%%%%%%%%%%%%%
%                                %
%                                %
%              Title             %
%                                %
%                                %
%%%%%%%%%%%%%%%%%%%%%%%%%%%%%%%%%%
\title{Конспект лекций по математическому анализу}
\author{Храбров Александр Игоревич}
\date{Первый курс, первый семестр 2020}
\begin{document}
\fi
\maketitle
%%%%%%%%%%%%%%%%%%%%%%%%%%%%%%%%%%
%                                %
%                                %
%        Table of content        %
%                                %
%                                %
%%%%%%%%%%%%%%%%%%%%%%%%%%%%%%%%%%
\tableofcontents
%\listoftheorems[ignore={lemma},show={conj, theorem}]
% To show Table of contents with
% definitions and lemmas
%----------------------------------------
% \listoftheorems[ignoreall,show={lemma}]
% \listoftheorems[ignoreall,show={conj}]
\newpage
\chapter{Введение}
\section{Множества}
\begin{conj} Множество - набор уникальных элементов \end{conj}

Множества - большие буквы $A, B,\dots$ \\
Элементы множеств - маленькие буквы $a, b,\dots$ \\
$x \in A - x$ пренадлежит $A$ \\
$x \notin A - x$ не пренадлежит $A$ \\
$\mathbb{N} = \{1, 2, 3, \dots\} \\
\mathbb{Z, Q} = \{{{m}\over{n}} : m \in \mathbb{Z}, n \in\mathbb{N}\} \\
\mathbb{R}$ - вещественные числа \\
$\mathbb{R}$ - комплексные числа \\
\begin{theorem} Правила Де Моргана \end{theorem}
    \begin{itemize}
        \item[] $A \; \setminus \; (\bigcup\limits_{\alpha \in I} B_{\alpha}) 
        = \bigcap\limits_{\alpha \in I}(A \setminus B_{\alpha})$

        \item[] $A \; \setminus \; (\bigcap\limits_{\alpha \in I} B_{\alpha}) 
        = \bigcup\limits_{\alpha \in I}(A \setminus B_{\alpha})$
    \end{itemize}
\begin{proof}
    Докажем для первой формулы. Вторая доказывается аналогично. \\
    $x \in A \; \setminus \; (\bigcup\limits_{\alpha \in I} B_{\alpha}) 
    \Longleftrightarrow \begin{cases}
        x \in A \\
        x \notin \bigcup\limits_{\alpha \in I} B_{\alpha}
    \end{cases}
    \Longleftrightarrow \begin{cases}
        x \in A \\
        x \notin B_{\alpha} \; \; $при всех$ \; \alpha
    \end{cases} 
    \Longleftrightarrow x \in A \; \setminus \; B_{\alpha}$ при всех $\alpha \in I
    \Longleftrightarrow x \in \bigcap\limits_{\alpha \in I}(A \setminus B_{\alpha})$ 
\end{proof} \newpage
\begin{theorem} Операции над множествами \end{theorem}
\begin{itemize}
    \item $A \cup B = \{x: x \in A $ или $ x \in B\}$
    \item $A \cap B = \{x: x \in A, x  \in B\}$
    \item $A \; \setminus \; B = \{x: x \in A, x  \notin B\}$
    \item $A \bigtriangleup B = (A \; \setminus \; B) \cup (B \; \setminus \; A)$
\end{itemize}
Замечание: $\bigtriangleup, \cup, \cap$ - комммутативны, ассоциативны
\begin{conj} 
    Декартово произведение множеств 
    $A \times B = \{\langle a, b \rangle : a \in A; b \in B \}$ 
\end{conj}
\begin{theorem} \end{theorem}
    \begin{itemize}
        \item[] $A \cap \bigcup\limits_{\alpha \in I} B_{\alpha} =
        \bigcup\limits_{\alpha \in I}(A \cap B_{\alpha}) $

        \item[] $A \cup \bigcap\limits_{\alpha \in I} B_{\alpha} =
        \bigcap\limits_{\alpha \in I}(A \cup B_{\alpha}) $
    \end{itemize}
\begin{proof}
        $x \in A \cap \bigcup\limits_{\alpha \in I} B_{\alpha}
        \Longleftrightarrow \begin{cases}
            x \in A \\
            x \in \bigcup\limits_{\alpha \in I} B_{\alpha}
        \end{cases} \Longleftrightarrow \begin{cases}
            x \in A \\
            x \in B_{\alpha}$ для некоторых $\alpha \in I
        \end{cases} \Longleftrightarrow 
        x \in A \cap B_{\alpha}$ для некоторых $\alpha \in I 
        \Longleftrightarrow 
        x \in \bigcup\limits_{\alpha \in I}(A \cap B_{\alpha})$
    \end{proof}
\begin{conj} 
    Упорядоченная пара $ \langle a, b \rangle $ - пара  ``пронумерованных'' элементов
\end{conj}
    $  \langle a, b \rangle $ = $  \langle c, d \rangle \rotatebox[origin=c]{150}{$\Longleftrightarrow$}$ 
\begin{scriptsize}
\estiloJava
\begin{lstlisting}[caption={}, label=]
    ((a == c) && (b == d))
\end{lstlisting}
\end{scriptsize}
\section{Отношения}







\ifdefined\niveldos\else
\end{document} 
\fi
