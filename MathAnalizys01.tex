\ifdefined\niveldos\else
\documentclass[12pt,letterpaper]{report}
 
%Russian-specific packages
%--------------------------------------
\usepackage[T2A]{fontenc}
\usepackage[utf8]{inputenc}
\usepackage[russian]{babel}
\usepackage[mathscr]{euscript}
\usepackage{mathrsfs}
\usepackage{amsmath,amsthm,amssymb,latexsym,amsfonts}
\usepackage{showkeys}
\usepackage{pythonhighlight}
\usepackage{mdframed}
\usepackage{lipsum}
\usepackage{soul}
\usepackage{amsmath,amssymb}
\usepackage{parskip}
\usepackage{graphicx}
\usepackage{mathtools}
\usepackage{stackengine} 
\usepackage{tocloft}
\usepackage{xcolor}
\usepackage{hyperref}
\usepackage{thmtools}
\usepackage{tikz}

\usepackage{amsthm}
\newtheorem{theorem}{Теорема}
\newtheorem{lemma}[theorem]{Лемма}
\newtheorem{conj}[theorem]{Определение}

% Definindo novas cores
\definecolor{verde}{rgb}{0.25,0.5,0.35}
\definecolor{jpurple}{rgb}{0.5,0,0.35}
\definecolor{darkgreen}{rgb}{0.0, 0.2, 0.13}
%\definecolor{oldmauve}{rgb}{0.4, 0.19, 0.28}
% Configurando layout para mostrar codigos Java
\usepackage{listings}

\newcommand{\estiloJava}{
\lstset{
    language=Java,
    basicstyle=\ttfamily\small,
    keywordstyle=\color{jpurple}\bfseries,
    stringstyle=\color{red},
    commentstyle=\color{verde},
    morecomment=[s][\color{blue}]{/**}{*/},
    extendedchars=true,
    showspaces=false,
    showstringspaces=false,
    numbers=left,
    numberstyle=\tiny,
    breaklines=true,
    backgroundcolor=\color{cyan!10},
    breakautoindent=true,
    captionpos=b,
    xleftmargin=0pt,
    tabsize=2
}}

\newcommand{\estiloR}{
  \lstset{ %
    language=R,                     % the language of the code
    basicstyle=\footnotesize,       % the size of the fonts that are used for the code
    numbers=left,                   % where to put the line-numbers
    numberstyle=\tiny\color{gray},  % the style that is used for the line-numbers
    stepnumber=1,                   % the step between two line-numbers. If it's 1, each line
                                    % will be numbered
    numbersep=5pt,                  % how far the line-numbers are from the code
    backgroundcolor=\color{white},  % choose the background color. You must add \usepackage{color}
    showspaces=false,               % show spaces adding particular underscores
    showstringspaces=false,         % underline spaces within strings
    showtabs=false,                 % show tabs within strings adding particular underscores
    frame=single,                   % adds a frame around the code
    rulecolor=\color{black},        % if not set, the frame-color may be changed on line-breaks within not-black text (e.g. commens (green here))
    tabsize=2,                      % sets default tabsize to 2 spaces
    captionpos=b,                   % sets the caption-position to bottom
    breaklines=true,                % sets automatic line breaking
    breakatwhitespace=false,        % sets if automatic breaks should only happen at whitespace
    title=\lstname,                 % show the filename of files included with \lstinputlisting;
                                    % also try caption instead of title
    keywordstyle=\color{blue},      % keyword style
    commentstyle=\color{darkgreen},   % comment style
    stringstyle=\color{red},      % string literal style
    escapeinside={\%*}{*)},         % if you want to add a comment within your code
    morekeywords={*,...}          % if you want to add more keywords to the set
}}

\DeclarePairedDelimiter\abs{\lvert}{\rvert}%
\DeclarePairedDelimiter\norm{\lVert}{\rVert}%

% Swap the definition of \abs* and \norm*, so that \abs
% and \norm resizes the size of the brackets, and the 
% starred version does not.
\makeatletter
\let\oldabs\abs
\def\abs{\@ifstar{\oldabs}{\oldabs*}}
%
\let\oldnorm\norm
\def\norm{\@ifstar{\oldnorm}{\oldnorm*}}
\makeatother

\newcommand*\circled[1]{\tikz[baseline=(char.base)]{
            \node[shape=circle,draw,inner sep=2pt] (char) {#1};}}

\definecolor{linkcolor}{HTML}{47528f} % цвет ссылок
\definecolor{urlcolor}{HTML}{47528f} % цвет гиперссылок
\hypersetup{pdfstartview=FitH,  linkcolor=linkcolor,urlcolor=urlcolor, colorlinks=true}
\newcommand{\RomanNumeralCaps}[1]
  {\MakeUppercase{\romannumeral #1}}
\usepackage{titlesec}
\renewcommand{\thesection}{\arabic{section}}
\renewcommand{\listtheoremname}{Определения и теоремы}
\renewcommand\qedsymbol{$\blacksquare$}
\newcommand\oast{\stackMath\mathbin{\stackinset{c}{0ex}{c}{0ex}{\ast}{\bigcirc}}}
\makeatletter
\renewenvironment{proof}[1][\proofname]{%
   \par\pushQED{\qed}\normalfont%
   \topsep6\p@\@plus6\p@\relax
   \trivlist\item[\hskip\labelsep\bfseries#1\@addpunct{.}]%
   \ignorespaces
}{%
   \popQED\endtrivlist\@endpefalse
}
\makeatother
%--------------------------------------
\DeclareMathOperator{\Mr}{M_{\mathbb{R}}}
\addtolength{\oddsidemargin}{-.875in}
	\addtolength{\evensidemargin}{-.875in}
	\addtolength{\textwidth}{1.75in}

	\addtolength{\topmargin}{-.875in}
	\addtolength{\textheight}{1.75in}
%%%%%%%%%%%%%%%%%%%%%%%%%%%%%%%%%%
%                                %
%                                %
%              Title             %
%                                %
%                                %
%%%%%%%%%%%%%%%%%%%%%%%%%%%%%%%%%%
\title{Конспект лекций по математическому анализу}
\author{Храбров Александр Игоревич}
\date{Первый курс, первый семестр 2020}
\begin{document}
\fi
\maketitle
%%%%%%%%%%%%%%%%%%%%%%%%%%%%%%%%%%
%                                %
%                                %
%        Table of content        %
%                                %
%                                %
%%%%%%%%%%%%%%%%%%%%%%%%%%%%%%%%%%
\tableofcontents
%\listoftheorems[ignore={lemma},show={conj, theorem}]
% To show Table of contents with
% definitions and lemmas
%----------------------------------------
% \listoftheorems[ignoreall,show={lemma}]
% \listoftheorems[ignoreall,show={conj}]
\newpage

\section{Теорема Штольца (для неопределённости $\frac \infty \infty$)}

\begin{theorem} Штольца № 1 \end{theorem}
    Пусть $(y_n)$ строго возрастает и $\lim y_n = +\infty$. 
    Тогда если $\lim \frac{x_n - x_{n - 1}}{y_n - y_{n - 1}} = l \in
    \overline{\mathbb{R}}$, то $\lim \frac{x_n}{y_n} = l$.
\begin{proof}
    \textbf{Ключевой случай $l = 0$:}

    Пусть \[\varepsilon_n := 
    \frac{x_n - x_{n - 1}}{y_n - y_{n - 1}} \rightarrow 0\]

    Зафиксируем $\varepsilon > 0$ и найдём $m$, т.ч. $
    \abs{\varepsilon_n} < \varepsilon$ при $n \geq m$.

    \[x_n - x_m = (x_n - x_{n - 1}) + (x_{n - 1} - x_{n - 2})
    + ... + (x_{m + 1} - x_m) = \sum_{k=m+1}^n \varepsilon_k
    \cdot (y_k - y_{k - 1})\]
    \[\abs{x_n - x_m} \leq \sum_{k=m+1}^n \abs{\varepsilon_k}
    \cdot (y_k - y_{k - 1}) < \sum_{k=m+1}^n \varepsilon
    \cdot (y_k - y_{k - 1}) = \varepsilon \cdot \sum_{k=m+1}^n 
    (y_k - y_{k - 1}) = \varepsilon \cdot (y_n - y_m) <
    \varepsilon y_n\]

    Можно считать, что $y_m > 0$ (по теореме о стабилизации знака).

    Заметим, что $\abs{x_m}$ фиксировано, а $y_n \rightarrow +\infty$
    $\Rightarrow$ $\lim \frac{\abs{x_m}}{y_n} = 0$ и
    $\frac{\abs{x_m}}{y_n} < \varepsilon$, начиная с некоторого номера.

    \[\abs{x_n} \leq \abs{x_m} + \abs{x_n - x_m} < 
    \abs{x_m} + \varepsilon y_n \Rightarrow \abs{\frac{x_n}{y_n}} <
    \frac{\abs{x_m}}{y_n} + \varepsilon < 2 \varepsilon\]
    начиная с некоторого номера $\Rightarrow \lim \abs{\frac{x_n}{y_n}} = 0 = l$.

    \textbf{Случай $l \in \mathbb{R}$:}
    \[\widetilde{x_n} := x_n - l \cdot y_n, \widetilde{x_n} -
    \widetilde{x_{n-1}} = x_n - x_{n-1} - l \cdot (y_n - y_{n-1})\]
    \[\frac{\widetilde{x_n} - \widetilde{x_{n-1}}}
    {y_n - y_{n - 1}} = \frac{x_n - x_{n - 1}}{y_n - y_{n - 1}}  - l
    \rightarrow 0 \xRightarrow[]{l = 0} \frac{\widetilde{x_n}}{y_n}
    \rightarrow 0 \Rightarrow \frac{\widetilde{x_n}}{y_n} =
    \frac{x_n - l \cdot y_n}{y_n} = \frac{x_n}{y_n} - l
    \rightarrow 0 \Rightarrow \frac{x_n}{y_n} \rightarrow l\]

    \textbf{Случай $l = +\infty$:}

    \[\frac{x_n - x_{n - 1}}{y_n - y_{n - 1}} \rightarrow +\infty
    \Rightarrow \frac{x_n - x_{n - 1}}{y_n - y_{n - 1}} > 1\] 
    начиная с некоторого номера \\
    $\Rightarrow x_n - x_{n - 1} > y_n - y_{n - 1} > 0 \Rightarrow
    x_n$ строго возрастает с нек. номера $m \Rightarrow$\\
    $\Rightarrow x_n - x_m > y_n - y_m \Rightarrow
    x_n > y_n + (x_m - y_m) \rightarrow +\infty \Rightarrow
    x_n \rightarrow +\infty$

    Рассмотрим
    \[\frac{y_n - y_{n - 1}}{x_n - x_{n - 1}} \rightarrow 0
    \xRightarrow[]{l = 0} \frac{y_n}{x_n} \rightarrow 0 \Rightarrow
    \frac{x_n}{y_n} \rightarrow +\infty\] (а не $\infty$, т.к. 
    $x_n > 0, y_n > 0$ с нек. номера)

    \textbf{Случай $l = -\infty$}

    Пусть $\widetilde{x_n} := -x_n$.
    \[\frac{x_n - x_{n-1}}{y_n - y_{n - 1}} \rightarrow -\infty
    \Rightarrow \frac{\widetilde{x_n} - \widetilde{x_{n-1}}}
    {y_n - y_{n - 1}} = -\frac{x_n - x_{n-1}}{y_n - y{n - 1}}
    \rightarrow +\infty \Rightarrow -\frac{x_n}{y_n} =
    \frac{\widetilde{x_n}}{y_n} \rightarrow +\infty
    \Rightarrow \frac{x_n}{y_n} \rightarrow -\infty\] 

\end{proof}

\textbf{Следствие.}
\[\text{Если } \lim a_n = a \in \overline{\mathbb{R}}
\text{, то } \lim \frac{a_1 + a_2 + ... + a_n}{n} = a\] 
\begin{proof}
    \[x_n := \sum_{k=1}^n a_k, \quad y_n := n \nearrow +\infty\]
    \[\lim \frac{x_n - x_{n-1}}{y_n - y_{n-1}} = \lim{a_n}{n - (n - 1)}
    = \lim a_n = a \Rightarrow \lim \frac{a_1 + a_2 + ... + a_n}{n} = a\]
\end{proof}

Пример. Найти предел:
\[m \in \mathbb{N}, \quad \frac{1}{n^{m + 1}} \cdot \sum_{k=1}^{n} k^m\]

\[x_n := \sum_{k=1}^{n} k^m, \quad y_n := n^{m + 1} \nearrow +\infty\]
\begin{gather*}
    \lim \frac{y_n - y_{n - 1}}{x_n - x_{n - 1}} =
    \lim \frac{n^{m + 1} - (n - 1)^{m + 1}}{n^m} =
    \lim \frac{n^{m+1} - (n^{m+1} + \sum_{k = 1}^{m + 1} 
    (C_{m+1}^k (-1)^k n^{m+1-k})}{n^m}) = \\
    = \lim \sum_{k = 1}^{m + 1} 
    ((-1)^{k+1} \cdot \frac{C_{m+1}^k}{n^{k - 1}}) =
    \lim C_{m + 1}^1 + \lim \sum_{k = 2}^{m + 1} 
    ((-1)^{k+1} \cdot \frac{C_{m+1}^k}{n^{k - 1}}) =
    (m + 1) + 0 = m + 1
\end{gather*}
\[\lim\frac{x_n - x_{n - 1}}{y_n - y_{n-1}} = \frac1{m + 1}
\Rightarrow \lim\frac{x_n}{y_n} = \frac1{m + 1}\]

\section{Теорема Штольца (для неопределённости $\frac 0 0$)}

\begin{theorem}Штольца № 2\end{theorem}
\[0<y_n<y_{n-1} \text{ и } \lim x_n = \lim y_n = 0
\text{ Тогда если }\lim\frac{x_n - x_{n-1}}{y_n - y_{n-1}} = l \in
\overline{\mathbb{R}}\text{, то }\lim \frac{x_n}{y_n} = l\]

\begin{proof}
    \textbf{Случай $l = 0$:}

    Пусть \[\varepsilon_n := 
    \frac{x_n - x_{n - 1}}{y_n - y_{n - 1}} \rightarrow 0\]

    Зафиксируем $\varepsilon > 0$ и найдём $m$, т.ч. $
    \abs{\varepsilon_n} < \varepsilon$ при $n \geq m$.
    \[x_n - x_m = \sum_{k=m+1}^n(x_k - x_{k-1}) =
    \sum_{k=m+1}^n \varepsilon_k (y_k - y_{k-1}) \Rightarrow
    \abs{x_n - x_m} \leq\] \[\leq \sum_{k=m+1}^n -\abs{\varepsilon_k}
    (y_k - y_{k-1}) < \varepsilon \sum_{k=m+1}^n (y_k - y_{k-1}) =
    \varepsilon (y_m - y_n)\]
    \[(x_n - x_m) < \varepsilon (y_m - y_n)\]
    \[\text{Устремим } n \text{ к } +\infty \Rightarrow
    \abs{x_n - x_m} \rightarrow \abs{-x_m} = x_m, \quad
    \varepsilon (y_m - y_n) \rightarrow \varepsilon y_m \Rightarrow\]
    \[\Rightarrow \text{по пред. переходу в нер., при }
    m \geq \text{нек. } N \quad \abs{x_m} < \varepsilon y_m
    \Rightarrow \abs{\frac{x_m}{y_m}} < \varepsilon
    \Rightarrow \lim \frac{x_m}{y_m} = 0\]

    \textbf{Случай $l \in \overline{\mathbb{R}}$:}
    Так же, как в теореме Штольца № 1
    \[\widetilde{x_n} := x_n - l \cdot y_n, \widetilde{x_n} -
    \widetilde{x_{n-1}} = x_n - x_{n-1} - l \cdot (y_n - y_{n-1})\]
    \[\frac{\widetilde{x_n} - \widetilde{x_{n-1}}}
    {y_n - y_{n - 1}} = \frac{x_n - x_{n - 1}}{y_n - y_{n - 1}}  - l
    \rightarrow 0 \xRightarrow[]{l = 0} \frac{\widetilde{x_n}}{y_n}
    \rightarrow 0 \Rightarrow \frac{\widetilde{x_n}}{y_n} =
    \frac{x_n - l \cdot y_n}{y_n} = \frac{x_n}{y_n} - l
    \rightarrow 0 \Rightarrow \frac{x_n}{y_n} \rightarrow l\]

    \textbf{Случай $l = +\infty$:}
    \[\frac{x_n - x_{n - 1}}{y_n - y_{n - 1}} \rightarrow +\infty
    \Rightarrow \frac{x_{n - 1} - x_n}{y_{n - 1} - y_{n}} =
    \frac{x_n - x_{n - 1}}{y_n - y_{n - 1}} > 1 \text{ начиная
    с некоторого номера} \Rightarrow\] \[\Rightarrow
    x_{n - 1} - x_n > y_{n - 1} - y_n > 0 \Rightarrow x_n
    \text{ строго убывает} \Rightarrow \lim \frac{y_n - y_{n - 1}}
    {x_n - x_{n - 1}} = 0 \xRightarrow[]{l = 0} \lim \frac{y_n}{x_n} = 0
    \Rightarrow \] \[\Rightarrow \frac{x_n}{y_n} = +\infty\]

    \textbf{Случай $l = -\infty$:}
    Так же, как в теореме Штольца № 1

    Пусть $\widetilde{x_n} := -x_n$.
    \[\frac{x_n - x_{n-1}}{y_n - y_{n - 1}} \rightarrow -\infty
    \Rightarrow \frac{\widetilde{x_n} - \widetilde{x_{n-1}}}
    {y_n - y_{n - 1}} = -\frac{x_n - x_{n-1}}{y_n - y{n - 1}}
    \rightarrow +\infty \Rightarrow -\frac{x_n}{y_n} =
    \frac{\widetilde{x_n}}{y_n} \rightarrow +\infty
    \Rightarrow \frac{x_n}{y_n} \rightarrow -\infty\] 

\end{proof}

\section{Подпоследовательности (определение и простейшие свойства).
Теорема о стягивающихся отрезках)}

\begin{conj}
Последовательность $(x_n)$, $n_1 < n_2 < n_3 < ...$ Тогда
$(x_{n_k})$ - подпоследовательность.
\end{conj}
\textbf{Замечание.} $n_k \geq k$ (по индукции)

\textbf{Свойства:}
\begin{enumerate}
    \item Если последовательность имеет предел, то подпоследовательность
    имеет тот же предел.
    \item Пусть две подпоследовательности в объединении дают исходную
    последовательность. Если подпоследовательности имеют одинаковый
    предел, то исходная последовательность имеет тот же предел.
\end{enumerate}

\begin{theorem}О стягивающихся отрезках.\end{theorem}
\[\text{Пусть }[a_1; b_1] \supset [a_2; b_2] \supset [a_3; b_3] 
\supset ... \text{ и } \lim (b_n - a_n) = 0\]
Тогда существует единственная точка $c$, принадлежащая всем отрезкам
и $\lim a_n = \lim b_n = c$.
\[\text{Т.е. } \bigcap_{n = 1}^{+\infty} [a_n; b_n] = {c}\]

\begin{proof}
    По теореме о вложенных отрезках $\bigcap_{n = 1}^{+\infty} [a_n; b_n]
    \neq \varnothing$.
    \[\text{Пусть } c,d \in \bigcap_{n = 1}^{+\infty} [a_n; b_n]
    \Rightarrow c, d \in [a_n; b_n] \forall n; \text{ НУО, } d \geq c\]
    \[0 \leq d - c \leq b_n - a_n \rightarrow 0 \Rightarrow c = d
    \text{, иначе } \exists n : b_n - a_n < \varepsilon = d - c\]
    \[0 \leq c - a_n \leq b_n - a_n \rightarrow 0
    \xRightarrow[]{\text{2 мил.}}
    c - a_n \rightarrow 0 \Rightarrow \lim a_n = c\]
    \[0 \leq b_n - c \leq b_n - a_n \rightarrow 0
    \xRightarrow[]{\text{2 мил.}}
    b_n - c \rightarrow 0 \Rightarrow \lim b_n = c\]
\end{proof}

\section{Теорема Больцано-Вейерштрасса в $\mathbb{R}$}
\begin{theorem}
    Из любой ограниченной последовательности можно выделить
    сходящуюся подпоследовательность.
\end{theorem}
\begin{proof}
${x_n}$ ограничено $\Rightarrow x_n \in [a; b]$

В каком-то из отрезков $[a; \frac{a + b}{2}]$ и $[\frac{a + b}{2}; b]$
содержится бесконечное число членов послед.\\
Назовём этот отрезок $[a_1; b_1]$.

В каком-то из отрезков $[a_1; \frac{a_1 + b_1}{2}]$ и 
$[\frac{a_1 + b_1}{2}; b_1]$
содержится бесконечное число членов послед.\\
Назовём этот отрезок $[a_2; b_2]$.

В каком-то из отрезков $[a_2; \frac{a_2 + b_2}{2}]$ и 
$[\frac{a_2 + b_2}{2}; b_2]$
содержится бесконечное число членов послед.\\
Назовём этот отрезок $[a_3; b_3]$.
\[...\]
\[[a; b] \supset [a_1; b_1] \supset [a_2; b_2] \supset
[a_3; b_3] \supset ...\]
\[b_n - a_n = \frac{b - a}{2^n} \rightarrow 0\]

Тогда по теореме о стягивающихся отрезках $\lim a_n = \lim b_n = c$

Выберем подпоследовательность. Берём $[a_1; b_1]$, в нём есть
какой-то член последовательности, назовём его $x_{n_1}$.

В $[a_2; b_2]$ содержится бесконечное число членов последовательности
$\Rightarrow$ есть член последовательности с номером, большим $n_1$.
Обозначим его $x_{n_2}$, тогда $n_2 > n_1$.
\[...\]
$x_{n_k} \in [a_k; b_k], n_1 < n_2 < n_3 < ...$, значит построили
подпоследовательность.

\[a_k \rightarrow c, \,\, b_k \rightarrow c \quad a_k \leq x_{n_k} \leq b_k
\xRightarrow[]{\text{2 мил.}} \lim x_{n_k} = c \]
\end{proof}

\section{Аналог теоремы Больцано–Вейерштрасса для неограниченной 
последовательности. Частичные пределы. Теорема о характеристике 
частичных пределов.}

\begin{theorem}\end{theorem}
\begin{enumerate}
    \item Неограниченная монотонная последовательность стремится
    к $+\infty$ или к $-\infty$.
    \item Из любой неограниченной последовательности можно выделить
    подпоследовательность, стремящуюся к $+\infty$ или к $-\infty$.
\end{enumerate}
\begin{proof}.
\begin{enumerate}
    \item Пусть $(x_n)$ возрастает. $(x_n)$ неограничена $\Rightarrow$
    никакое $u$ не является верхней границей $\Rightarrow \exists m : x_m
    x_m > u \Rightarrow u < x_m \leq x_{m + 1} \leq x_{m + 2} \leq \dots
    \Rightarrow x_n > u$, начиная с некоторого номера $\Rightarrow
    \lim x_n = +\infty$

    \item Пусть $(x_n)$ неограничена сверху.
    
    $1$ не является верхней границей $\Rightarrow \exists x_{n_1} > 1$;\\
    $\max\{2, x_1, x_2, \dots, x_{n_1}\}$ не является верхней границей
    $\Rightarrow \exists x_{n_2} > \max\{\dots\} \Rightarrow x_{n_2} > 2,\\
    n_2 > n_1$;\\
    $\max\{3, x_1, x_2, \dots, x_{n_2}\}$ не является верхней границей
    $\Rightarrow \exists x_{n_3} > \max\{\dots\} \Rightarrow x_{n_3} > 3,\\
    n_3 > n_2$;\\
    и т.д.

    Итого, $x_{n_k} > k$ и $n_1 < n_2 < \dots \Rightarrow (x_{n_k})$
    -- подпоследовательность $(x_n)$ и $\lim x_{n_k} = +\infty$ по
    предельному переходу в неравенстве.
\end{enumerate}
\end{proof}

\begin{conj}
$a$ -- частичный предел последовательности $(x_n)$, если найдётся
подпоследовательность $x_{n_k} \rightarrow a$.
\end{conj}
\begin{theorem}
$a$ -- частичный предел последовательности $\Leftrightarrow$
в любой окрестности точки $a$ найдётся бесконечное число членов
последовательности.
\end{theorem}
\begin{proof} $ $

    ''$\Longrightarrow$'':

    Если $a = \lim x_{n_k}$ и $U_a$ -- окрестность точки $a$, то
    все $x_{n_k}$ кроме конечного числа лежат в $U_a \Rightarrow$
    в $U_a$ лежит бесконечное число членов последовательности $(x_n)$.

    ''$\Longleftarrow$'':

    Будем строить подпоследовательность, имеющую предел $a$.

    В $B_{1}(a)$ найдётся бесконечное число членов последовательности,
    возьмём какой-то и назовём его $x_{n_1}$.\\
    В $B_{1/2}(a)$ найдётся бесконечное число членов
    последовательности, значит найдётся член $(x_n)$ с индексом, большим
    $n_1$, назовём его $x_{n_2}$.\\
    В $B_{1/3}(a)$ найдётся бесконечное число членов
    последовательности, значит найдётся член $(x_n)$ с индексом, большим
    $n_2$, назовём его $x_{n_3}$.\\
    $\dots$

    $n_1 < n_2 < n_3 < \dots$\\
    $x_{n_k} \in B_{1/k}(a) \Rightarrow \rho(x_{n_k}, a) < \frac1k
    \Rightarrow \rho(x_{n_k}, a) \rightarrow 0 \Rightarrow
    \lim x_{n_k} = a$

\end{proof}

\section{Фундаментальные последовательности. Свойства. Критерий Коши.}

\begin{conj}
Фундаментальная последовательность (сходящаяся в себе,
последовательность Коши)
\end{conj}
Пусть $(X, \rho)$ -- метрическое пространство. $x_n \in X$.
$x_n$ -- фундаментальная последовательность, если $\forall
\varepsilon > 0 \,\, \exists N : \forall n, m \geq N \,\,
\rho(x_n, x_m) < \varepsilon$

\textbf{Свойства:}
\begin{enumerate}
    \item Сходящаяся последовательность фундаментальна.
    
    \textbf{Доказательство:}\\
    Пусть $\lim x_n := a$. Зафиксируем $\varepsilon > 0$.
    Тогда $\exists N :\\ \forall n \geq N \,\,\, \rho(x_n, a) < 
    \frac{1}{2} \varepsilon$\\ 
    $\forall m \geq N \,\,\, \rho(x_m, a) < \frac{1}{2} \varepsilon$\\
    $\Rightarrow \rho(x_n, x_m) \leq \rho(x_n, a) + \rho(x_m, a) <
    \varepsilon$

    \item Фундаментальная последовательность ограничена
    
    \textbf{Доказательство:}\\
    Берём $\varepsilon = 1$. Тогда $\exists N : \forall n, m \geq N \,\,
    \rho(x_n, x_m) < 1 \Rightarrow$\\
    $\Rightarrow \forall n \geq N \,\, \rho(x_n, x_N) < 1
    \Leftrightarrow x_n \in B_1(x_N)$\\
    $R := \max\{\rho(x_1, x_N), \rho(x_2, x_N), \dots, 
    \rho(x_{N-1}, x_N)\} \Rightarrow \forall n \,\, x_n \in B_R(x_N)$

    \item Если у фундаментальной последовательности есть сходящаяся
    подпоследовательность, то фундаментальная последовательность
    имеет тот же предел.

    \textbf{Доказательство:}\\
    Пусть $\lim x_{n_k} = a$. Зафиксируем $\varepsilon > 0$.\\
    $\exists K : \forall k \geq K \quad \rho(x_k, a) <
    \frac{1}{2} \varepsilon$\\
    $\exists N : \forall n, m \geq N \quad \rho(x_n, x_m) <
    \frac{1}{2} \varepsilon$\\
    Возьмём $N \geq 0$ и подберём такое $k$, что $k \geq N$\\
    и $n_k \geq N$ (например, $k \geq \max{N, K}$ подходит)\\
    Тогда $\rho(x_n, x_{n_k}) < \frac{1}{2} \varepsilon$
    (т.к. $n_k \geq N$)\\
    И тогда $\rho(x_{n_k}, a) < \frac{1}{2} \varepsilon$
    (т.к. $k \geq K$)\\
    $\Rightarrow \rho(x_n, a) \leq \rho(x_n, x_{n_k}) +
    \rho(x_{n_k}, a) < \varepsilon \Rightarrow \lim x_n = a$
\end{enumerate}

\begin{theorem}Критерий Коши\end{theorem}
Числовая последовательность имеет предел $\Leftrightarrow$
она фундаментальна.

\begin{proof} $ $

''$\Longrightarrow$'':\\
По свойству 1.

''$\Longleftarrow$'':\\
фундаментальность $\xRightarrow[]{\text{св-во 2}}$
ограниченность $\xRightarrow[]
{\text{Больцано–Вейерштрасса}}$\\
$\begin{rcases*}
    \Rightarrow \text{сущ. сходящаяся подпосл.}\\
    \quad\quad\text{фундаментальность}
\end{rcases*}$
$\xRightarrow[]{\text{св-во 3}}$ существует конечный предел.

\end{proof}

\section{Теорема Больцано–Вейерштрасса в $\mathbb{R}^d$.
Полнота $\mathbb{R}^d$ }

\begin{conj}
Полнота метрического простраства
\end{conj}
Пусть $(X, \rho)$ -- метрическое пространство.
$X$ - полное, если любая фундаментальная последовательность
в нём имеет предел.

\begin{theorem}
    $\mathbb{R}^d$ - полное пространство.
\end{theorem}
\begin{proof} $ $

    Возьмём фундаментальную последовательность $(x_n)$.
    $x_n = (x_n^{(1)}, x_n^{(2)}, \dots, x_n^{(d)})$

    \[\forall \varepsilon > 0 \,\, \exists N :
    \forall n, m \geq N \,\, \rho(x_n, x_m) <
    \varepsilon \Rightarrow\]
    \[\Rightarrow \abs{x_n^{(k)} -
    x_m^{(k)}} \leq \sqrt{(x_n^{(1)} -
    x_m^{(1)})^2 + (x_n^{(2)} - x_m^{(2)})^2 +
    \dots + (x_n^{(d)} - x_m^{(d)})^2} < \varepsilon
    \Rightarrow\] \[ \Rightarrow
    \text{числовая послед. } x_n^{(k)}
    \text{ фундаментальна } \Rightarrow \text
    {у неё есть конечный предел}\] \[\lim x_n^{(k)}
    = a_k \Rightarrow \lim x_n = a, \quad a = 
    (a_1, a_2, \dots, a_d) \]
    Т.к. в $\mathbb{R}^d$ покоординатная и
    сходимость по метрике -- одно и то же.

\end{proof}

\begin{theorem}
Больцано–Вейерштрасса в $\mathbb{R}^d$.
\end{theorem}
\begin{proof}
Пусть векторная последовательность
$x_n = (x_n^{(1)}, x_n^{(2)}, \dots, x_n^{(d)})$ ограничена.
Это равносильно тому, что все её координатные последовательности
ограничены.

Выделим из первой координатной последовательности сходящуюся
подпоследовательность $(x_{n_{1, k}}^{(1)})$. Тогда получим
подпоследовательность $(x_{n_{1, k}})$, первая координатная
последовательность которой сходится, а остальные ограничены.

Тогда в ней можно выделить такую подпоследовательность
$(x_{n_{2, k}})$ так, чтобы вторая координатная последовательность
сходилась.

Повторим так ещё $d - 2$ раз и получим то, что в векторной 
подпоследовательности $(x_{n_{k}})$, где $n_k = n_{d, k}$, 
любая координатная последовательность сходится $\Rightarrow$
$(x_{n_{k}})$ тоже сходится, т.к. в $\mathbb{R}^d$ покоординатная и
сходимость по метрике -- одно и то же.

\end{proof}


\section{Верхний и нижний пределы. Определение и
теорема существования. Связь между частичными пределами и  
верхним и нижним пределами.}

\begin{conj}
Нижний и верхний пределы
\end{conj}
$x_n$ - числовая последовательность.

$\underline{\lim} x_n := \liminf x_n := \lim \inf_{k \geq n} x_k$ -- 
нижний предел.

$\overline{\lim} x_n := \limsup x_n := \lim \sup_{k \geq n} x_k$ -- 
верхний предел.

$y_n := \inf_{k \geq n} x_k = \inf\{x_n, x_{n+1}, x_{n+2}, \dots\}$
$\quad y_n \leq y_{n+1}$\\
$z_n := \sup_{k \geq n} x_k = \sup\{x_n, x_{n+1}, x_{n+2}, \dots\}$
$\quad z_n \geq z_{n+1}$

\begin{theorem}
    $\underline{\lim}$ и $\overline{\lim}$ существуют в 
    $\overline{\mathbb{R}}$ и $\underline{\lim} \leq \overline{\lim}$
\end{theorem}
\begin{proof} $ $

Про $\underline{\lim}$: $y_n \leq y_{n+1} \Rightarrow (y_n)$ --
возрастающая последовательность $\Rightarrow$ у неё есть предел в
$\overline{\mathbb{R}}$.

Про $\overline{\lim}$: $z_n \geq z_{n+1} \Rightarrow (z_n)$ --
убывающая последовательность $\Rightarrow$ у неё есть предел в
$\overline{\mathbb{R}}$.

Про неравенство $\underline{\lim} \leq \overline{\lim}$:
$y_n \leq z_n$, $y_n \rightarrow \underline{\lim}$,
$z_n \rightarrow \overline{\lim} \Rightarrow$ по предельному
переходу в неравенстве $\underline{\lim} \leq \overline{\lim}$.
\end{proof}

\begin{theorem}\end{theorem}
\begin{enumerate}
    \item $\overline{\lim}$ -- наибольший частичный предел
    \item $\underline{\lim}$ -- наименьший частичный предел
    \item $\exists \lim \in \overline{\mathbb{R}} \Leftrightarrow
    \overline{\lim} = \underline{\lim}$ и в этом случае 
    $\lim = \overline{\lim} = \underline{\lim}$ 
\end{enumerate}
\begin{proof} $ $

    \begin{enumerate}
        \item $a := \overline{\lim} \, x_n$
        
        Рассмотрим \textbf{случай} $a \in \mathbb{R}$

        Докажем, что $a$ -- частичный предел.

        $a = \lim z_n$, $z_n = \sup_{k \geq n} x_k$, $z_n \searrow a$

        Будем строить некоторую подпоследовательность $(x_{n_k})$.\\
        Найдётся $n_k \geq n_{k-1} : x_{n_k} > a - \frac{1}{k}$.
        Пусть не нашлось $\Rightarrow x_n \leq a - \frac{1}{k} \forall
        n \geq n_{k-1} \Rightarrow \sup \{x_{n_{k-1}}, x_{n_{k-1} + 1},
        \dots \} \leq a - \frac{1}{k} \Rightarrow a \leq z_{n_{k - 1}} 
        \leq a - \frac{1}{k}$. Противоречие

        $a - \frac{1}{k} \rightarrow a$, $z_{n_k} \rightarrow a$,
        $a - \frac{1}{k} < x_{n_k} \leq z_{n_k} \xRightarrow[]
        {\text{2 мил.}} x_{n_k} \rightarrow a$

        Докажем, что $a$ -- наибольший частичный предел.

        Пусть $b$ - частичный предел $\Rightarrow b = \lim x_{n_k}$.
        Но $x_{n_k} \rightarrow b$, $z_{n_k} \rightarrow a \Rightarrow$
        по предельному переходу $b \leq a$. 

        Рассмотрим \textbf{случай} $a = -\infty$.

        Тогда $z_n \rightarrow -\infty$, но $z_n = 
        \sup\{x_n, x_{n+1},\dots\} \geq x_n \Rightarrow x_n
        \rightarrow -\infty$.

        Рассмотрим \textbf{случай} $a = +\infty$.

        Тогда $z_n = +\infty \Rightarrow \sup {x_1, x_2, \dots} =
        +\infty \Rightarrow x_n$ не ограничена сверху $\Rightarrow$
        в ней найдётся подпоследовательность, стремящаяся к $+\infty$.

        \item Доказывается аналогично
        
        \item 
        ''$\Longrightarrow$'':
        
        Если $a = \lim x_n$, то все подпоследовательности стремятся
        к $a \Rightarrow$ все частичные пределы равны $a \Rightarrow
        \overline{\lim} x_n = \underline{\lim} x_n = \lim x_n = a$.

        ''$\Longleftarrow$'':

        $y_n \rightarrow a$, $z_n \rightarrow a$, $y_n \leq x_n \leq z_n$
        $\xRightarrow[]{\text{2 мил.}} x_n \rightarrow a \Rightarrow
        \lim x_n = \overline{\lim}\, x_n = \underline{\lim}\, x_n = a$
    \end{enumerate}
\end{proof}

\textbf{Замечание.} Арифметики для верхних и нижних пределов нет.

Пример.
\[x_n = (-1)^n, \quad y_n = (-1)^{n + 1} \Rightarrow
\underline{\lim}\, x_n = \underline{\lim}\, y_n = -1\]
\[x_n + y_n = 0 \Rightarrow \underline{\lim}\, (x_n + y_n) = 
\underline{\lim}\, (x_n + y_n) = 0\]
\[\underline{\lim}\, x_n + \underline{\lim}\, y_n = -2 < 0 =
\underline{\lim}\, (x_n + y_n)\]

\section{Характеристика верхних и нижних пределов с помощью 
$N$ и $\varepsilon$. Сохранение неравенств.}

\begin{theorem}\end{theorem}
\begin{enumerate}
    \item $a = \underline{\lim}\, x_n \in \mathbb{R} \Leftrightarrow
    \begin{cases}
        \forall \varepsilon > 0 \,\, \exists N : \forall n \geq N
        \,\, x_n > a - \varepsilon\\
        \forall \varepsilon > 0 \,\, \forall N \,\, \exists n \geq N
        : x_n < a + \varepsilon\\
    \end{cases}$
    \item $b = \overline{\lim}\, x_n \in \mathbb{R} \Leftrightarrow
    \begin{cases}
        \forall \varepsilon > 0 \,\, \exists N : \forall n \geq N
        \,\, x_n < b + \varepsilon \quad \circled{1}\\
        \forall \varepsilon > 0 \,\, \forall N \,\, \exists n \geq N
        : x_n > b - \varepsilon \quad \circled{2}\\
    \end{cases}$
\end{enumerate}
\begin{proof} $ $

    2. Докажем \circled{1} $\Leftrightarrow \forall \varepsilon > 0\,\, 
    \exists N : z_N < b + \varepsilon$

    ''$\Longrightarrow$'':
    \[\forall \varepsilon > 0 \,\, \exists N : \forall n \geq N
    \,\, x_n < b + \varepsilon \Rightarrow
    \forall \varepsilon > 0 \,\, \exists N : \forall n \geq N
    \,\, x_n < b + \frac{\varepsilon}{2} \Rightarrow\] \[\Rightarrow
    z_N = \sup \{x_N, x_{N + 1}, \dots\} \leq b + \frac{\varepsilon}{2}
    < b + \varepsilon \Rightarrow \forall \varepsilon > 0\,\, 
    \exists N : z_N < b + \varepsilon\]

    ''$\Longleftarrow$'':
    \[\text{Зафиксируем } \varepsilon > 0
    \Rightarrow \exists N : z_N < b + \varepsilon \Leftrightarrow
    \sup\{x_N, x_{N + 1}, \dots\} < b + \varepsilon \Rightarrow
    x_n < b + \varepsilon \,\, \forall n \geq N\]

    Докажем \circled{2} $\Leftrightarrow \forall \varepsilon > 0 \,\,
    \forall N \,\, z_N > b - \varepsilon$

    ''$\Longrightarrow$'':
    \[\forall \varepsilon > 0 \,\, \forall N \,\, \exists n \geq N
    : x_n > b - \varepsilon \text{ при этом } z_N = \sup\{
    x_N, x_{N+1}, x_{N+2}, \dots\} \Rightarrow \forall
    \varepsilon > 0 \,\, \forall N \,\, z_N > b - \varepsilon\]

    ''$\Longleftarrow$'':
    \[\text{Зафиксируем } \varepsilon > 0 \text{ и } N \Rightarrow
    z_N > b - \varepsilon \Leftrightarrow \sup\{ x_N, x_{N+1}, \dots\}
    > b - \varepsilon \Rightarrow \exists n \geq N : x_n > b -
    \varepsilon,\] \[\text{иначе } \forall n \geq N : x_n \leq b -
    \varepsilon \text{ и тогда } \sup\{ x_N, x_{N+1}, \dots\} \leq
    b - \varepsilon \Leftrightarrow z_N \leq b - \varepsilon\]

    \circled{1} + \circled{2} $\Leftrightarrow
    \begin{cases}
        \forall \varepsilon > 0\,\, 
        \exists N : z_N < b + \varepsilon\\
        \forall \varepsilon > 0 \,\,
        \forall N \,\, z_N > b - \varepsilon\\
    \end{cases}$
    $\Leftrightarrow$ т.к. $z_n \searrow$
    $\begin{cases}
        \forall \varepsilon > 0\,\, 
        \exists N : \forall n \geq N \,\, z_n < b + \varepsilon\\
        \forall \varepsilon > 0 \,\,
        \forall N \,\, z_N > b - \varepsilon\\
    \end{cases}$

    Это и есть определение предела $\Rightarrow b = 
    \overline{\lim}\, x_n$

    В обратную сторону, первая строка следует из определения предела,
    вторая строка следует из того, что $(z_n) \searrow$. Более того,
    $(z_n) \searrow$, $\lim z_n = b \Rightarrow z_n \geq b$

\end{proof}

\begin{theorem}\end{theorem}
Если $x_n \leq y_n$, то 
$\underline{\lim}\, x_n \leq \underline{\lim}\, y_n$ и  
$\overline{\lim}\, x_n \leq \overline{\lim}\, y_n$

\begin{proof} $ $

$x_n \leq y_n \Rightarrow \inf\{x_n, x_{n + 1}, ...\} \leq
\inf\{y_n, y_{n + 1}, ...\} \Rightarrow$ по пред. переходу 
$\underline{\lim}\, x_n \leq \underline{\lim}\, y_n$

Аналогично для $\overline{\lim}\, x_n \leq \overline{\lim}\, y_n$.
\end{proof}

\section{Сходимость рядов. Необходимое условие сходимости рядов. Примеры.}

\begin{conj}Ряд\end{conj}
$x_n \in \mathbb{R}, \quad \sum_{n=1}^{+\infty} x_n$ -- ряд.

\begin{conj}Частичная сумма ряда\end{conj}
$S_n := \sum_{k=1}^{n} x_k$

\begin{conj}Сумма ряда\end{conj}
Cумма ряда -- $\lim S_n$, если он существует.

\begin{conj}Сходимость ряда\end{conj}
Ряд сходится, если $\exists \lim S_n \in \mathbb{R}$\\
В противном случае ряд расходится.

\begin{theorem}
Необходимое условие сходимости
\end{theorem}
Если $\sum_{n = 1}^{+\infty} x_n$ сходится, то $\lim x_n = 0$.

\begin{proof}
    Если ряд сходится, то $S := \lim S_n \in \mathbb{R}$. Тогда
    $x_n = S_n - S_{n - 1} \Rightarrow \lim x_n = \lim S_n - 
    \lim S_{n - 1} = S - S = 0$
\end{proof}

\textbf{Примеры:}
\begin{enumerate}
    \item Геометрическая прогрессия $1 + q + q^2 + \dots$
    $\sum_{n = 0}^{+\infty} q^n$

    При $\abs{q} < 1$ $S_n = 1 + q + q^2 + \dots + q^{n - 1} =
    \frac{1 - q^n}{1 - q} \rightarrow \frac{1}{1 - q}$

    При $\abs{q} > 1$ ряд расходящийся, т.к. не выполнено
    необходимое условие.

    \item $1 - 1 + 1 - 1 + 1 - 1 + \dots$
    
    $S_{2n} = 0, \,\, S_{2n + 1} = 1 \Rightarrow$ предела нет.

    \item Гармонический ряд $1 + \frac{1}{2} + \frac{1}{3} + \dots$
    $\sum_{n = 1}^{+\infty} \frac{1}{n}$

    $H_n := \sum_{k = 1}{n} \frac{1}{k}$ -- гармонические числа.
    $H_n$ монотонно возрастает.
    \[H_{2^n} = 1 + \frac{1}{2} + \underbrace{\left(\frac{1}{3} + 
    \frac{1}{4}\right)}_{> 2 \cdot \frac{1}{4} = \frac{1}{2}} +
    \underbrace{\left(\frac{1}{5} + \frac{1}{6} + \frac{1}{7} + 
    \frac{1}{8}\right)}_{> 4 \cdot \frac{1}{8} = \frac{1}{2}} +
    \dots + \underbrace{\left(\frac{1}{2^{n - 1} + 1} + 
    \frac{1}{2^{n-1} + 2} + \dots + \frac{1}{2^n} \right)}_
    {> 2^{n-1} \cdot \frac{1}{2^n} = \frac{1}{2}} >\]
    \[> 1 + \underbrace{\frac{1}{2} + \frac{1}{2} + ... + \frac{1}{2}}_
    {n \text{ шт.}} = 1 + \frac{n}{2} \Rightarrow \text{ частичные
    суммы сколь угодно большие } \Rightarrow \lim H_n = +\infty\]

    Гармонический ряд -- расходящийся ряд, члены которого стремятся к $0$.

    \item \[\frac{1}{1 \cdot 2} + \frac{1}{2 \cdot 3} + \frac{1}{3 \cdot 4}
    + \dots \quad\quad\sum_{n=1}^{+\infty} \frac{1}{n\cdot(n+1)}\]
    \[\frac{1}{k\cdot(k+1)} = \frac{1}{k} - \frac{1}{k + 1} \Rightarrow
    S_n = \frac{1}{1 \cdot 2} + \frac{1}{2 \cdot 3} + \frac{1}{3 \cdot 4}
    + \dots + \frac{1}{n\cdot(n+1)} = 1 - \frac{1}{n+1} \rightarrow 1\]

\end{enumerate}

\section{Простейшие свойства сходящихся рядов.}

\begin{enumerate}
    \item Сумма ряда единственна
    
    \begin{proof}
        Утверждение про единственность предела частичных сумм
    \end{proof}

    \item Расстановка скобок не меняет суммы ряда (если она была)
    
    \begin{proof}
        $\underbracket[0pt][2pt]{x_1}_{S_1} + (x_2 + x_3 +
        \underbracket[0pt][2pt]{x_4}_{S_4}) + (x_5 + 
        \underbracket[0pt][2pt]{x_6}_{S_6}) + (x_7 + x_8 + 
        \underbracket[0pt][2pt]{x_9}_{S_9})...$

        Т.е. из последовательности частичных сумм просто выбрали
        другую подпоследовательность, ну таким образом, если предел был,
        то он такой же и остался.
    \end{proof}

    \textbf{Замечание.} Он расстановки скобок сумма ряда могла
    появиться.

    Пример. Ряд $1 - 1 + 1 - 1 + 1 - 1 + 1 - 1 + \dots$ расходится.
    Но при расстановке следующим образом скобок:
    $(1 - 1) + (1 - 1) + (1 - 1) + (1 - 1) + \dots$ получаем, что ряд
    имеет сумму $0$.

    \item Добавление/отбрасывание конечного числа членов не влияет на
    сходимость, но влияет на сумму.

    \begin{proof}
        Рассмотрим отбрасывание.

        Ряд $x_1 + x_2 + x_3 + \dots$, частичная сумма которого
        $S_n$, переделали в $x_{k+1} + x_{k+2} + x_{k+3} + \dots$,
        частичная сумма которого $\widetilde{S}_n := x_{k+1} + x_{k+2}
        + \dots + x_{k+n} = S_{k + n} - S_{k}$. Т.к. $k$ фиксировано
        отсюда видно, что если $S_n$ (не) имеет предел, то и
        $\widetilde{S}_n$ (не) имеет предел, и наоборот.

        Добавление - просто обратная операция.
    \end{proof}

    \item Если $\sum_{n = 1}^{+\infty} a_n$ и $\sum_{n = 1}^{+\infty} b_n$
    сходятся, то $\sum_{n = 1}^{+\infty} (a_n \pm b_n)$ сходится и
    $\sum_{n = 1}^{+\infty} (a_n \pm b_n) =\\= \sum_{n = 1}^{+\infty} a_n
    \pm \sum_{n = 1}^{+\infty} b_n$

    \item Если $\sum_{n = 1}^{+\infty} a_n$ сходится, то
    $\sum_{n = 1}^{+\infty} c a_n$ сходится и $\sum_{n = 1}^{+\infty} 
    c a_n = c \cdot \sum_{n = 1}^{+\infty} a_n$
    \end{enumerate}


\ifdefined\niveldos\else

\end{document} 
\fi
